\subsection{Background and Motiviation}
Upcoming wide field surveys will find a large number of supernovae, and will have the ability to trace large scale structure. The best methods to do so need to be explored. In particular, the interesting methods will probably use both the distance  scales measured by supernovae using their standard candle properties and their properties as tracers of large scale structure simultaneously for the same supernova. One such probe is a probe of peculiar velocity correlations using SNIa, where the standard candle property is used to derive the distance scale, while the measured redshift (spectroscopic or photometric) is used simultaneously to gauge the peculiar velocity of the SNIa (which is potentially the peculiar velocity of the host galaxy). The spatial correlation of such peculiar velocities is related to the matter over-density at the location through the Poisson Equations, and thus can be used to constrain the over-densities and the growth function. This can be used to constrain the cosmology (for example, see \cite{2011PhRvD..83d3004B}). For other computational approaches to similar science cases, such as \cite{2017ApJ...847..128H}.

