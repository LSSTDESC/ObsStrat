

SN cosmology is systematics-limited. Increasing the statistics in the
Hubble diagram must be coupled with (1) advances in the measurement of
the SN distances (i.e.  a control at the per-mil level of the
photometry and survey flux calibration, and possibly a 3-parameter SN
standardization technique) (2) a better control of the SN
astrophysical environment and its potential impact on the SN
light curves and distances (local host properties, absorption) (3) a
better control of the SN diversity (SN~Ia sub-populations, population
drift with redshift) (4) a precise determination of the survey
selection function (SN identification, residual contamination by
non-SN~Ia's as a function of redshift).

Access to spectroscopy will not scale with the large amount of SNe
LSST will deliver. About 10\% of LSST SNe will benefit from a live
spectrum. Securing spectroscopic host redshifts for the full LSST
sample using the fiber spectrographs available in the southern
hemisphere is challenging -- although doable.  As a consequence, all
the studies listed above, in particular SN~Ia identification and the
standardization of SN luminosity distances will rely on the supernova
light curves only. Obtaining high quality SN light curves is
therefore a key design point of the SN survey.  The average quality of
the SN light curves depends exclusively on the observing strategy.

Furthermore, spectroscopic time being scarse, we cannot afford to
waste it.  This means that all transients identified
spectroscopically, around peak luminosity, as SNe~Ia, {\em must}
eventually have light curves of sufficient quality to end up in the
Hubble diagram.  This puts another requirement on the regularity and
predictability of the observing strategy.

Four key facets of observing strategy that have an impact on the number and on the quality of well-measured supernovae may be identified: {\it a regular cadence} (typical values: three to four days) is important to get well-sampled light curves ; minimal inter-night gaps are mandatory to keep a high detection efficiency of the supernovae; the {\it season length} has an impact on the total number of supernovae that may be collected ; 170 to 180 days are values of interest for supernova science (in particular for deep fields); {\it depth} quantified by m5, the five-sigma depth, which is the magnitude corresponding to a flux with a signal-to-noise ratio (SNR) equal to 5. Since only light curve points of well-measured supernovae with SNR $>$5 may be considered, m5 has an impact on the redshift limit of observation (photostatistic limit); {\it spatial coverage and uniformity} which has an impact both on the wide and on the deep surveys; it may be interesting to observe deep fields evenly distributed in Ra (and Galactic/Ecliptic planes avoided) so as to search for anisotropies using individual Hubble diagrams.

In this note are described the studies performed within the DESC-SN working group to assess impacts of observing strategies on supernovae science. An overview of considered observing strategies is given in a first part. The set of proposed supernovae metrics is detailed in a second part. A ranking of considered observing strategies (according to the results collected in the second part) is given in a third section.

