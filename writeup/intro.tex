\section*{Introduction}

The Large Synoptic Survey Telescope (LSST) project is currently
constructing an 8.4m telescope and preparing to use it to conduct a 10
year astronomical imaging survey of the sky. This imaging survey has
four primary science drivers: constraining dark energy and dark matter
parameters, creating an invertory of solar system objects, studying
transient astronomical objects, and mapping the Milky Way. Science
collaborations have formed to prepare for the analysis of LSST data in
support of each of these goals. These include the Dark Energy Science
Collaboration (DESC), formed to advise the LSST project on optimizing
the survey for cosmology, and prepare for and perform cosmological
analysis on this data set.

With a field of view of $9.6\square^\circ$ and an effective aperature
of 6.4 meters, the telescope will have an etendue of
319m^2$\square^\circ$. Fast readout and slew times (totaling an
average of less than 10 seconds per visit, including slews) will
enable it to collect $\sim1000$ exposures per night, enough to image
over $\sfrac{1}{8}$ of the sky twice per night. The instrument will be
equiped with a 5 slot filter changer and a complement of 6 filters
(\textit{u, g, r, i, z,} and \textit{y}). Filters will be switched
into and out of the changer during the day.

The current baseline strategy combines a primary ``wide-fast-deep''
(WFD) survey of $18,000\square^\circ$ (or more), taking 85\%-90\% of
the time, with a collection of ``mini-surveys.'' The WFD survey will
collect images in all 6 filters, and complete and median of $\sim825$
exposures per pointing by the end of the 10 year survey. Of this, at
least $2000\squary^\circ$ will be regularly imaged twice on the same
night, with an interval of $\sim40$ seconds to \sim30$ minutes.

Of the mini-surveys, the one with the most direct relevance to dark
energy is the ``deep drilling field'' (DDF) survey. The DDF will
consist of at least 5 pointings imaged at a shorter cadence (and
correspondingly greater coadded depth) than typical area in the WFD.

Although some parameters of the survey strategy are firmly set, many
details are currently tentative or undetermined.  The LSST project has
created two pieces of software to help develop and test survey
strategy: {\tt OpSim} and {\tt MAF}. {\tt OpSim} is the ``operations
simulator,'' which simulates survey strategies and produces exposure
lists (with metadata) of resultant surveys that would result (given
some assumptions about observing conditions). {\tt MAF} is an analysis
tool for examining the results of such simulations.

In the summer of 2015, the LSST hosted an ``observing strategy
workshop'' at which these tools were presented to the community, and
during which the community and project began writing a white paper
exploring different survey strategies and presenting metrics for
scientific use for a wide range of science projects. An initial
version of this paper, originally entitled ``Science-Driven
Optimization of the LSST Observing Strategy,'' was completed in the
summer of 2017. It is now referred to as the ``Community Observing
Strategy Evaluation Paper'' (COSEP).

The project developed significantly after the 2015 observing
strategy workshop. {\tt OpSim} development has continued, and
additional suggestions for significant alterations in strategy were
proposed. It also became clear that several tradeoffs in observing
strategy required further study. Some of these include target depth in
each band, footprint area, cadence, and observing rules.

The project therefore issue a ``call for white papers'' in June of
2018. The call describes the essential parameters for LSST observing,
give references to collection of {\tt OpSim} simulation results, and
presents a set of questions to the community of astronomers interested
in using LSST data. It also includes a \LaTeX  template for responses.
