\section{Static probes}

\newcommand{\todorm}[1]{\textbf{TO DO: #1}}

Here we present results for joint cosmological forecasts of the ``static'' probes: WL, LSS, and CL.
Unlike Section~\ref{sec:wl}, which focuses specifically on systematics mitigation for weak lensing,
this section focuses on dark energy constraining power.  The forecasting methodology follows
that in the DESC Science Requirements Document v1 \cite{DESCSRD2018}, though with modified
assumptions about the baseline LSST analysis for Y1 and Y10 based on the new strategies, and filling
in Y3 and Y6 forecasts as well. 
\todorm{Say more about the analysis, i.e., what is 3x2-point and how does CL fit in?}

We carried out forecasts for Y1, Y3, Y6, and Y10.  The reason for these choices is as follows: the
Y1 WL, CL, and LSS analysis will already be more powerful than all previous imaging surveys
combined, so first individual and joint-probe results will be produced at that time.  After that
point, there will likely be some regrouping to address whatever systematic uncertainties are
identified as the tallest poles in the Y1 analysis, so the next substantive analysis updates will
likely occur at Y3.  After this point Y6 and Y10 are somewhat arbitrary but suitably-spaced
benchmarks.  We use the Dark Energy Task Force (DETF \todorm{cite this!}) Figure of Merit
(FoM). \todorm{Define this if it hasn't been defined elsewhere in the writeup already.}

Given our adopted analyses, the following factors affect constraining power in the forecasts:
\begin{itemize}
\item Survey area $f_\text{sky}$: roughly speaking, with all other factors fixed, the FoM for WL,
  CL, and LSS scales like $f_\text{sky}^2$.  This assumes that noise terms that have a covariance
  that scales like $f_\text{sky}^{-1}$ dominate.  \todorm{Check scalings.}  A secondary impact of
  $f_\text{sky}$ is to modify the area overlap with spectroscopic surveys such as TiDES/4MOST and
  DESI, which will be used to place priors on photometric redshift biases and scatters
  \todorm{citations!}. The size of those priors therefore depends on $f_\text{sky}$ in some complex
  way related to the modifications in overlap with those two surveys.  However, this is a higher
  order effect that we will neglect in our initial analysis.
\item Survey median depth: Survey homogeneity is an important factor in precision measurements of
  galaxy densities, shears, and photometric redshifts.  We define a survey area after some depth
  cuts that are intended to roughly homogenize the area and eliminate regions with significant
  uncertainties due to dust reddening within the Milky Way.  These cuts are somewhat arbitrarily
  defined and could in principle be tuned for different strategies to maximize constraining power
  given the combination of depth and area factors.  The median survey depths after this cut, and
  their standard deviation, was discussed in the LSS section.  \todorm{Once Humna's text is visible,
    make this connection more clear/direct.}  The impact of the median survey depth is as follows:
  \begin{itemize}
    \item To modify the LSS tracer sample number density, assuming that we allow it to float in some
      way that is tied to the median depth.  The redshift distribution may also change.
    \item To modify the WL source sample number density; this sample is generally defined using a
      SNR cut rather than a strict magnitude cut.  The redshift distribution may also change.
    \item As shown in \cite{DESCSRD2018}, even going from Y1 to Y10, the number density changes are
      substantial while the redshift distribution changes are relatively modest.  Hence within a
      given year (e.g., all Y1 forecasts in this paper when considering all the strategies) we will
      focus on the changes in number density and neglect the differences in LSS and WL sample number
      redshift distribution between the strategies.  This simplifies the forecasting process.
  \end{itemize}
\item Photometric redshift quantity: this is tabulated in \todorm{refer to static tables or photo-z
  section once Humna and Melissa's work is included in the writeup}.  For a given size of prior on
  photo-z scatter and bias, modest changes (10s of percent) in photo-z quality do not have a
  significant impact on the forecasts, so we neglect this in the initial forecasts.
\end{itemize}

The above considerations have informed our approach to forecasting.  In particular, we group the
strategies based on those with similar usable areas and depths after the depth cuts described
above.  We only carry out forecasts for those groups of strategies, defined as follows
(\todorm{Consider turning this into a table, with FoM results and some explanation.}):
\begin{enumerate}
\item Y1: There are four groups of strategies: (1a) `mothra\_2045', (1b) `pontus\_2002', (1c)
  `pontus\_2502', and (1d) everything else.  The DESC SRD Y1 baseline had a median depth of 25.13 in
  $i$-band and an area of 12.3k~deg$^2$.  Group (1d) is quite similar to this in median depth and
  hence in number densities and redshift distributions, but with an area that is 14\% higher.  Group
  (1a) gives an increased depth by 0.4 magnitudes, which will substantially increase the WL and LSS
  number counts and modify their redshift distribution, but has a usable area 45\% below group (1d).
  Group (1b) is 0.2 magnitudes shallower than the DESC SRD Y1 baseline, but has the largest area,
  15.5k~deg$^2$; hence we have to make the LSS and WL sample number densities 20\% and 30\% lower
  than in the DESC SRD Y1 baseline, while keeping the redshift distribution fixed and accounting for
  the larger area.   Group (1c) has a similar depth as (1d) but 15\% smaller area than (1d), so can
  use the same forecasting parameters except for the area.
\item Y3: \todorm{}
\item Y6: \todorm{}
\item Y10: There are three groups of strategies: (10a) `mothra\_2045', (10b) `pontus\_2002' (largest
  area at 19.2k~deg$^2$, but shallowest and hence somewhat worse in terms of photo-z quality), and (10c)
  everything else (typically falling within $\pm 500$~deg$^2$ of 14.5k~deg$^2$).  The DESC SRD Y10
  baseline had a median depth of 26.35 in $i$-band and 14.3k~deg$^2$.  This is similar area to
  (10c), but roughly 0.2 mag shallower in median depth.  Roughly speaking, as mentioned above, we
  expect a modest change in the redshift distribution and no change in area for (10c) compared to
  the DESC SRD Y10 baseline, but the number density could increase by $\sim$20\% for LSS and $30$\%
  for WL.  In contrast, (10a) and (10b) are more similar in depth to the DESC SRD baseline, and
  hence we use the same number densities and redshift distributions, but with 19\% lower and 35\%
  higher areas than the DESC SRD baseline, respectively.
\end{enumerate}

\todorm{Tie results to LSS, WL section results.  Boil it down to simple statements about the
  features of the strategy that matter, rather than about these particular strategies.
  Consider iterative survey depth definition
  depending on the strategy.  Impact of high airmass, unusual seeing distributions, etc.  Impact of
  changes in DESI, 4MOST overlap.  Better quantification of photo-z impact.  Consistency of
  systematics and photo-z calculations with area/depth calculations.  Fitting formula for
  FoM given area and median depth.  Scatter as a function of redshift in the forecasts?}

