\section{Static probes}

\newcommand{\todorm}[1]{\textbf{TO DO: #1}}

Here we present results for joint cosmological forecasts of the ``static'' probes: WL, LSS, and CL.
Unlike Section~\ref{sec:wl}, which focuses specifically on systematics mitigation for weak lensing,
this section focuses on dark energy constraining power.  The forecasting methodology follows
that in the DESC Science Requirements Document v1 \cite{DESCSRD2018}, though with modified
assumptions about the baseline LSST analysis for Y1 and Y10 based on the new strategies, and filling
in Y3 and Y6 forecasts as well. 
\todorm{Say more about the analysis, i.e., what is 3x2-point and how does CL fit in?}

We carried out forecasts for Y1, Y3, Y6, and Y10.  The reason for these choices is as follows: the
Y1 WL, CL, and LSS analysis will already be more powerful than all previous imaging surveys
combined, so first individual and joint-probe results will be produced at that time.  After that
point, there will likely be some regrouping to address whatever systematic uncertainties are
identified as the tallest poles in the Y1 analysis, so the next substantive analysis updates will
likely occur at Y3.  After this point Y6 and Y10 are somewhat arbitrary but suitably-spaced
benchmarks.  We use the Dark Energy Task Force (DETF \todorm{cite this!}) Figure of Merit
(FoM). \todorm{Define the DETF FoM if it hasn't been defined elsewhere in the writeup already.}

Given our adopted analyses, the following factors affect constraining power in the forecasts:
\begin{itemize}
\item Survey area $f_\text{sky}$: roughly speaking, with all other factors fixed, the FoM for WL,
  CL, and LSS scales like $f_\text{sky}^2$.  This assumes that noise terms that have a covariance
  that scales like $f_\text{sky}^{-1}$ dominate.  \todorm{Check scalings.}  A secondary impact of
  $f_\text{sky}$ is to modify the area overlap with spectroscopic surveys such as TiDES/4MOST and
  DESI, which will be used to place priors on photometric redshift biases and scatters
  \todorm{citations!}. The size of those priors therefore depends on $f_\text{sky}$ in some complex
  way related to the modifications in overlap with those two surveys.  However, this is a higher
  order effect that we will neglect in our initial analysis.
\item Survey median depth: Survey homogeneity is an important factor in precision measurements of
  galaxy densities, shears, and photometric redshifts.  We define a survey area after some depth
  cuts that are intended to roughly homogenize the area and eliminate regions with significant
  uncertainties due to dust reddening within the Milky Way.  These cuts are somewhat arbitrarily
  defined and could in principle be tuned for different strategies to maximize constraining power
  given the combination of depth and area factors.  The median survey depths after this cut, and
  their standard deviation, were discussed in the LSS section.
  The impact of the median survey depth is as follows:
  \begin{itemize}
    \item To modify the LSS tracer sample number density, assuming that we allow it to float in some
      way that is tied to the median depth.  The redshift distribution may also change.
    \item To modify the WL source sample number density; this sample is generally defined using a
      SNR cut rather than a strict magnitude cut.  The redshift distribution may also change.
    \item As shown in \cite{DESCSRD2018}, even going from Y1 to Y10, the number density changes are
      substantial while the redshift distribution changes are relatively modest.  Hence within a
      given year (e.g., all Y1 forecasts in this paper when considering all the strategies) we will
      focus on the changes in number density and neglect the differences in LSS and WL sample number
      redshift distribution between the strategies.  This simplifies the forecasting process.
  \end{itemize}
\item Photometric redshift quantity: this is tabulated in \todorm{refer to static tables or photo-z
  section once Humna and Melissa's work is included in the writeup}.  For a given size of prior on
  photo-z scatter and bias, modest changes (10s of percent) in photo-z quality do not have a
  significant impact on the forecasts, so we neglect this in the initial forecasts.
\end{itemize}

The above considerations have informed our approach to forecasting.  We have developed a FoM
emulator that takes the usable area and median depth for a given survey strategy after some cuts to
identify the usable area for 
WL, CL, and LSS, and returns an estimate of the FoM for that strategy with and without Stage III
priors.  The approach to forecasting is described at
\url{https://github.com/LSSTDESC/ObsStrat/tree/static/static#building-an-area--depth-fom-emulation-code}.
\todorm{Translate that Markdown text into paper text.  Provide some simple formulae and a table of
  FoM values for all strategies at Y1, Y3, Y6, and Y10 -- filling in Table~\ref{table:staticfom}.
  Emphasize the main lessons based on the
  emulation process, rather than the specifics for the strategies that are now available, and use
  the table to illustrate those lessons learned.}

\begin{table}[ht]
\caption{Illustration of the WL, LSS, and CL joint FoM values for the strategies that have been
  released so far.  \todorm{Fill in numbers; rank based on one column and explain which one/why.
    Decide if we want to show only without Stage III or also with Stage III.}}
\begin{tabular}{lllll}
\label{table:staticfom}
OpSim run & Y1 FoM & Y3 FoM & Y6 FoM & Y10 FoM \\ \hline
mothra\_2045   &  & & & \\
kraken\_2044   &  & & & \\
nexus\_2097    &  & & & \\
pontus\_2002   &  & & & \\
kraken\_2036   &  & & & \\
pontus\_2502   &  & & & \\
kraken\_2035   &  & & & \\
colossus\_2664 &  & & & \\
kraken\_2026   &  & & & \\
baseline2018a  &  & & & \\
colossus\_2665 &  & & & \\
colossus\_2667 &  & & & \\
kraken\_2042   &  & & & \\
pontus\_2489   &  & & & \\
\end{tabular}
\end{table}



\todorm{Tie results to LSS, WL section results.  Boil it down to simple statements about the
  features of the strategy that matter, rather than about these particular strategies.
  Consider iterative survey depth definition
  depending on the strategy.  Impact of high airmass, unusual seeing distributions, etc.  Impact of
  changes in DESI, 4MOST overlap.  Better quantification of photo-z impact.  Consistency of
  systematics and photo-z calculations with area/depth calculations.  Fitting formula for
  FoM given area and median depth.  Scatter as a function of redshift in the forecasts?}

