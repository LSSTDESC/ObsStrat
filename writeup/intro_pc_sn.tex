\subsection{Background and Motiviation}
\begin{itemize}
\item Large scale structure a study of density fields which can be estimated from the matter detected (eg. counting galaxies etc.) or from peculiar velocities through the Poisson equation. While harder the latter method is independent of light to matter ratio and therefore bias.
\item Peculiar velocities can be measured from galaxies using the redshift and galaxy distances based on different estimators eg. Tully Fisher, Fundamental Plane, etc. However SNIa, while fewer in number, are known to be much better distance estimators. Interesting to study independently.
\item For a survey like LSST which spans many years, the number of SN increases with survey length. 10 years is better than most surveys, good place to try this.
\end{itemize}

\begin{itemize}
    \item signal good at low redshift
    \item uncertainty of v(r, a) proportional to sqrt of number of pairs of SN. 
    \item uncertainty decreases as sqrt(N) ~ sqrt(A) for perfect measurement
    \item Edge effects important if not connected ? Contiguous areas better ?
    \item We look at the cadence, assume we will be characteriing the SN with only LSST and incorporate the uncertainty on distance modulus induced by cadence.
    \item Here we do a quick estimate of the power, using some approximations. These are described below in the methods section
\end{itemize}

