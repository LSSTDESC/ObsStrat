\batchmode
%%\documentclass[onecolumn]{lsstdescnote}
\documentclass[modern]{lsstdescnote}
%%\documentclass[modern,skiphelvet]{lsstdescnote}
\usepackage{lsstdesc_macros}
\usepackage{graphicx}
\usepackage{fancyvrb}
\usepackage{color}
\usepackage{xcolor}
\usepackage{verbatim}
\usepackage{amssymb}
\usepackage{amsmath}
\usepackage{hyperref}
\usepackage{natbib}
\usepackage{caption}
\usepackage{subcaption}
\usepackage{xfrac}

\hypersetup{urlcolor=blue, colorlinks=true} 
\usepackage{float}

%used in SL section
\usepackage{mathptmx}
\usepackage{siunitx} %SI-einheiten
\usepackage{lmodern} %weitere mathematischen Symbole
\usepackage{placeins} %floatbarrier

\DeclareSIUnit\parsec{pc}
\DeclareSIUnit\lightyear{ly}
\DeclareSIUnit\year{yr}
\DeclareSIUnit\erg{erg}
\DeclareSIUnit\ster{ster}
\DeclareSIUnit\arcsec{arcsec}
\DeclareSIUnit\deg{deg}

%% Use TeX's Palatino clone
%% \usepackage{tgpagella}
%% \usepackage[T1]{fontenc}
%% \normalfont

\definecolor{gruen}{cmyk}{0.35,0.01,0.80,0.1}

\graphicspath{{figures/}}

\providecommand{\e}[1]{\ensuremath{\times 10^{#1}}}
\newcommand{\given}[2]{\ensuremath{P(#1|#2)}}
\newcommand{\x}[0]{\ensuremath{\vec{x}}}
\newcommand{\gauss}[3]{\ensuremath{\frac{1}{\sqrt{2\pi#2^2}}\text{exp}\left(- \frac{(#1-#3)^2}{2#2^2}\right)}}

% used in SL section
\newcommand{\todo}[2]{\textcolor{red}{\textbf{TODO (#1): #2}}}

\raggedbottom

% Hack to keep ``modern'' style option from making the TOC with absurdly big line spacing
\AtBeginDocument{\addtocontents{toc}{\protect\setlength{\parskip}{0pt}}}

\begin{document}
\title{LSST DESC Observing Strategy}
\author{Members of the LSST DESC Observing Strategy Taskforce}
\date{\today}

\maketitle

\begin{abstract}
In 2018, the Large Synoptic Survey Telescope (LSST) project issued a
call for whitepapers to inform observing cadence and strategy. In
response to this call, the LSST Dark Energy Science Collaboration
(DESC) constituted an Observing Strategy Task-Force (OSTF) which
developed metrics by which observing strategies could be evaluated for
use studying dark energy using a variety of techniques, and applied
these metrics to a variety of sample simulations of different
strategies. Except for the studies the supernova cosmology groups
(which is maintained separately), internal documentation of these
studies by the different subgroups of the OSTF are collected here.
\end{abstract}

\pagebreak
\tableofcontents

\pagebreak
\section*{Preface}
\addcontentsline{toc}{subsection}{Preface}

The Large Synoptic Survey Telescope (LSST) is an astronomical sky
survey designed to image $25000\square\degree$ of the sky (including
the entire southern hemisphere) in 6 different filters, repeatedly for
10 years. Such a data set will allow creation not only of a catalog of
objects in the covered area, but also the generation of light curves
of variable objects in the covered area. The LSST Dark Energy Science
Collaboration (DESC) formed to advise the LSST project on optimizing
the survey for cosmology, and prepare for and perform cosmological
analysis on this data set.

In the summer of 2018, the LSST project issued a call for white papers
to provide input on the effects of cadence and other survey strategy
considerations on science. Due to the tight constraints on the length
and format of responeses, the DESC Observing Strategy Task Force
(OSTF) composed a collection of supporting documents, with different
intended audiences, levels of refinement, and detail.

This document collects contributions from different analysis groups
intended for internal use, for example as a reference for writing more
refined documentation and as support for future work. It is intended to
be a dynamic document, tracking the active work of the analysis groups
as it progresses.


\pagebreak
\section*{Introduction}

The Large Synoptic Survey Telescope (LSST) project is currently
constructing an 8.4m telescope and preparing to use it to conduct a 10
year astronomical imaging survey of the sky. This imaging survey has
four primary science drivers: constraining dark energy and dark matter
parameters, creating an invertory of solar system objects, studying
transient astronomical objects, and mapping the Milky Way. Science
collaborations have formed to prepare for the analysis of LSST data in
support of each of these goals. These include the Dark Energy Science
Collaboration (DESC), formed to advise the LSST project on optimizing
the survey for cosmology, and prepare for and perform cosmological
analysis on this data set.

With a field of view of $9.6\square^\circ$ and an effective aperature
of 6.4 meters, the telescope will have an etendue of
319m^2$\square^\circ$. Fast readout and slew times (totaling an
average of less than 10 seconds per visit, including slews) will
enable it to collect $\sim1000$ exposures per night, enough to image
over $\sfrac{1}{8}$ of the sky twice per night. The instrument will be
equiped with a 5 slot filter changer and a complement of 6 filters
(\textit{u, g, r, i, z,} and \textit{y}). Filters will be switched
into and out of the changer during the day.

The current baseline strategy combines a primary ``wide-fast-deep''
(WFD) survey of $18,000\square^\circ$ (or more), taking 85\%-90\% of
the time, with a collection of ``mini-surveys.'' The WFD survey will
collect images in all 6 filters, and complete and median of $\sim825$
exposures per pointing by the end of the 10 year survey. Of this, at
least $2000\squary^\circ$ will be regularly imaged twice on the same
night, with an interval of $\sim40$ seconds to \sim30$ minutes.

Of the mini-surveys, the one with the most direct relevance to dark
energy is the ``deep drilling field'' (DDF) survey. The DDF will
consist of at least 5 pointings imaged at a shorter cadence (and
correspondingly greater coadded depth) than typical area in the WFD.

Although some parameters of the survey strategy are firmly set, many
details are currently tentative or undetermined.  The LSST project has
created two pieces of software to help develop and test survey
strategy: {\tt OpSim} and {\tt MAF}. {\tt OpSim} is the ``operations
simulator,'' which simulates survey strategies and produces exposure
lists (with metadata) of resultant surveys that would result (given
some assumptions about observing conditions). {\tt MAF} is an analysis
tool for examining the results of such simulations.

In the summer of 2015, the LSST hosted an ``observing strategy
workshop'' at which these tools were presented to the community, and
during which the community and project began writing a white paper
exploring different survey strategies and presenting metrics for
scientific use for a wide range of science projects. An initial
version of this paper, originally entitled ``Science-Driven
Optimization of the LSST Observing Strategy,'' was completed in the
summer of 2017. It is now referred to as the ``Community Observing
Strategy Evaluation Paper'' (COSEP).

The project developed significantly after the 2015 observing
strategy workshop. {\tt OpSim} development has continued, and
additional suggestions for significant alterations in strategy were
proposed. It also became clear that several tradeoffs in observing
strategy required further study. Some of these include target depth in
each band, footprint area, cadence, and observing rules.

The project therefore issue a ``call for white papers'' in June of
2018. The call describes the essential parameters for LSST observing,
give references to collection of {\tt OpSim} simulation results, and
presents a set of questions to the community of astronomers interested
in using LSST data. It also includes a \LaTeX  template for responses.


\pagebreak
\section{Dark Matter}
\subsubsection{Microlensing: search for intermediate-mass Galactic black holes}
\emph{Authors: Marc Moniez}
    \begin{itemize}
    \item Introduction
    
    The gravitational microlensing effect is the temporary magnification of a source
when a massive compact object passes close enough to its line of sight.
A review of the microlensing formalism can be found in \cite{j2006} and \cite{Rahvar_2015}.
Assuming a single point-like lens of mass $M$ located at distance $D_L$ is deflecting the
light from a single point-like source located at distance $D_S$, the magnification $A(t)$
of the source luminosity as a function of time $t$ is given by \cite{Paczynski_1986} :
%\cite{Schneider}:
\begin{equation}
\label{magnification}
A(t)=\frac{u(t)^2+2}{u(t)\sqrt{u(t)^2+4}}\ ,
\end{equation}
where $u(t)$ is the distance of the lensing object to the undeflected line of sight, divided by
the Einstein radius $R_{\mathrm{E}}$ :
%\begin{eqnarray}
%R_{\mathrm{E}} &=& \sqrt{\frac{4GM}{c^2}D_S x(1-x)} \\
%&\simeq& 4.54\ \mathrm{AU}.\left[\frac{M}{\Msol}\right]^{\frac{1}{2}}
%\left[\frac{D_S}{10 kpc}\right]^{\frac{1}{2}}
%\frac{\left[x(1-x)\right]^{\frac{1}{2}}}{0.5}, \nonumber
%\end{eqnarray}
\begin{equation}
R_{\mathrm{E}}\!\! =\!\! \sqrt{\frac{4GM}{c^2}D_S x(1-x)}
\simeq\! 4.54\ \mathrm{AU}.\left[\frac{M}{M_\odot}\right]^{\frac{1}{2}}\!
\left[\frac{D_S}{10 kpc}\right]^{\frac{1}{2}}\!\!
\frac{\left[x(1-x)\right]^{\frac{1}{2}}}{0.5}, \nonumber
\end{equation}
$G$ is the Newtonian gravitational constant, and $x = D_L/D_S$.
Assuming a lens moving at a constant relative transverse
velocity $v_T$, reaching its minimum
distance $u_0$ (impact parameter) to the undeflected line of sight
at time $t_0$, $u(t)$ is given by $u(t)=\sqrt{u_0^2+(t-t_0)^2/t_{\mathrm{E}}^2}$,
%\begin{equation}
%\label{impact}
%u(t)=\sqrt{u_0^2+\left( \frac{t-t_0}{t_{\mathrm{E}}}\right)^2},
%\end{equation}
where $t_{\mathrm{E}}=R_{\mathrm{E}} /v_T$ is the lensing timescale:
\begin{eqnarray}
t_{\mathrm{E}} \sim
79\ \mathrm{days} \times %\\
\left[\frac{v_T}{100\, km/s}\right]^{-1}
\left[\frac{M}{M_\odot}\right]^{\frac{1}{2}}
\left[\frac{D_S}{10\, kpc}\right]^{\frac{1}{2}}
\frac{[x(1-x)]^{\frac{1}{2}}}{0.5}\; . %\nonumber
\end{eqnarray}
The so-called simple microlensing effect (point-like source and lens
with rectilinear motions) has the following characteristic
features: 
Given the low probability of the alignment,
the event should be singular in the history of the source
(as well as of the deflector);
the magnification, independent of the color,
is a simple function of time
depending only on ($u_0, t_0, t_{\mathrm{E}}$),
with a symmetrical shape;
as the source and the deflector are independent,
the prior distribution of the events' impact parameters must be uniform;
all stars at the same given distance have the same probability to be lensed;
therefore the sample of lensed stars should be representative
of the monitored population at that distance, particularly with respect to
the observed color and magnitude distributions.

    \item Search for long-timescale events
    
        The past and present microlensing surveys all suffered from a drastic decline of the
detection efficiency for events with durations $t_{\mathrm{E}}$ larger than a few years, which are expected from massive black hole lenses ($M>10.M_\odot$). This is the reason why the published limits on the contribution of compact objects to the Galactic dark matter are not constraining beyond this mass. With 10 years of continuous observations, the light-curves measured by LSST will enable us to reach the sensitivy either to detect black holes up to $1000.M_\odot$ and measure their Galactic density, or to exclude their contribution to a significant fraction of the Galactic hidden matter.
The main condition to succeed in this task is to ensure a time-sampling that spans the entire LSST survey duration and avoids very long gaps within the light-curves (apart the unavoidable inter-seasonal gaps). The final efficiency to long timescale events will not be sensitive to the details of the cadencing, as long as gaps longer the half a year are avoided.

Synergies with the past and present databases have also to be seriously considered as the best way to confirm or not microlensing candidates.

    \item Search for short-timescale events
    
    The previous conclusion does not apply when searching for shorter timescale events (few days to few months), since a sparse sampling makes such research inefficient. Nevertheless, if the objective is to perform optical depth measurements, then a non uniform, variable (adaptative) sampling should be appropriate to probe all timescales. But the search for short timescale events will only be efficient during a small fraction of the LSST schedule, and a systematic harvest --as liked by the planet hunters-- doesn't seem practicable with LSST alone. The ultimate efficiency in this case should be studied in collaboration with followup setups triggered by the LSST alert system.
    \end{itemize}  

\pagebreak

\newcommand{\md}{\mathrm{d}}

\section{Gravitational Waves}

\subsection{Serendipitous Detections of Kilonovae}
\textit{Authors: Christian N. Setzer\footnote{christian.setzer@fysik.su.se}, Rahul Biswas, Hiranya V. Peiris}\newline
\newline
The Large Synoptic Survey Telescope (LSST) will detect millions of transients on a nightly basis \citep{LSSTScienceCollaboration2009}. This is expected to include many rare transients, such as the electromagnetic signal from the merger of two neutron stars, known as kilonovae (KNe). The coincident detection of gravitational and electromagnetic waves in 2017, an event designated GW170817, gave us the first evidence that such kilonovae events exist \citep{TheLIGOScientificCollaboration2017, Cowperthwaite2017}. This confirmed long-standing predictions of multi-messenger signals from binary neutron star mergers \citep{Li1998}. With this single event, it was possible to obtain information on cosmological parameters, yielding a measurement of the Hubble Constant: $H_0 = 70.0^{+12.0}_{-8.0} \mathrm{km\,s^{-1}\,Mpc^{-1}}$ \citep{Abbott2017}.

Given the luminosity of GW170817, and the predicted peak values of the optical/infrared after-glow from theory, we expect to detect such events with LSST at greater distances than possible with current GW detectors \citep{Chen2017a}. What are we able to do with detections of kilonovae if we only obtain the electromagnetic counterpart? To investigate this question, it is useful to estimate how many of these events we would be able to detect. Due to the rapid evolution, short lifetime and lower luminosity of these events, we expect the number of detectable events to be sensitive to many properties of the cadence \citep{LSSTScienceCollaboration2017}. Below we investigate the impact of different LSST cadence strategies on the sample sizes of serendipitous KNe detections.

\subsubsection{Modeling}
To simulate the electromagnetic signal from KNe we separately consider two models. The only known observation of a kilonova is GW170817. To begin predicting observations, assuming the evolution of the spectral energy distribution (SED) of other kilonovae are identical to this event is a reasonable first approximation. To characterize that event we use the time-series SED, provided by the Dark Energy Survey (DES), which was used in the analysis of \citet{Scolnic2017a}. This model uses multi-band photometry to calibrate a spectral time-series to observations, using photometry from \citet{Soares-Santos2017} and
\citet{Cowperthwaite2017}.

Based on the diversity of other electromagnetic transients, it would be naive to assume that GW170817 is representative of the electromagnetic properties of the entire range of neutron star mergers. Thus, we adopt an alternative model that is physically motivated and spans a parameter space describing a population of KNe. We have chosen a semi-analytic eigenmode expansion (SAEE) model for the SED evolution of KNe \citep{Rosswog2018}. The semi-analytic model, while one-dimensional, uses a sophisticated radiation transport treatment building on the works of \citet{Wollaeger2017} and \citet{Pinto2000}.
\begin{table}[h!]
  \centering
  \begin{tabular}{c|c|c}
    SAEE Model Parameter & Range & Units \\
    \hline
    $\kappa$ & $\mathrm{binomial}[1, 10]$ & $\mathrm{cm^2 g^{-1}}$ \\
    \hline
    $\mathrm{m_{ej}}$ & $[0.01, 0.2]$ & $\mathrm{M_{\odot}}$ \\
    \hline
    $\mathrm{v_{ej}}$ & $[0.01, \, 0.5 (\mathrm{m_{ej}}/0.01)^{-\mathrm{log}_{20}(2)}]$ & $c$
  \end{tabular}
  \caption{The space of parameters describing the population of KNe in the SAEE model.}
  \label{tab: ross_params}
\end{table}

This model uses three parameters: the gray opacity ($\kappa$); the median ejecta mass ($\mathrm{m_{ej}}$); and the median ejecta velocity ($\mathrm{v_{ej}}$), to generate the time-series SED evolution for a KN event. The space spanned by these parameters is shown in Table \ref{tab: ross_params}; also see Fig. \ref{fig: ross_params}. For $\mathrm{v_{ej}}$, this bound was based on the condition that the total energy of the event remained under $10^{52}\, \mathrm{ergs}$. This was an upper bound adopted in the exploration of this parameter space with numerical hydrodynamics simulations; see Fig. 4 of \citet{Rosswog2016a}. The binomial distribution on the gray opacity is a standard assumption in the numerical simulation community; there is a high degree of uncertainty in this parameter for KNe \citep{Rosswog2018, Kasen2013}.

\begin{figure}[t!]
  \centering
  \includegraphics[scale=0.58]{Rosswog_parameter_dist}
  \caption{The parameter space of the two continuously varying parameters which characterize the population of KNe of the SAEE model. The intersection of the red lines shows where the event GW170817 falls within the parameter space for a best fit to the infrared photometric light curves,($ \mathrm{v_{ej}} = 0.15\,c\,; m_{\mathrm{ej}} = 0.06 \, M_{\odot}\,; \kappa =10 \, \mathrm{cm^2 /g}$) as determined by \citet{Rosswog2018}.}\label{fig: ross_params}
\end{figure}

To simulate observations of KNe, we begin with the number of transient events, $N_{\mathrm{total}}$, per comoving volume, $V_{\mathrm{co}}$, per event rest-frame time, $t_{\mathrm{rest}}$, which we call the comoving event rate density:

\begin{equation}\label{eqn:erd}
   \Gamma_{\mathrm{co}} =\frac{\md N_{\mathrm{total}}}{\md V_{\mathrm{co}}\, \md t_{\mathrm{rest}}}\,.
\end{equation}

This rate, $\Gamma_{\mathrm{co}}$, is assumed to be constant. We now wish to compute the redshift distribution of such events in the observer's frame and can define the distribution of events in redshift space, $n_{\mathrm{events}}$, and the cumulative number of events, $N_{\mathrm{total}}$, observed out to some redshift. These two quantities are related by

\begin{equation}\label{eqn:total_dist}
N_{\mathrm{total}}(z) = \int_0^{z} n_{\mathrm{events}}(z) \, \md z\,.
\end{equation}
Further, from Eq. \ref{eqn:erd}, we obtain
\begin{equation}
N_{\mathrm{ total}}(z) = \int_0^{T (z)} \int_0^{V_{\mathrm{co}}(z)}  \Gamma_{\mathrm{co}} \, \md V_{\mathrm{co}}(z) \,\md t_{\mathrm{rest}} \, ,
\end{equation}

where $T(z)$ is the total time elapsed in the rest frame at redshift $z$. In terms of quantities directly relatable to known properties of the observations (observation time-interval, $\Delta T_{\mathrm{obs}}$, redshift, $z$, and the sky area, $\Delta \Omega_{\mathrm{obs}}$) we find the redshift distribution

\begin{align}\label{eqn:reddist}
n_{\mathrm{events}}(z) = \, c^3  \frac{\Gamma_{\mathrm{co}}}{(1+z)H(z)} \left[\int^{z}_0 \frac{\md z'}{H(z')} \right]^2 \, \Delta T_{\mathrm{obs}} \, \Delta \Omega_{\mathrm{obs}}\,.
\end{align}

The number of observed events then follows a Poisson distribution, where $k$ are the recorded counts, and the cumulative distribution function equal to

\begin{equation}\label{eqn:invcum}
P(k;N_{\mathrm{total}}(z)) =e^{-N_{\mathrm{ total}}(z)} \sum_{i=0}^k \frac{N_{\mathrm{ total}}(z)^i}{i!}\,.
\end{equation}

Now, choosing a maximum redshift, $z_{\mathrm{max}}$, and, computing $N_{\mathrm{ total}}$, we obtain a realization of the observed events. As this is a discrete distribution, we must use an approximation of inverse of the cumulative probability distribution Eq. \ref{eqn:invcum}. This is done by randomly drawing from a uniform unit interval distribution, $u\in(0,1]$, which represents evaluations of the cumulative distribution function. Then we find the value of $k$ which gives the closest value to our draw, $u$, from the uniform distribution. This $k$ corresponds to the realization of the total number of events from the Poisson distribution, $N_{\mathrm{ total}}^{\mathrm{realization}}$. With this realization it is now possible to build a realization of the redshift distribution.

Given the cumulative redshift distribution of events, $N_{\mathrm{total}}(z)$, we scale this to the total number of events $N_{\mathrm{total}}(z_{\mathrm{max}})$. This yields a cumulative probability distribution function of transient events as a function of redshift, $F(z) = N_{\mathrm{total}}(z)/ N_{\mathrm{total}}(z_{\mathrm{max}})$. Next, we draw from a uniform unit interval distribution, $u_i\in(0,1]$, where $i$ runs from $1$ to $N_{\mathrm{ total}}^{\mathrm{realization}}$. Then, using the inverse of the cumulative distribution function, we map these draws from the uniform distribution to the redshifts which generate these values, $F^{-1}(u_i) \rightarrow z_i$. This produces a set of redshifts for all the transient events that occur within the observer frame represented by $\Delta \Omega_{\mathrm{obs}}$, $\Delta T_{\mathrm{obs}}$, and the redshift range, $z \in [0,z_{\mathrm{max}}]$.

\subsubsection{Simulations}
With the number of events and their redshift distribution calculated, we place the KNe randomly, with a uniform prior in right ascension and declination within the declination band that covers the entire LSST sky footprint for a given cadence. Additionally, the time of explosion for each event is randomly chosen with a uniform prior over the survey lifetime. This includes some buffer at the start of the survey to include events that would only overlap at the end of their evolution. Then, given a choice of KNe model, we generate a time-series SED for each event. Depending on the pairing of redshift, right ascension, and declination that it has been assigned, the SED is redshifted, dimmed, and modified with dust extinction according to the extinction map $E(B-V)$ and LSST per-band-filter corrections from
\citet{Schlafly2011}. Each object is also assigned a peculiar velocity which Doppler-shifts the SED in the event rest frame, such that $1 + z_{\mathrm{obs}} = (1+z_{\mathrm{cosmo}})(1+z_{\mathrm{pec}})$. A summary of these parameter choices is presented in Table \ref{tab: sim_params}.

\begin{table}[h!]
\centering
\resizebox{\columnwidth}{!}{%
  \begin{tabular}{c|c|c}
  Simulation Parameter & Values & Note \\
  \hline
$z$ & $[0, 0.5]$ & $z_{\mathrm{max}}$ at Einstein Telescope Sensitivity \citep{Chen2017a}. \\
 \hline
$R_{\mathrm{KN}}$ & $1000 \mathrm{Gpc^{-3} \,yr^{-1}}$ & Rate used in \citep{Scolnic2017a}. \\
\hline
Dust Map & Per-band-filter extinction & SFD \citep{Schlafly2011} \\
\hline
Peculiar Velocity & Gaussian($\mu = 0; \sigma = 300 \mathrm{km\, s^{-1}}$) & `Typical' values \citep{Davis2010}.
  \end{tabular}
}
 \caption{Parameters used to generate the simulated LSST kilonovae observations.}\label{tab: sim_params}
\end{table}

With this transient distribution in place, the cadence specified by an OpSim database is applied as a filter to make mock observations of each event. For a given transient, all pointings that overlap with the event's spatial location and lifetime are found. Using OpSim information such as the magnitude five-sigma limiting depth for the band of the simulated LSST pointing and assuming a circular field-of-view geometry, the instrument-measured flux and many other properties of the observations are computed. This set of simulations is filtered through a second stage given by a set of detection criteria below.

\subsubsection{Metric}
The metric we have chosen for evaluation of a cadence is the number of KNe that are detected given the simulations of observations described above. The criteria that we use to determine detections is that of \citet{Scolnic2017a}. These criteria are the following:
\begin{itemize}
  \item Two alerts separated by $\geq 30$ minutes.
  \item Observations in at least two filters with $\mathrm{SNR} \geq 5$.
  \item Observations with $\mathrm{SNR} \geq 5$ separated by maximum 25 days.
  \item A minimum of one observation of the same location within 20 days before the first $\mathrm{SNR} \geq 5$ observation.
  \item A minimum of one observation of the same location within 20 days after the last $\mathrm{SNR} \geq 5$ observation.
\end{itemize}

In this context an alert is an observation that has a measured signal-to-noise ratio greater than five after template subtraction. This is simulated using the per-filter efficiency vs. true signal-to-noise ratio response function. As this function is unknown for LSST prior to operation, we have used the set of functions that were adopted for the DES Y1 analysis, given their filter-set is similar to LSST [R. Kessler, private communication]. The observations are filtered according to these criteria and the subset of transients which pass all conditions are labeled as detections.

These are the detection results for kilonovae using a single realization of the redshift distribution for each cadence. Each of these is itself a draw from a Poisson distribution as explained above. The uncertainty on these values is then the square root of the sample mean. For the case of one sample, this mean is the number of detections that are found per cadence, and the uncertainty on the number of detections is the square root of the numbers of detections.
\begin{figure}[h!]
  \centering
  \includegraphics[scale=0.79]{figures/wfd_detection_counts_by_cadence}
  \caption{Ranking of cadence strategies for the WFD part of the survey for both KNe models, based on the number of detections found in each cadence.}
  \label{fig:cadence_ranking}
\end{figure}
\subsubsection{Results}
We rank the cadences for each KNe model based on the number of detections which pass our detections metric; see Fig. \ref{fig:cadence_ranking} and \ref{fig:cadence_ranking_ddf}. We immediately see that the DES-GW model predicts a larger number of detections than the SAEE population model. This is in part due to the early blue component, included in the DES-GW SED, being absent from the SAEE KNe model \citep{Villar2017b}. Furthermore, confirming the conclusion of \citet{Scolnic2017a}, the majority of our detections will come from the wide-fast-deep (WFD) region of the survey as opposed to the deep-drilling fields (DDF). The typical redshift range of detected KNe, as shown in Fig. \ref{fig:typical_nz}, is between $0 \leq z \leq 0.2$, though some detections can extend out to $z \approx 0.3$. While this varies between cadences and models, this range is quite typical for all the cadences considered. The effect of changing KNe model can also be seen in Fig. \ref{fig:typical_nz}. It shows that at all redshifts the detection numbers are slightly suppressed for the SAEE model in comparison to the DES-GW model.

\begin{figure}[h!]
  \centering
  \includegraphics[scale=0.79]{figures/ddf_detection_counts_by_cadence}
  \caption{Same as the above figure, but for the DDF.}
  \label{fig:cadence_ranking_ddf}
\end{figure}

\begin{figure}[h!]
  \centering
  \includegraphics[scale=0.82]{figures/both_nz_base_kraken_2026}
  \caption{Example redshift distribution of detected KNe within the sample of all simulated KNe. This distribution is from the unofficial project baseline cadence \protect \textit{kraken 2026} for both the DES-GW and the SAEE KNe models. The design sensitivities of future GW detectors to detect typical binary neutron star mergers are shown by vertical lines \citep{Chen2017a, Chamberlain2017, Scolnic2017a}}
  \label{fig:typical_nz}
\end{figure}

\FloatBarrier

\pagebreak
\batchmode
\documentclass[a4paper,10pt]{article}
\usepackage{graphicx}
\usepackage{color}
\usepackage{hyperref}
\usepackage{nameref}
\usepackage{fancyvrb}
\hypersetup{urlcolor=red, colorlinks=true, linkcolor=blue}
\usepackage{float}

\textheight=25.5cm
\textwidth=17.5cm
\voffset=0.cm
\hoffset=-0.0cm
\oddsidemargin -1cm
\evensidemargin -1cm
\topmargin -2cm
\baselineskip=0.900cm
\setlength{\parindent}{0in}

\graphicspath{{../figures/}}

\newcommand{\ttt}[1]{\texttt{#1}}
\newcommand{\cl}[1]{\textcolor{magenta}{#1}}

\title{LSS: Summary}
\author{Humna Awan, Eric Gawiser}
\date{\today}

\begin{document}
\maketitle
%%%%%%%%%%%%%%%%%%%%%%%%%%%%%%%%%%%%%%%%%%%%%%%%%%%%%%%%%%%
\section*{Summary: Metrics' Interplay}

Large scale structure studies require some important features: survey uniformity over an area that maximizes the usable survey footprint with deep photometric data. To check the survey uniformity, we consider the \nameref{translational dithers} Metric which illustrates the need for large translational dithers. Using the translationally dithered survey, we then use the \nameref{area} metric to define an extragalactic footprint that will give us the galaxy samples we need for our science. We then consider the \nameref{median depth} and \nameref{depth std} Metrics to asses the survey depth and survey uniformity in our footprint. These feed into \nameref{OS systematics} Metric which allows us to assess the impacts of the artifacts that are induced by the observing strategy. \\

Also, in order to maximize the usability of WFD for extragalactic science, we propose a reconfigured WFD footprint, avoiding the high-extinction regions in the sky but still covering the nominal 18,000 deg$^2$. In order to ensure that \nameref{median visits} Metric results are agreeable in this proposed footprint, we have requested an \ttt{OpSim} to get a better sense of coverage we can expect. \\

Furthermore, in order to minimize systematics, we consider the \nameref{rotational dithers} Metric for rotational uniformity and \nameref{4MOST+DESI overlap} Metric to ensure that we optimize the WFD footprint to get the spectroscopic sample we need for our photo-z calibrations. Finally, we consider the \nameref{seeing} Metric in order asses the impacts of variations in seeing as it impacts the size and depth of the sample we can expect to see.

%%%%%%%%%%%%%%%%%%%%%%%%%%%%%%%%%%%%%%%%%%%%%%%%%%%%%%%%%%%
\newpage
\section*{Median Number of Visits Per Field (After Y10)\label{median visits}}
This metric ensures that we are meeting the LSST SRD requirement that the median number of WFD visits per field after Y10 is at least 825.

\begin{minipage}{\columnwidth}
\vspace*{2em}
\centering
\includegraphics[width=.9\columnwidth]{lss_compare_median_nvisits_11dbs.png}
\vspace*{2em}
\end{minipage}

\cl{Will add the rest of the cadences (i.e., the \ttt{alt\_sched} and \ttt{FBS} outputs) but need to create a \ttt{Stacker} for \ttt{fieldIDs} since the new schedulers don't have \ttt{fieldIDs} associated with visits.}


%%%%%%%%%%%%%%%%%%%%%%%%%%%%%%%%%%%%%%%%%%%%%%%%%%%%%%%%%%%
\newpage
\section*{Rotational Uniformity\label{rotational dithers}}
This metric looks at the distributions of \ttt{rotTelPos} and \ttt{rotSkyPos} as we'd like them to be uniform. The nominal pointings leads a pile-up in \ttt{rotTelPos} distribution at +/-90, 0 degrees. We see that rotational dithers help achieve more uniform distributions (and we understand the pile-up at 0 deg in the dithered \ttt{rotTelPos} distribution: there are visits that do not get dithered under the current rotational dithering scheme).

\begin{minipage}{\columnwidth}
\vspace*{2em}
\centering
\includegraphics[width=\columnwidth]{lss_compare_rotDiths_15dbs.png}
\vspace*{2em}
\end{minipage}
The plot here shows the distributions for WFD visits. An analog analysis can be done for DDF ones.

\cl{The right column will be updated once we have per night rotational dithers. Hopefully it'll fix the pile-up at zero in the dithered \ttt{rotTelPos} distribution (and consequently the \ttt{rotSkyPos} distribution). Also, the outputs from new schedulers need to be added.}

%%%%%%%%%%%%%%%%%%%%%%%%%%%%%%%%%%%%%%%%%%%%%%%%%%%%%%%%%%%
\newpage
\section*{Translational Uniformity\label{translational dithers}}
This metric looks at the need for (large) translational dithers. An undithered survey not only leads to survey non-uniformity but also a comparatively shallower survey. We showed in Awan+16 that translational dithers help with both issues.

\begin{minipage}{\columnwidth}
\vspace*{2em}
\centering
\includegraphics[width=.75\columnwidth]{lss_compare_depth_median_15dbs_undith.png}
\vspace*{2em}
\end{minipage}

\begin{minipage}{\columnwidth}
\vspace*{2em}
\centering
\includegraphics[width=.75\columnwidth]{lss_compare_depth_std_15dbs_undith.png}
\vspace*{2em}
\end{minipage}

Here, the dithered survey implemented per night, random translational dithers, as large as the LSST FOV. Also, the depth statistics are for the extragalactic WFD footprint.

%%%%%%%%%%%%%%%%%%%%%%%%%%%%%%%%%%%%%%%%%%%%%%%%%%%%%%%%%%%
\newpage
\section*{Extragalactic Footprint\label{area}}
This metric looks at the usable WFD area for extragalactic science, which we achieve by implementing an extinction cut and a depth cut on the coadded depth footprint.  Specifically, we retain only the area with E(B-V)$<$0.2 with limiting $i$-band coadded 5$\sigma$ depth of 24.5 for Y1, 25.0 for Y3, 25.5 for Y6, and 26.0 for Y10.

\begin{minipage}{\columnwidth}
\vspace*{2em}
\centering
 \includegraphics[width=.75\columnwidth]{lss_compare_area_22dbs.png}
\vspace*{2em}
\end{minipage}

\cl{This figure will be updated once we cater our depth cuts to each cadence}.

%%%%%%%%%%%%%%%%%%%%%%%%%%%%%%%%%%%%%%%%%%%%%%%%%%%%%%%%%%%
\newpage
\section*{Median Depth in the Extragalactic Footprint\label{median depth}}
This metric looks at the median depth in the extragalactic footprint. We would like to go deeper to probe fainter galaxy samples to get better constraining power. Here, we look at the $i$-band coadded 5$\sigma$ depth after Y1, 3, 6, 10.

\begin{minipage}{\columnwidth}
\vspace*{2em}
\centering
 \includegraphics[width=.75\columnwidth]{lss_compare_depth_median_22dbs.png}
\vspace*{2em}
\end{minipage}

\cl{This figure will be updated once we cater our depth cuts to each cadence}.

%%%%%%%%%%%%%%%%%%%%%%%%%%%%%%%%%%%%%%%%%%%%%%%%%%%%%%%%%%%
\newpage
\section*{Depth Uniformity in the Extragalactic Footprint\label{depth std}}
This metric looks at the depth uniformity in the extragalactic footprint. Here, we model this as the standard deviation in the $i$-band coadded 5$\sigma$ depth in the extragalactic footprint.  We would like to minimize non-uniformity to minimize window function uncertainties. 

\begin{minipage}{\columnwidth}
\vspace*{2em}
\centering
 \includegraphics[width=.75\columnwidth]{lss_compare_depth_std_22dbs.png}
\vspace*{2em}
\end{minipage}

\cl{This figure will be updated once we cater our depth cuts to each cadence}.

%%%%%%%%%%%%%%%%%%%%%%%%%%%%%%%%%%%%%%%%%%%%%%%%%%%%%%%%%%%
\newpage 
\section*{Impacts of Artificial Structure\label{OS systematics}}
This metric looks at the effectiveness of each cadence in minimizing the uncertainties in the artificial structure that is induced the observing strategy. It is an implementation of Equation 9.4 in LSST Observing Strategy Community White Paper.

\cl{This figure should be ready by 10/21/18}.

\begin{minipage}{\columnwidth}
\vspace*{2em}
\centering
% \includegraphics[width=.75\columnwidth]{lss_compare_4MOSToverlap_22dbs.png}
\vspace*{2em}
\end{minipage}

%%%%%%%%%%%%%%%%%%%%%%%%%%%%%%%%%%%%%%%%%%%%%%%%%%%%%%%%%%%
\newpage
\section*{4MOST(+DESI) Overlap\label{4MOST+DESI overlap}}
This metric looks at the overlap between LSST footprint and spectroscopic surveys like 4MOST and DESI.

\begin{minipage}{\columnwidth}
\vspace*{2em}
\centering
 \includegraphics[width=.8\columnwidth]{lss_compare_4MOSToverlap_22dbs.png}
\vspace*{2em}
\end{minipage}

\cl{This figure will be updated since I received the DESI footprint a short while ago. We would like the footprint with either 4MOST or DESI overlap (not necessarily both but potentially a preference for DESI vs 4MOST?). Also, we need to quantify the minimum size (and other properties) of the spectroscopic sample needed for our photo-z calibrations}.

%%%%%%%%%%%%%%%%%%%%%%%%%%%%%%%%%%%%%%%%%%%%%%%%%%%%%%%%%%%
\newpage
\section*{Impacts of Seeing\label{seeing}}
This metric looks at the impacts of seasonal variations in seeing, as implemented by Eric Nielson. It is critical to get realistic seeing implemented in our simulated cadences as ignoring seeing variations leads to inaccurate calculation of the 5$\sigma$ point-source depth which impacts all derived quantities, e.g. the coadded 5$\sigma$ depth.

\begin{minipage}{\columnwidth}
\centering
\vspace*{2em}
\includegraphics[width=.75\columnwidth]{lss_compare_depth_median_10dbs_ow6_ow7_opsim.png}
\vspace*{2em}
\end{minipage}

Here, \ttt{ow6} and \ttt{ow7} identify two different outputs from \ttt{owsee}, using data from different but overlapping years in Pachon.


\end{document}

\pagebreak
\section{Photometric Redshifts}\label{sec:pz}

A one page summary of our analysis of the {\tt OpSim} and {\tt ALTSched} results is provided in Section \ref{mlg_sec:pz_summary}. A more in-depth evaluation of how the photometric redshifts derived from LSST photometry will evolve over the 10 year survey, for a variety of observing strategies that includes the formal {\tt OpSim} runs along with more general scenarios in which e.g., 5\% of the total observing time is redistributed to or from each filter in turn, is currently underway and being documented in this Overleaf draft: \url{https://www.overleaf.com/read/fgnvddbnrmgk}.


% % % % % % % % % % % % % % % % % % % % % % % % % % % % %
\subsection{Evaluating the Photometric Evolution of the Observing Strategies}\label{ssec:pz_opsim_phot}

To evaluate the photometric quality produced by each of the proposed observing strategies, we use the Metric Analysis Framework (MAF) as follows.

{\bf Slicer:} We use a HEALpix slicer with {\tt nside} $=16$ and a random dither to the field RA and Dec ({\tt latCol = randomDitherFieldPerNightRa} and {\tt lonCol = randomDitherFieldPerNightDec}). This slicer returns metrics in fields with an approximate resolution of $220$ arcminutes ($3.67$ degrees), similar to a single LSST pointing, and the dither helps to smooth out field-to-field variations in a realistic way. The dither option is not needed or applied with simulations from the feature-based scheduler ({\tt rolling\_10yrs}) or {\tt ALTSched}. 

{\bf Metric:} We use {\tt ExgalM5}, which returns the $5{\sigma}$ limiting magnitude for a given filter, corrected for Galactic dust extinction. We require that a given field have observations in all $6$ filters to be considered as conceivably be appropriate for photo-$z$ estimation (and recall that we impose the constraint that a galaxy must be {\it detected} in $griz$ to get a photo-$z$ estimate). We furthermore impose that the Galactic extinction $E(B-V) \leq 0.2$ mag and that the $5{\sigma}$ $i$-band detection limit be at least $24.5/25.0/25.5/26.0$ magnitudes at $1/3/6/10$ years.

{\bf Bundle:} We run the metric and slicer for years $1$, $3$, $6$, and $10$. We only consider images that are obtained as part of the WFD survey component, to avoid depth pockets from mini-surveys or deep drilling fields. We furthermore only consider fields with a Galactic longitude of $60 < l < 300$ degrees {\em or} Galactic latitude of $b>30$ degrees, as would a cosmological science analysis relying on the LSST photo-$z$, but the $E(B-V) \leq 0.2$ mag is actually more constraining than this location constraint.

Using the above slicer, metric, and bundle, we generate a single file per year per {\tt OpSim} or {\tt ALTSched} run, in which each row represents a ``field" that was observed in all $6$ filters as part of the WFD survey, and the columns are right ascension, declination, and the $5{\sigma}$ limiting magnitude in filters $ugrizy$. 

In Figures \ref{fig:maglims} and \ref{fig:maglims_2} we plot the evolution in the median limiting magnitude for each filter (individual plots), as a function of survey year, for the observing strategy simulations discussed in Section \ref{sec:intro}. Figure \ref{fig:maglims} compares {\tt OpSim} runs that change the WFD survey area and/or the visit strategy, while Figure \ref{fig:maglims_2} investigates the impact of adopting a rolling cadence. Additionally, the bottom panel of Figure \ref{fig:maglims_2} compares the baseline runs of {\tt OpSim} and {\tt ALTSched}. As we can see from Figures \ref{fig:maglims} and \ref{fig:maglims_2}, many of the runs produce a photometric evolution that is equivalent to the baseline, {\tt kraken\_2026}, which is always plotted with dark blue circles. 

{\bf Figure \ref{fig:maglims}, top panel --} For runs that change the survey area, it is clear that including the Galactic Plane as part of a $18,000$ square degree WFD survey area ({\tt colossus\_2664}, red triangles) does not appreciably degrade the photometric results. However, extending the survey area at the expense of depth does make a significant impact, whether or not the same-night revisits are executed ({\tt pontus\_2002}, green squares, and {\tt kraken\_2044}, yellow triangles). It appears that the impact to the overall depth is slightly worse if the revisits are done, and so we will only simulate photo-$z$ results for the large WFD option ({\tt pontus\_2002}). 

{\bf Figure \ref{fig:maglims}, bottom panel --} For runs that change the survey's visits, we can see that avoiding revisits ({\tt colossus\_2667}) or snaps ({\tt kraken\_2042}) does not have a significant impact on the median depth evolution, and so we will only simulate photo-$z$ for the ``many visits" option with 20/40 second exposures in filters $grizy$/$u$ ({\tt pontus\_2489}). 

{\bf Figure \ref{fig:maglims_2}, top panel --} For runs that adopt a rolling cadence for the baseline WFD area of $18,000$ square degrees, the effect on the photometric quality is significant: a rolling cadence ({\tt rolling\_10yrs}, magenta pentagons) produces a degraded median depth in nearly all filters at all times including year 10 (compared to the baseline, blue circles). Exactly why this is the case, and why the issue is exacerbated for $y$-band, is unclear at this time -- but may be because {\tt rolling\_10yrs} was pretty good about only taking $y$-band when the full moon was up (i.e., not spending dark time on $y$-band visits)\footnote{From a conversation with Peter Yoachim.}. For this work, we will only simulate photo-$z$ for the option with two declination bands and a 25\% WFD component in a baseline area of $18,000$ square degrees ({\tt rolling\_10yrs}). We do not evaluate the {\tt OpSim} runs that adopt a rolling cadence over the wider PanSTARRs-like $25,000$ square degrees, {\tt mothra\_2049} and {\tt nexus\_2097}, because they suffer from the same bug of missing visits as {\tt pontus\_2502} and {\tt mothra\_2045}.

{\bf Figure \ref{fig:maglims_2}, bottom panel --} We also show the photometric evolution of the {\tt ALTSched} simulations without and with a rolling cadence that alternates between two declination bands (cyan circles and black pentagons, respectively). These differ enough from the {\tt OpSim} baseline run enough that we will simulate photo-$z$ results for both {\tt ALTSched} observing strategies. In particular, the improved depth in all filters with {\tt ALTSched}'s rolling cadence is significant (black pentagons at x-axis year = 1). This is attributed to the fact that the {\tt ALTSched} rolling cadence only observes one declination band in year 1, where as {\tt OpSim}'s {\tt rolling\_10yrs} allows the deprioritized declination band $25\%$ of the baseline number of visits. Furthermore, by year 10 the rolling and non-rolling versions of {\tt ALTSched} have converged at a common final depth in all filters, unlike the {\tt OpSim}'s baseline and rolling cadence (in which the rolling cadence achieves a shallower final result; the underlying cause of this is unclear).

\begin{figure*}
\begin{center}
\includegraphics[width=15cm,trim={0cm 0cm 0cm 0cm}, clip]{figures/maglims_area.png}
\includegraphics[width=15cm,trim={0cm 0cm 0cm 0cm}, clip]{figures/maglims_visits.png}
\caption{The median 5$\sigma$ magnitude limit for all fields with at least one visit in three filters, as a function of survey year, for LSST filters $ugrizy$ (top left to bottom right panel in each 3x2 panel set). The baseline survey, {\tt kraken\_2026}, is shown with blue circles in each set. The magnitude limit evolution over the 10 year survey is demonstrated for OpSim runs that change the survey area (top) or change the visit strategy (bottom). Results for rolling cadences are shown in Figure \ref{fig:maglims_2}. This plot demonstrates that the overall evolution in photometric quality is similar to the baseline for the majority of the OpSim runs. The two runs that differ the most from the baseline are {\tt pontus\_2002}, in which the WFD survey area is extended at the expense of depth per field (green squares, top), and {\tt pontus\_2489}, in which the $u$-band filter is done with $40$-second exposures and all other filters with $20$-second exposures (orange hexagons, bottom). \label{fig:maglims}}
\end{center}
\end{figure*}

\begin{figure*}
\begin{center}
\includegraphics[width=15cm,trim={0cm 0cm 0cm 0cm}, clip]{figures/maglims_roll_3.png}
\includegraphics[width=15cm,trim={0cm 0cm 0cm 0cm}, clip]{figures/maglims_roll_4.png}
\caption{Similar to Figure \ref{fig:maglims}, but for a strategy that adopts a rolling cadence over the baseline WFD area of $18,000$ square degrees: two bands of declination, each prioritized in alternating years, while the deprioritized band gets 25\% of the baseline number of visits. Recall that {\tt pontus\_2502} has a bug that limited the number of visits and is shown only for reference in the top plot (lime green pentagons); {\tt rolling\_10yrs} is the OpSim database used to represent a rolling cadence option in this analysis (magenta pentagons). In the bottom plot, we compare to the results of the {\tt ALTSched}, without and with a rolling cadence (cyan circles and black pentagons, respectively). These two plots demonstrate that the overall evolution in photometric quality for rolling cadence options is not equivalent to the baseline's uniform progression cadence (blue circles). Also keep in mind that the median magnitudes for a rolling cadence with two declination bands only applies to half of the sky area compared to the baseline, which is a drawback of this kind of rolling cadence that is not expressed by this plot. \label{fig:maglims_2}}
\end{center}
\end{figure*}

{\bf Summary --} As demonstrated above, it turns out that many of the {\tt OpSim} runs result in a similar evolution for the median limiting magnitude for each filter as a function of survey year, and thus would not produce different photo-$z$ results, except for these three: {\tt pontus\_2002}, which has a larger WFD survey area of $24700$ square degrees ($-78 < \delta < +18$); {\tt pontus\_2489}, which does $20$-second visits in $grizy$ filters and $40$-second visits in $u$-band, increasing the $u$-band depth and leading to a larger total number of visits in the others filters; and {\tt rolling\_10yrs}, a rolling cadence that alternates between two bands of declination. We will only simulate photo-$z$ results for those three {\tt OpSim} runs, plus the {\tt OpSim} baseline, and the {\tt ALTSched} baseline and rolling cadence.


% % % % % % % % % % % % % % % % % % % % % % % % % % % % %
\subsection{The CMNN Photo-$z$ Estimator}\label{ssec:pz_exp_cmnn}

For this work we have used the color-matched nearest-neighbors (CMNN) photometric redshift estimator from \cite[][herafter G18]{2018AJ....155....1G}. The CMNN should not be taken as representing the ``best" photo-$z$ estimator or the ``official" LSST algorithm -- it is neither of these things. It is a simple algorithm that produces photo-$z$ results of a statistical quality that directly correlates with the photometric quality of the input, and thus is very useful for evaluating the impact on photo-$z$ of any changes to the LSST photometric quality -- such as the {\tt OpSim} runs that we consider in this work.

As described in G18, this estimator uses a training set of galaxies with known redshifts and a test set for which photo-$z$ are to be estimated. For each galaxy in the test set, the estimator first identifies a color-matched subset of training galaxies by calculating the Mahalanobis distance in color-space between the test galaxy and all training-set galaxies. The Mahalanobis distance in this case is the difference between the test- and training-set galaxy color, divided by the photometric error of the test-set galaxy color, summed over all available colors (Equation 1 in G18). Then, a threshold value is applied that defines a ``good" color match. This threshold is set by the percent point function (PPF): for example, for $N_{\rm dof}=5$, PPF$=95$ per cent of all training galaxies consistent with the test galaxy will have $D_M < 11.07$ (where $N_{\rm dof}$, the number of degrees of freedom, is the number of colors). The estimator then chooses one of the color-matched training-set galaxies (e.g. the nearest-neighbor or a random selection), and uses that galaxy's known redshift as the test-set galaxy's photo-$z$ (a "redshift donor"). The uncertainty in the photo-$z$ estimate is taken to be the standard deviation of the true redshifts of training-set galaxies in the color-matched subset. 

Compared to G18, there are some minor differences in how this photo-$z$ estimator was applied in this work. Here, we require that a test-set galaxy be detected in the LSST filters $griz$ and thus have colors $g-r$, $r-i$, and $i-z$ or else a photo-$z$ estimate is not attempted, and we've used a threshold of PPF=$0.68$ to define the color-matched subset of training galaxies. Unlike G18, we choose randomly from the color-matched subset of training galaxies instead of choosing the nearest neighbor. This difference leads to less accurate and less precise photometric redshifts, but by using a random selection in the photo-$z$ estimator, the {\tt OpSims} runs that degrade/enhance the LSST photometry end up having a larger relative impact on the resulting photometric redshifts. This helps with the experiment at hand, which is to assess the relative impacts of different {\tt OpSim} runs, and {\em not} to generate the best possible photo-$z$ results for each {\tt OpSim} run.

As a final note, to accelerate processing time, we have applied both the color and magnitude pre-cuts to the training set, as described in G18. The color cut is fairly benign, but the magnitude pre-cut effectively works as a ``pseudo-prior" by cutting down the training set to the 10\% of training galaxies with an $i$-band magnitude nearest to the test galaxy's $i$-band magnitude. The ``pseudo-prior" may improve accuracy of the photo-$z$ estimate for some, but can also introduce a bias in the results. Since all of the experiments in this work will be looking at {\it relative} changes to the photo-$z$ quality as various inputs are changed, this kind of degradation to the {\it absolute} photo-$z$ quality is acceptable in this case.


% % % % % % % % % % % % % % % % % % % % % % % % % % % % %
\subsection{Simulating LSST Photometry for the Training and Test Catalogs} \label{ssec:pz_exp_cats}

For this work we use the same simulated mock galaxy catalog as used in G18. This catalog contains the ``true" redshift and the ``true" apparent magnitudes in $ugrizy$ for all galaxies. The training and test set galaxies are drawn randomly from this catalog. We use $1\times10^6$ galaxies for the training set, and $5\times10^4$ galaxies for the test set. Justification for the sizes of the test and training sets is provided in G18, but here we note that the size of the test set is adequate to achieve statistical measures in the high-$z$ bins.

For the training set, we simulate observed apparent magnitudes for all galaxies assuming that the $5{\sigma}$ photometric depths in each filter are equivalent to the LSST baseline 10-year survey: $u=26.09,g=27.38,r=27.53,i=26.83,z=26.06,y=24.86$. This is the same as what was used in G18. For the test set, we also simulate observed apparent magnitudes for all galaxies using the $5{\sigma}$ photometric depths in each filter, but these depths change for the different LSST observing strategies that we consider in this work. We randomly assign each test galaxy to one of the simulated ``fields" from our MAF, as described above in Section \ref{ssec:pz_opsim_phot}. For all training- and test-set galaxies, once we know the $5{\sigma}$ photometric depth to apply, we derive the expected photometric error based on each galaxy's ``true" catalog magnitude (using Equation 5 from \citealt{2008arXiv0805.2366I}), and then add a random scatter proportional to this uncertainty to simulate observational uncertainties. This method is described in more depth, with plots of error {\it vs.} apparent magnitude, in G18.

In all of our experiments, both the test and training sets are limiting to $i<25$ mag, so that we are simulating photo-$z$ for samples of ``good" galaxies. Furthermore, in our implementation of the CMNN algorithm, galaxies are required to have at least 3 colors or else a photo-$z$ estimate is not attempted. Thus, our results represent the photo-$z$ quality of a ``gold'' sample of galaxies for cosmological analyses, and do not reflect any impact on the survey area or number of galaxies available for a cosmological analyses -- only the photo-$z$ quality of the ``best'' observed galaxies.


% % % % % % % % % % % % % % % % % % % % % % % % % % % % %
\subsection{Analysis Methodology}\label{ssec:pz_exp_meth}

We take these blurbs from G18 to describe our statistical measures of photo-$z$ quality and the common plot styles that we will use to represent our results. For this analysis we will use only the point estimates of photo-$z$ for each of our test galaxies, although CMNN is capable of returning a full posterior (e.g., Schmidt et al. 2018, in prep.). Although we have included simulated galaxies with $z<0.3$ and $z>3$ in both the test and training sets, our analysis focuses on a cosmological set of galaxies with $0.3 \leq z_{\rm phot} \leq 3$.

{\bf Statistical Measures --} Where $z_{\rm true}$ is the ``true" catalog redshift and $z_{\rm phot}$ is the photo-$z$, the error is $\Delta z_{(1+z)} = (z_{\rm true} - z_{\rm phot})/(1+z_{\rm phot})$. Including a factor of $(1+z)$ in the denominator acts to compensate for larger uncertainties at high-$z$. For all of our results we calculate the robust standard deviation in $\Delta z_{(1+z)}$ as the FWHM of the interquartile range (IQR) divided by $1.349$ ($\sigma_{\rm IQR}$) and the robust bias as the mean value of $\Delta z_{(1+z)}$ in the IQR ($\overline{\Delta z_{\rm(1+z), IQR}}$). We reject catastrophic outliers ($|z_{\rm spec}-z_{\rm phot}| > 1.5$) from the IQR before calculating the standard deviation and bias, {\em which is different from our analyses in past work} such as \cite{2018AJ....155....1G}. We bootstrap our uncertainties on these statistical measures by randomly drawing a subsets and recalculating the statistics 1000 times. Outlier galaxies are identified as those with $\Delta z_{(1+z)} > 3\sigma_{\rm IQR}$ or $\Delta z_{(1+z)} > 0.06$, whichever is {\it larger}, where $\sigma_{\rm IQR}$ is calculated from all galaxies in $0.3 \leq z_{\rm phot} \leq 3.0$ (i.e., outliers are defined globally). The fraction of outlier galaxies that we calculate {\it includes} catastrophic outliers (as defined above).

{\bf Plot Styles --} To visualize our photo-$z$ results we create plots that compare the true $vs.$ photometric redshifts, in which outlier galaxies are typically colored red and a solid line of $z_{\rm true} = z_{\rm phot}$ is drawn to guide the eye. These plots are useful to obtain a global sense of the photo-$z$ quality and the structure in the outliers positions, especially the features that are perpendicular to the $z_{\rm true} = z_{\rm phot}$, which represent photo-$z$ degeneracies caused by the Balmer break passing between filters. We also plot the robust standard deviation, robust bias, and fraction of outliers as a function of $z_{\rm phot}$, typically as a way to directly compare the bulk photo-$z$ results of experiments in which the simulated galaxy photometry has been altered in some way. 


% % % % % % % % % % % % % % % % % % % % % % % % % % % % %
\subsection{Results}\label{ssec:pz_results}

As discussed in Section \ref{ssec:pz_opsim_phot}, only three of the {\tt OpSim} runs result in a significantly different evolution in the median photometric depth of WFD fields compared to the baseline, and so we only simulate photo-$z$ results for the following: 

\begin{itemize}
\item {\tt OpSim} {\tt kraken\_2026}, the baseline run
\item {\tt OpSim} {\tt pontus\_2002}, a larger WFD survey area of $24700$ square degrees ($-78 < \delta < +18$)
\item {\tt OpSim} {\tt pontus\_2489}, $20$-second visits in $grizy$ filters and $40$-second visits in $u$-band, increasing the $u$-band depth and leading to a larger total number of visits in the others filters
\item {\tt OpSim} {\tt rolling_10yrs}, a rolling cadence that alternates between two bands of declination
\item {\tt ALTSched} {\tt altched\_baseline}, the baseline run
\item {\tt ALTSched} {\tt altsched\_rolling}, a rolling cadence that alternates between two bands of declination
\end{itemize}

In Figure \ref{fig:tzpz} we plot the true {\it vs.} the photometric redshifts at years $1$, $3$, and $10$ of the LSST survey, for each of the four {\tt OpSim} runs considered in this work. While the improvement over the years of the survey is easily noticed in these plots (compare columns), very little distinction is visible in the photo-$z$ results for the different {\tt OpSim} runs (compare rows). For this reason, we do not include the {\tt ALTSched} results in this format. The following plots of statistical measures are more useful in this regard.

In Figure \ref{fig:stats_opsim} we compare the statistical measures of robust standard deviation and bias from the IQR (after the rejection of catastrophic outliers), and the fraction of outliers (including catastrophic) as a function of photo-$z$ bin for each of the four considered {\tt OpSim} runs, at $1$ and $10$ years of the LSST survey. We draw the following conclusions from Figure \ref{fig:stats_opsim}. 
\begin{itemize}
\item Extending the survey area to $-78 < \delta < +18$ for a large WFD area of $24,700$ square degrees (green line; {\tt pontus\_2002}) results in degraded photo-$z$ quality compared to the baseline (blue line; {\tt kraken\_2026}). This degradation is most clearly seen in redshift bins $1.5 \lesssim z_{\rm phot} \lesssim 2.2$ at 1 year.
\item The 20/40s "many visits" option that lengthens the $u$-band exposure time to $40$ seconds and shortens it to $20$ seconds in $grizy$ (orange line; {\tt pontus\_2489}) performs similarly to the baseline, but with a small but significant improvement in the standard deviation and fraction of outliers at $z_{\rm phot} \gtrsim 1.5$.
\item The rolling cadence that alternates between two declination bands ({\tt rolling\_10yrs}; magenta line) produces degraded results that are similar to those from extending the total WFD area (green line).
\end{itemize}

In Figure \ref{fig:stats_altsched} we compare the statistical measures of robust standard deviation and bias from the IQR (after the rejection of catastrophic outliers), and the fraction of outliers (including catastrophic) as a function of photo-$z$ bin for the {\tt OpSim} baseline and rolling cadence to the {\tt ALTSched} baseline and rolling cadence. We draw the following conclusions from Figure \ref{fig:stats_altsched}. 
\begin{itemize}
\item The {\tt OpSim} and {\tt ALTSched} baselines deliver equivalent results in terms of photo-$z$ quality.
\item The {\tt ALTSched} rolling cadence delivers the better photo-$z$ at year 1 across all redshift bins than the {\tt OpSim} rolling cadence, or either simulator's baseline. We attribute this to the fact that {\tt ALTSched}'s rolling cadence does not spend any time observing the deprioritized declination band, whereas {\tt OpSim}'s {\tt rolling\_10yrs} allows it $25\%$ of the baseline number of visits.
\item Of these four simulations, the {\tt OpSim} baseline cadence delivers the best photo-$z$ at year 10.
\end{itemize}

In Figure \ref{fig:evol} we chart the evolution in the statistical measures over year of the survey in redshift bins of $0.8 \leq z_{\rm phot} \leq 1.2$  and $1.8 \leq z_{\rm phot} \leq 2.2$. We can see that all four {\tt OpSim} runs progress in a similar fashion {\it except} for the rolling cadence with two declination bands done in alternating years (magenta line; {\tt rolling\_10yrs}), which shows a mild improvement at year 1 (but recall the improved statisics would only apply to half of the sky area). The {\tt ALTSched} rolling cadence shows an even more pronounced improvement at year 1, but by year 10 is consistent with the baseline cadence.

In Table \ref{tab:zbins} we provide the relative robust standard deviation results in two redshift bins, $0.3<z_{\rm phot}<1.5$ and $1.5<z_{\rm phot}<3.0$, for years 1, 3, 6, and 10, for each of the considered {\tt OpSim} and {\tt ALTSched} runs. Note that these redshift bins are not the same as those used for Figure \ref{fig:evol}. These results are reported as {\it relative to} the baseline robust standard deviation for the low-$z_{\rm phot}$ bin at LSST year 10 (i.e., all values of the robust standard deviation are divided by this value). Note that the ``many visits" survey ({\tt pontus\_2489}) {\it performs better than the {\tt OpSim} baseline in the high-$z$ bin for all years}, and has an equivalent performance to the baseline in the low-$z$ bin.

\begin{figure}
\begin{center}
\includegraphics[width=4.5cm,trim={0cm 0cm 0cm 0cm},clip]{figures/tzpz_kraken2026_1.png}
\includegraphics[width=4.5cm,trim={0cm 0cm 0cm 0cm},clip]{figures/tzpz_kraken2026_3.png}
\includegraphics[width=4.5cm,trim={0cm 0cm 0cm 0cm},clip]{figures/tzpz_kraken2026_10.png}
\includegraphics[width=4.5cm,trim={0cm 0cm 0cm 0cm},clip]{figures/tzpz_pontus2002_1.png}
\includegraphics[width=4.5cm,trim={0cm 0cm 0cm 0cm},clip]{figures/tzpz_pontus2002_3.png}
\includegraphics[width=4.5cm,trim={0cm 0cm 0cm 0cm},clip]{figures/tzpz_pontus2002_10.png}
\includegraphics[width=4.5cm,trim={0cm 0cm 0cm 0cm},clip]{figures/tzpz_pontus2489_1.png}
\includegraphics[width=4.5cm,trim={0cm 0cm 0cm 0cm},clip]{figures/tzpz_pontus2489_3.png}
\includegraphics[width=4.5cm,trim={0cm 0cm 0cm 0cm},clip]{figures/tzpz_pontus2489_10.png}
\includegraphics[width=4.5cm,trim={0cm 0cm 0cm 0cm},clip]{figures/tzpz_rolling10yrs_1.png}
\includegraphics[width=4.5cm,trim={0cm 0cm 0cm 0cm},clip]{figures/tzpz_rolling10yrs_3.png}
\includegraphics[width=4.5cm,trim={0cm 0cm 0cm 0cm},clip]{figures/tzpz_rolling10yrs_10.png}
\caption{True {\it vs.} photometric redshifts for all test galaxies, with statistical outliers plotted as red points. Columns from left to right represent results at years 1, 3, and 10. Rows from top to bottom represent results from the {\tt OpSim} baseline ({\tt kraken\_2026}), large WFD area with declinations including  $-78<\delta<+18$ degrees ({\tt pontus\_2002}), the ``many visits" strategy with 40/20 second exposures in $u$/$grizy$ filters ({\tt pontus\_2489}), and a rolling cadence with two declination bands ({\tt rolling\_10yrs}). Results from {\tt ALTSched} are not plotted in this way. \label{fig:tzpz}}
\end{center}
\end{figure}

\begin{figure}
\begin{center}
\includegraphics[width=6cm,trim={0cm 0cm 0cm 0cm},clip]{figures/year1_IQRs.png}
\includegraphics[width=6cm,trim={0cm 0cm 0cm 0cm},clip]{figures/year10_IQRs.png}
\includegraphics[width=6cm,trim={0cm 0cm 0cm 0cm},clip]{figures/year1_bias.png}
\includegraphics[width=6cm,trim={0cm 0cm 0cm 0cm},clip]{figures/year10_bias.png}
\includegraphics[width=6cm,trim={0cm 0cm 0cm 0cm},clip]{figures/year1_fout.png}
\includegraphics[width=6cm,trim={0cm 0cm 0cm 0cm},clip]{figures/year10_fout.png}
\caption{Statistical measures of robust standard deviation (from the IQR; top), robust bias (middle), and fraction of outliers (bottom) as a function of photo-$z$, at years $1$ (left column) and $10$ (right column). Note that the $y$-axes are not matched between columns. Results are presented for the baseline ({\tt kraken\_2026}; blue), large WFD area with declinations including  $-78<\delta<+18$ degrees ({\tt pontus\_2002}; green), the ``many visits" strategy with 40/20 second exposures in $u$/$grizy$ filters ({\tt pontus\_2489}; orange), and a rolling cadence with two declination bands ({\tt rolling\_10yrs}; magenta). Horizontal dashed line represents the LSST Science Requirement Document's target for a photo-$z$ bin of $0.3 \leq z_{\rm phot} \leq 3.0$., and horizontal colored bars represent the statistical for a photo-$z$ bin of $0.3 \leq z_{\rm phot} \leq 3.0$. \label{fig:stats_opsim}}
\end{center}
\end{figure}

\begin{figure}
\begin{center}
\includegraphics[width=6cm,trim={0cm 0cm 0cm 0cm},clip]{figures/ALTyear1_IQRs.png}
\includegraphics[width=6cm,trim={0cm 0cm 0cm 0cm},clip]{figures/ALTyear10_IQRs.png}
\includegraphics[width=6cm,trim={0cm 0cm 0cm 0cm},clip]{figures/ALTyear1_bias.png}
\includegraphics[width=6cm,trim={0cm 0cm 0cm 0cm},clip]{figures/ALTyear10_bias.png}
\includegraphics[width=6cm,trim={0cm 0cm 0cm 0cm},clip]{figures/ALTyear1_fout.png}
\includegraphics[width=6cm,trim={0cm 0cm 0cm 0cm},clip]{figures/ALTyear10_fout.png}
\caption{Statistical measures of robust standard deviation (from the IQR; top), robust bias (middle), and fraction of outliers (bottom) as a function of photo-$z$, at years $1$ (left column) and $10$ (right column). Note that the $y$-axes are not matched between columns. Results are presented for the {\tt OpSim} baseline ({\tt kraken\_2026}; blue) and rolling cadence ({\tt rolling\_10yrs}; magenta), and the {\tt ALTSched} baseline ({\tt altsched\_baseline}; cyan) and rolling cadence ({\tt altsched\_rolling}; black). Horizontal dashed line represents the LSST Science Requirement Document's target for a photo-$z$ bin of $0.3 \leq z_{\rm phot} \leq 3.0$., and horizontal colored bars represent the statistical for a photo-$z$ bin of $0.3 \leq z_{\rm phot} \leq 3.0$. \label{fig:stats_altsched}}
\end{center}
\end{figure}

\begin{figure}
\begin{center}
\includegraphics[width=6cm,trim={0cm 0cm 0cm 0cm},clip]{figures/zbin1_IQRs.png}
\includegraphics[width=6cm,trim={0cm 0cm 0cm 0cm},clip]{figures/zbin2_IQRs.png}
\includegraphics[width=6cm,trim={0cm 0cm 0cm 0cm},clip]{figures/zbin1_bias.png}
\includegraphics[width=6cm,trim={0cm 0cm 0cm 0cm},clip]{figures/zbin2_bias.png}
\includegraphics[width=6cm,trim={0cm 0cm 0cm 0cm},clip]{figures/zbin1_fout.png}
\includegraphics[width=6cm,trim={0cm 0cm 0cm 0cm},clip]{figures/zbin2_fout.png}
\caption{Statistical measures of robust standard deviation (from the IQR; top), robust bias (middle), and fraction of outliers (bottom) as a function of the LSST survey year, for bins of photometric redshift $0.8 \leq z_{\rm phot} \leq 1.2$ (left column) and  $1.8 \leq z_{\rm phot} \leq 2.2$ (right column). Note that the $y$-axes are not matched between columns. Results are presented for the baseline ({\tt kraken\_2026}; blue), large WFD area with declinations including  $-78<\delta<+18$ degrees ({\tt pontus\_2002}; green), the ``many visits" strategy with 40/20 second exposures in $u$/$grizy$ filters ({\tt pontus\_2489}; orange), a rolling cadence with two declination bands ({\tt rolling\_10yrs}; magenta), and the {\tt ALTSched} results for a baseline (cyan) and rolling (black) cadence. Horizontal dashed line represents the LSST Science Requirement Document's target for a photo-$z$ bin of $0.3 \leq z_{\rm phot} \leq 3.0$. \label{fig:evol}}
\end{center}
\end{figure}

\begin{table}\label{tab:zbins}
\caption{Relative Robust Standard Deviation of the Simulated Photometric Redshifts}
\begin{center}
\begin{tabular}{|l|cc|cc|cc|cc|}
\hline
{\tt OpSim}/{\tt ALTSched} Run & \multicolumn{2}{|c|}{Year 1} & \multicolumn{2}{|c|}{Year 3} & \multicolumn{2}{|c|}{Year 6} & \multicolumn{2}{|c|}{Year 10} \\ 
\multicolumn{1}{|r|}{$z_{\rm phot}$ Bin:} & 0.3--1.5 & 1.5--3.0 & 0.3--1.5 & 1.5--3.0 & 0.3--1.5 & 1.5--3.0 & 0.3--1.5 & 1.5--3.0 \\
\hline
baseline (kraken\_2026)                 & 1.95 & 4.07 & 1.42 & 2.58 & 1.15 & 2.05 & {\bf 1.00} & 1.83 \\ 
large WFD (pontus\_2002)                & 2.04 & 4.83 & 1.54 & 3.13 & 1.28 & 2.37 & 1.10 & 1.98 \\
many visits (pontus\_2489)              & 1.96 & 4.03 & 1.42 & 2.48 & 1.17 & 2.03 & 1.02 & 1.73 \\
$2$ $\delta$-band roll (rolling\_10yrs) & 1.87 & 3.92 & 1.53 & 3.04 & 1.24 & 2.29 & 1.10 & 1.98 \\
{\tt ALTSched} baseline                       & 1.98 & 4.35 & 1.52 & 2.89 & 1.25 & 2.28 & 1.11 & 1.95 \\
{\tt ALTSched} rolling                        & 1.69 & 3.38 & 1.55 & 3.08 & 1.26 & 2.35 & 1.11 & 1.93 \\
\hline
\end{tabular}
\end{center}
\end{table}


% % % % % % % % % % % % % % % % % % % % % % % % % % % % %
\subsection{Summary of Key Results} \label{ssec:pz_execsum}

\noindent
1. Most of the {\tt OpSim} and {\tt ALTSched} observing strategies result in a similar evolution of the median photometric depth in each filter, except: (a) when the WFD area is extended to $24,700$ square degrees (degraded photometric depth); (b) when 20/40 second visits are used for $grizy$/$u$ filters (improved photometric depth in $u$-band); or (c) when a rolling cadence is adopted (improved photometric depth of observed area in year 1). Photo-$z$ results are only simulated for these strategies that deliver a different photometric depth compared to the baseline cadence.

\medskip \noindent
2. The observing strategy in which 20/40 second visits are used for $grizy$/$u$ is the only one which exhibits improvements to the photo-$z$ quality over the baseline cadence option.

\medskip \noindent
3. The {\tt OpSim} and {\tt ALTSched} simulations offer essentially equivalent photo-$z$ results, with the exception that the {\tt ALTSched} rolling cadence delivers a significant improvement in year 1 only. We attribute the latter to the fact that the {\tt ALTSched} rolling cadence does not allocate any visits to the deprioritized declination band, whereas {\tt OpSim}'s rolling cadence allows it $25\%$ of the baseline number of visits.


\pagebreak
% Need these new commands to compile:
%\newcommand{\todo}[2]{\textcolor{red}{\textbf{TODO (#1): #2}}}
%\newcommand{\comment}[1]{\textcolor{blue}{\textbf{#1}}}


\newcommand{\altsched}{\textcolor{magenta}{alt\_sched}} 
\newcommand{\altschedrolling}{\textcolor{blue}{alt\_sched\_rolling}}
\newcommand{\baseline}{\textcolor{orange}{baseline2018a}}
\newcommand{\colossusfour}{\textcolor{orange}{colossus\_2664}}
\newcommand{\colossusfive}{\textcolor{orange}{colossus\_2665}}
\newcommand{\colossusseven}{\textcolor{magenta}{colossus\_2667}}
\newcommand{\krakentwosix}{\textcolor{orange}{kraken\_2026}}
\newcommand{\krakenfive}{\textcolor{orange}{kraken\_2035}}
\newcommand{\krakenthreesix}{\textcolor{blue}{kraken\_2036}}
\newcommand{\krakentwo}{\textcolor{orange}{kraken\_2042}}
\newcommand{\krakenfour}{\textcolor{magenta}{Kraken\_2044}}
\newcommand{\mothrafive}{\textcolor{blue}{mothra\_2045}}
\newcommand{\mothranine}{\textcolor{blue}{Mothra\_2049}}
\newcommand{\nexusseven}{\textcolor{blue}{Nexus\_2097}}
\newcommand{\pontuszerozerotwo}{\textcolor{orange}{Pontus\_2002}}
\newcommand{\pontusfivezerotwo}{\textcolor{orange}{pontus\_2502}}
\newcommand{\pontusnine}{\textcolor{magenta}{pontus\_2489}}
\newcommand{\pontusfivezerosix}{\textcolor{magenta}{pontus\_2506}}
\newcommand{\rollingmixopsim}{\textcolor{magenta}{rolling\_mix\_10yrs\_opsim}}
\newcommand{\rollingopsim}{\textcolor{orange}{rolling\_10yrs\_opsim}}


\section{Strong Lensing}
\textit{Authors: Simon Huber\footnote{shuber@mpa-garching.mpg.de}, Sherry H.~Suyu\footnote{suyu@mpa-garching.mpg.de}, Tanja Petrushevska\footnote{tanja.petrushevska@ung.si}, Timo Anguita, Phil Marshall, Lynne Jones }

\

The Hubble constant $H_0$ is one of the key parameters to describe the
universe. Current observations of the CMB (cosmic microwave
background) assuming a flat $\Lambda$CDM cosmology and the standard
model of particle physics yield $H_0 = 67.36 \pm 0.54 \, {\rm km\,s^{-1}\,Mpc^{-1}}$
\citep{Planck:2018vks} which is in tension with $H_0 =
73.52 \pm
  1.62 \, {\rm km\,s^{-1}\,Mpc^{-1}}$ from the local distance ladder
\citep{Riess:2016jrr,Riess:2018byc}. In order to verify or refute this
$3.6 \sigma$ tension, further independent methods are needed. 

One such method is lensing time delay cosmography which can determine
$H_0$ in a single step. The basic idea is to measure the time delays
between multiple images of a strongly lensed variable source
\citep{Refsdal:1964}. This time delay, in combination with mass
profile reconstruction of the lens and line-of-sight mass structure,
yields directly a ``time-delay distance'' that is inversely
proportional to the Hubble constant ($t \propto D \propto
H_0^{-1}$). Applying this method to four lensed quasar systems, the
H0LiCOW collaboration \citep{Suyu:2016qxx} together with the
COSMOGRAIL collaboration
\citep[e.g.]{Eigenbrod:2005ie,2013Tewes,2017Courbin} measured $H_0 =
72.5^{+2.1}_{-2.3} \,{\rm km\,s^{-1}\,Mpc^{-1}}$ in flat
$\Lambda$CDM \citep{Birrer:2018vtm}, which is in agreement with the
local distance ladder and higher than CMB measurements.  Another
promising approach goes back to the initial idea of
\cite{Refsdal:1964} using lensed supernovae (LSNe) instead of quasars
for time-delay cosmography. Here we investigate the prospects of using
LSST for measuring time delays of both lensed supernovae and lensed
quasars.

\subsection{Supernovae Lensed by Galaxies}
\textit{Contributors: Simon Huber, Sherry H.~Suyu}

\

Even though the number of LSNe is significantly lower than the number of
lensed quasars, LSNe have some important advantages. First 
the sharp rise
and decline of SN light curves make time-delay measurements easier and possible
on shorter time scales in comparison to stochastically varying quasars. Second LSNe Ia are very promising
to break the model degeneracies \citep{Schneider:2013wga} in two
independent ways using dynamics \citep{Barnabe2011,2017:Yildirim} or
the standard candle nature of SNe Ia.  

So far only two systems with resolved
multiple images have been observed, namely SN ``Refsdal''
\citep{Kelly:2015xvu,Kelly:2015vjq} and iPTF16geu
\citep{Goobar:2016uuf}. But LSST will play a key role to detect many
more LSNe. At the moment we expect to find approximately $50$ resolved
LSNe Ia \citep{Oguri:2010} or $900$ in total \citep{Goldstein:2017bny}
over the 10 year survey. No other survey is capable of providing such
high numbers.

The goal of this section is to evaluate different
cadences for LSNe time-delay measurements. For this purpose we have investigated
20 different observing strategies: 15 from the call for whitepapers
(baseline2018a, kraken\_2026, colossus\_2665, pontus\_2002,
colossus\_2664, colossus\_2667, pontus\_2489, kraken\_2035,
mothra\_2045, pontus\_2502,
kraken\_2036, kraken\_2042, kraken\_2044, mothra\_2049, nexus\_2097)\footnote{\url{http://astro-lsst-01.astro.washington.edu:8080/}},
pontus\_2506 which is a cadence from Tiago Ribeiro doing the revisit
after 30 minutes in different filters, alt\_sched and
alt\_sched\_rolling from Daniel Rothchild and collaborators\footnote{\url{http://altsched.rothchild.me:8080/}}, and rolling\_10yrs\_opsim and rolling\_mix\_10yrs\_opsim from Peter Yoachim\footnote{https://github.com/yoachim/SLAIR\_runs}. 

For a better interpretation we subdivide the strategies into three categories: 

(1) \textcolor{orange} {baseline like in terms of inter-night gap and cumulative season length} (mean values), (2) \textcolor{blue}{shorter inter-night gap, shorter cumulative season length} and (3) \textcolor{magenta}{shorter inter-night gap, baseline like cumulative season length}. In addition cadences with a \textbf{large nominal WFD footprint} start with a capital letter.

The mean cumulative season length and mean inter-night gap for the categorization are shown in Figure \ref{fig:cadences categories}. The results are calculated from simulations of the 20 observing strategies by taking the mean of all fields under consideration. The mean cumulative season length is calculated by taking the mean of the summed up season length over all seasons. For the inter-night gap the revisits of a field within hours in the same filter are summarized into a single visit. We look at two different cases. The first case considers 719 LSST fields where all lie in the WFD survey. These fields are picked by taking all fields with center in $\mathrm{Dec} \in [-58,-2] \, \si{\deg}$ and  $\mathrm{RA} \in [0,120] \cup [330,360] \, \si{\deg}$, where all DDFs are excluded. By choice the Galactic Plane, the South Celestial Pole and the Northern Ecliptic are completely excluded and the 719 fields represent only the WFD survey and are used to classify the cadences. For comparison we consider in the second case all 5292 LSST fields covering the entire sky, where we only take into account those fields of the 5292 where observations are taken.


\begin{figure}
\centering
\includegraphics[scale=0.6]{figures/sl_cadences.pdf}
\caption{The mean inter-night gap (upper panel) and mean cumulative season length (lower panel) for 20 different observing strategies for two cases. The first case ("WFD" in solid black) considers 719 LSST fields, which all lie in the WFD survey. The shaded region encloses the 99th percentile of the WFD fields. The second case ("all" in dotted blue) considers all LSST fields (5292) where observations are taken. In the upper panel cadences with the black line below the black dot-dashed line are those with a significantly better inter-night gap than the baseline cadence (magenta and blue strategies). By looking at the lower panel these are subdivided into strategies with a cumulative season length similar to the baseline cadence (magenta) and a significantly worse cumulative season length (blue). Observing strategies in orange have a baseline like mean inter-night gap and mean cumulative season length.}
\label{fig:cadences categories}
\end{figure}

To simulate observations randomly, we have used 202
mock LSNe Ia from the OM 10 catalog \citep{Oguri:2010},
and produced the light curves for the mock SNe images with
the spherically symmetric SN Ia W7 model \citep{1984:Nomoto}
calculated with ARTIS (Applied Radiative Transfer In Supernovae)
\citep{Kromer:2009ce} in combination with magnifications maps from
GERLUMPH \citep{Vernardos:2015wta} to include the effect of
microlensing similar as in \citep{Goldstein:2017bny}. We then simulate
data points for the light curves, following the observation pattern from different cadences
and uncertainties according to the LSST science book
\citep{2009:LSSTscience}. To measure the time delay from the simulated
observation we use the free knot spline optimizer from PyCS (Python
Curve Shifting) \citep{2013:Tewesb,Bonvin:2015jia}. Details of this
work will be presented in Huber et al. (in preparation).


The structure of this subsection is organized as follows. In
\ref{sec:simulation of mock data} we describe how we simulate and
evaluate the mock data and in \ref{sec:results} we present our results
where we have quantified 20 cadences for LSNe Ia.

\subsubsection{Simulating and evaluating mock data}
\label{sec:simulation of mock data}
To simulate mock data for the different cadences we have picked 10
fields in the WFD (wide fast deep survey) which are listed in Table
\ref{tab: 10 wfd fields}. For a given cadence for each of these
fields, we store the following for each visit of the field: date
(mjd), filter(s) observed, and 5-$\sigma$ depth $m_5$. Such an
observing sequence of visits is illustrated for the ``baseline2018a''
cadence in figure \ref{fig:observation patter LSST 10 year survey},
where for one field in the WFD all observations within the 10 year
survey are shown. 
%
\begin{figure}
\centering
\includegraphics[scale=0.7]{figures/sl_field_number3_baseline2018a_Daniel.pdf}
\caption{This illustrates for the observing strategy ``baseline2018a'' the mjd and filter when observations are taken over the 10 year survey for field number 4 in table \ref{tab: 10 wfd fields} in the wide fast deep survey}
\label{fig:observation patter LSST 10 year survey}
\end{figure}
%
%\FloatBarrier 
To simulate observational data of a LSN system, we place randomly in
one of the 10 choosen fields from Table \ref{tab: 10 wfd fields} a
mock LSNe Ia from the OM 10 catalog, which is a mock catalog for
strong gravitational lenses \citep{Oguri:2010}. The catalog contains
about 400 LSNe Ia (The catalog is 10 times oversampled which means
that the total number of LSNe Ia is about 40) for LSST with an image
separation larger than $\SI{0.5}{\arcsec}$. For the lens model an SIE
(singular isothermal ellipsoid) \citep{Kormann:1994} is assumed and
therefore the time delay $\tau$, the convergence $\kappa$ and the
shear $\gamma$ is known for each of the multiple SNe images. By
assuming a W7 model and placing each of the SNe images randomly in the
corresponding magnification map one can calculate mock light
curves. Furthermore we place the mock system randomly in time, such
that the detection criterion applied in OM 10 is fulfilled. The
criterion is that the peak of the i-band magnitude of the fainter image
for a double or the 3rd brightest image for a quad, falls in the
observing season.
By combining this with the observing sequence, we get simulated
observations as illustrated in figure \ref{fig: simulated observation}
for a quad system. The error is calculated according to \cite[sec 3.5,
p. 67]{2009:LSSTscience}.
\begin{figure}[h!]
\centering
\includegraphics[scale=0.7]{figures/sl_Obsevation_number399_baseline2018a_filter_i_oversampling_00.pdf}
\caption[]{In this figure the i-band light curves of a mock quad LSNe Ia are shown. The observation sequence is for a random field in WFD survey for the cadence ``baseline2018a''.}
\label{fig: simulated observation}
\end{figure}
%
\begin{table}
\centering
\begin{tabular}{c|c|c|c|c|c|c|c|c|c|c}
field number & 1 & 2 & 3 & 4 & 5& 6 & 7 & 8 & 9 & 10  \\
\hline
RA& 0.0 & 32.1 & 65.8 & 50.9 &44.9& 125.6 & 155.0 & 207.7 & 304.3 & 327.5  \\
\hline
DEC& -7.4 & -44.2 & -7.2 & -30.0 & -50.9& -11.4 & -25.6 & -45.3 & -55.2 & -35.9  \\
\end{tabular}
\caption{The 10 fields of the wide fast deep survey, where the observation sequence for different cadences was considered.}
\label{tab: 10 wfd fields}
\end{table}
%
%\FloatBarrier

To evaluate the mock data and get a measured time delay we use the
free knot spline optimizer from PyCS (Python Curve Shifting)
\citep{2013:Tewesb,Bonvin:2015jia}. PyCS was initially developed to
measure time delays in strongly lensed quasars, and is not yet
optimised for LSNe Ia, such as fitting simultaneously multiple filters
and using SN template light curves.  Applying PyCS to individual
filter's light curves, we get a single independent time delay for each
filter.  We combine the 6 delays from the 6 LSST filters afterwards
into a single delay, but we expect more precise and accurate delays by
using multi-color fitting in the future. We also expect improvements
in delay measurements with the use of SNe Ia template instead of
splines.  

%so there are still
%some improvements for LSNe Ia, which will be implemented in the
%future. The first one is that to date multi-color time delay fitting
%is not possible, which means that we get a single independent time
%delay for each filter. We combine this 6 delays afterwards to a single
%delay, but we expect more precise and accurate delays by using
%multi-color fitting. A second improvement might be the use of SNe Ia
%templates instead of splines. This means that we do a conservative
%time delay estimate and with future improvements we might be able to
%measure time delays even better. 

To have sufficient statistics, we investigate for each cadence
strategy 202 mock LSNe Ia, where we pick 50 \% doubles and 50 \%
quads. For each of the mock systems we draw 100 random starting
configurations. A starting configuration corresponds to a random
position in the microlensing map and a random field from Table
\ref{tab: 10 wfd fields}, where it is placed randomly in one of the
observing seasons such that the detection requirement from OM 10 is
fulfilled. For each of these starting configurations we then draw 1000
different noise realizations of light curves, where we also shift the
time delays for each noise realizations randomly by $-3$ to
$\SI{3}{\day}$, to estimate uncertainties of delay measurements with
PyCS. For each realization we calculate the deviation from the true
time delay as
%
\begin{equation}
\tau_\mathrm{d} = \frac{t_\mathrm{measured} - t_\mathrm{true}}{t_\mathrm{true}}.
\label{eq: deviation from true time delay}
\end{equation}
For one strategy and double LSNe Ia, we have thus $1 {\rm (delay\ for\ the\
one\ pair\ of\ images)} \times 6 {\rm (filters)} \times 100 {\rm
(starting\ configurations)} \times 1000 {\rm (noise\ realisations)}$
time-delay deviations as in \eqref{eq: deviation from true time delay}.
%, where the 6 stands for the 6 LSST filters. 
For the 6 pairs of images for a quad system we have a sample of $6
\times 6 \times 100 \times 1000$. The resulting distribution of
time-delay deviation is investigated for each pair of images and each
filter separately. From the $100 \times 1000$ time-delay deviations we
define accuracy as the median $\tau_\mathrm{d,50}$ and precision as
$\delta = (\tau_\mathrm{d,84}-\tau_\mathrm{d,16})/2$, where
$\tau_\mathrm{d,84}$ is the 84th and $\tau_\mathrm{d,16}$ the 16th
percentile. Measuring $H_0$ with 1\% accuracy requires that the accuracy
in the delay deviation $\tau_\mathrm{d,50}$ is $<1\%$ (since $H_0 \propto
t_{\rm true}^{-1}$). Since the 6 time-delay deviations from the 6 filters are independent we combine them into a single time-delay deviation via the weighted mean. This means that in the end we have for one strategy and a mock LSNe Ia one 
\begin{equation}
\tau_\mathrm{d,50} \pm \delta
\label{eq: accuracy and precission}
\end{equation}
per pair of images.


\subsubsection{Results}
\label{sec:results}
In this section we summarize the results and quantify the 20
investigated cadences. Given that $H_0 \propto \frac{1}{t}$, where $t$
is the time delay between two images, we aim for accuracy
($\tau_\mathrm{d,50}$) smaller than 1 percent and precision ($\delta$)
smaller than 5 percent in equation \ref{eq: accuracy and precission}. The accuracy
requirement is needed for measuring $H_0$ with 1\% uncertainty, and
the precision requirement ensures that the delay uncertainty does not
dominate the overall uncertainty on $H_0$ given typical mass modeling
uncertainties of $\sim 5\%$ \citep[e.g.,][]{Suyu2018}.  A quad system is counted as successful if one of the 6 delays fulfills this requirement. 
We have investigated two different cases, first using LSST data only to measure time delays and second, using LSST as a discovering machine in combination with follow-up observations to measure delays. 
For just using LSST data we have investigated 202 mock LSNe Ia and for using LSST in combination with follow-up 100 mock systems have been investigated, where for each case 50 \% are doubles and 50 \% are quads. 
We assume follow-up observation would start 2 days 
after the third LSST data point in any filter exceeds the 5-$\sigma$ 
depth, where the follow-up is done in 3 filters (g,r,i) every second night.
The fraction of systems for which the time-delay measurement fulfills the above defined requirement is summarized in columns 3 and 5 of Table \ref{tab: quantify different observing strategies}. These numbers have to be combined with the total number of LSNe Ia we expect to detect for different strategies. We approximate the total number of LSNe Ia as

\begin{align}
\label{eq: total number of LSNe Ia from modified OM 10}
N_\mathrm{LSNe Ia, cad} = N_\mathrm{LSNe Ia, OM 10} \frac{\Omega_\mathrm{cad}}{\Omega_\mathrm{OM 10}} \frac{\bar{t}_\mathrm{eff,cad}}{t_\mathrm{eff, OM 10}}
\end{align}
%
where $N_\mathrm{LSNe Ia, OM10} = 45.7$, $\Omega_\mathrm{OM10} = \SI{20000}{\square\deg}$ and $t_\mathrm{eff, OM10}=\SI{2.5}{\year}$ from \cite{Oguri:2010}. $\bar{t}_\mathrm{eff,cad}$ is the mean effective/cumulative seasonal length for a given cadence strategy, where we have averaged over all 719 WFD fields. $\Omega_\mathrm{cad}$ is the survey area for a given observing strategy. Instead of taking the nominal values (see column 2 Table \ref{tab:LSST Survey Area for different observing strategies}) we calculate the area from fields represented by our study. These are fields with a mean cumulative season length and inter-night gap similar or even better than the 719 WFD fields, meaning: Cumulative season length ($t_\mathrm{eff}$) longer than the lower 99th percentile and inter-night gap ($t_\mathrm{gap}$) shorter than the upper 99th percentile. Further we also take into account the 5-$\sigma$ depth ($m_5$)\footnote{Here we consider only the main relevant bands g,r,i and z.}. Here we consider all fields with ($m_5+0.2 \mathrm{mag}$) greater than the lower 99th percentile of the 719 WFD fields. The relaxed 5-$\sigma$ depth is necessary in order to represent the wider areas as suggested by the nominal values. The area can then be calculated from the number of systems fulfilling the above defined criteria times the field of view of $\SI{9.6}{\square\deg}$ taking into account the overlap factor of the fields:
%
\begin{align}
\Omega_\mathrm{cad} = f_\mathrm{overlap} \cdot N_\mathrm{cad,criteria} \cdot \SI{9.6}{\square\deg},
\end{align}
where
$f_\mathrm{overlap}=\frac{5292 \cdot \SI{9.6}{\square\deg}}{4 \pi \cdot (\SI{180}{\deg}/\pi)^2} \approx 0.812.$
The calculated area is shown in column 3 of Table \ref{tab:LSST Survey Area for different observing strategies}. They still represent the wider areas as suggested by the nominal values, but are more representative in terms of delay measurements than the nominal areas because they contain observational constraints on cumulative season length, inter-night gap and 5-$\sigma$ depth\footnote{The 0.2 mag difference in the 5-$\sigma$ depth would reduce for the LSST + follow-up case the number of detected systems by a few percents, whereas, it does not matter for the case of using LSST only.}.

\begin{table}
\centering

\begin{tabular}{c|c|c|c}                                                                                               
& $\Omega_\mathrm{nom}$ & $\Omega_\mathrm{cad}$ &  $(\Omega_\mathrm{cad}-\Omega_\mathrm{nom})/\Omega_\mathrm{nom} [\%]$ \\
\hline
\altsched          &  18000 &  17703 &  -1.6 \\
\altschedrolling   &  18000 &  20463 &  13.7 \\
\baseline          &  18000 &  17306 &  -3.9 \\
\colossusfour      &  18000 &  18202 &   1.1 \\
\colossusfive      &  18000 &  18475 &   2.6 \\
\colossusseven     &  18000 &  17797 &  -1.1 \\
\krakentwosix      &  18000 &  17119 &  -4.9 \\
\krakenfive        &  18000 &  17680 &  -1.8 \\
\krakenthreesix    &  18000 &  17719 &  -1.6 \\
\krakentwo         &  18000 &  17828 &  -1.0 \\
\krakenfour        &  24700 &  24010 &  -2.8 \\
\mothrafive        &  18000 &  16417 &  -8.8 \\
\mothranine        &  24700 &  21874 & -11.4 \\
\nexusseven        &  24700 &  20471 & -17.1 \\
\pontuszerozerotwo &  24700 &  22926 &  -7.2 \\
\pontusnine        &  18000 &  17758 &  -1.3 \\
\pontusfivezerotwo &  18000 &  17602 &  -2.2 \\
\pontusfivezerosix &  18000 &  18132 &   0.7 \\
\rollingopsim      &  18000 &  18148 &   0.8 \\
\rollingmixopsim   &  18000 &  18132 &   0.7 \\
\end{tabular} 
 \caption{Survey areas for different observing strategies. The second column contains the nominal values and the third column shows the area used for Equation \ref{eq: total number of LSNe Ia from modified OM 10} taking into account observational constraints (see text). The forth column shows the deviation from the nominal values in percent.}
 \label{tab:LSST Survey Area for different observing strategies}
\end{table}




The results in terms of total number of LSNe Ia for the investigated cadences are shown in column 6 of Table \ref{tab: quantify different observing strategies}, where we see that for most of the rolling cadences (blue strategies) fewer LSNe Ia will be detected, because of the shorter cumulative season lengths $\bar{t}_\mathrm{eff,cad}$. The total number depends on the selection criteria assumed in \cite{Oguri:2010}. If we relax on the criteria like the image separation these numbers will be higher, but the order of the strategies will be unchanged.

Columns 2 and 4 in Table \ref{tab: quantify different observing strategies} contain the total number of LSNe Ia where the delay can be measured with accuracy $<$ 1 \% and precision $<$ 5 \% over the 10 year survey. For the case of using only LSST data, we see that even for the best strategies we will just have
a few systems where time-delay measurements are possible. Follow-up observations are therefore necessary to
increase the number of LSNe Ia with delays as shown in column 2 of Table \ref{tab: quantify different observing strategies}. These results are also more qualitatively summarized in Table \ref{tab: favoured strategies}

To summarize, for our science case of measuring time delays from as many lensed SNe as possible, it would be more effective to use LSST as a discovering machine with additional follow-up, instead of relying on LSST completely for the delay measurements. From our investigations we find that long cumulative seasonal lengths $\bar{t}_\mathrm{eff,cad}$ and a more frequent sampling are important. To go more into detail, we request for 10 seasons with a season length of 170 days or longer. Rolling cadences are clearly disfavored, because their shortened cumulative season lengths lead to overall a more negative impact on the number of LSNe Ia with delays, compared to the gain from the increased sampling frequency\footnote{rolling\_mix\_10\_yrs\_opsim has a cumulative season length close to baseline2018a and does the revisit in different filters. It is therefore the only rolling cadence performing better than the baseline cadence.}. To improve the sampling, single visits per night would be helpful. Since this will make the science case of fast-moving transients impossible we suggest doing one revisit within a night but in a different band than the first visit. Further improvements are the replacement of $2 \times \SI{15}{\s}$ exposure by $1 \times \SI{30}{\s}$ for an increased efficiency and redistributing some of the visits in y band to g, r, i and z.
%of their shorter cumulative season lengths $\bar{t}_\mathrm{eff,cad}$ although they improve the sampling. Therefore we reject to improve on one of the 3 parameters our science case is mostly sensitive to, by worsen at the same time one of the others significantly. 
%

Further \cite{Goldstein:2018bue} performed detailed simulations of the gLSN population using a completely independent technique and pipeline and reached similar conclusions to the ones presented here: rolling cadences are strongly disfavored, and wide-area, long-season surveys with well sampled light curves are optimal.


\begin{table}
\centering
\begin{tabular}{c|cc|cc|c}
\multicolumn{1}{c}{}& \multicolumn{2}{c}{\textbf{LSST + follow-up}}  & \multicolumn{2}{c}{\textbf{LSST only}} & \multicolumn{1}{c}{ } \\

& total  & total   & total  & total  & total \\
& number  & fraction  & number  & fraction  & number \\
&  with & with & with& with & of\\
& delays & delays& delays& delays&LSNe Ia\\
\hline
\krakenfour        &  27.7 &  27.2 &  5.9 &   5.8 &  102.0 \\
\colossusseven     &  27.5 &  32.7 &  7.1 &   8.4 &   84.0 \\
\altsched          &  21.8 &  35.3 &  7.9 &  12.8 &   61.7 \\
\pontusfivezerosix &  20.1 &  27.8 &  6.1 &   8.4 &   72.2 \\
\krakentwo         &  19.7 &  25.2 &  4.5 &   5.8 &   78.0 \\
\pontusnine        &  19.3 &  23.8 &  6.0 &   7.4 &   81.1 \\
\rollingmixopsim   &  18.9 &  23.8 &  7.6 &   9.5 &   79.4 \\
\krakentwosix      &  18.7 &  25.8 &  3.5 &   4.8 &   72.4 \\
\pontuszerozerotwo &  18.1 &  21.0 &  1.2 &   1.4 &   86.1 \\
\colossusfive      &  17.2 &  22.4 &  2.9 &   3.7 &   76.8 \\
\baseline          &  16.5 &  22.4 &  2.7 &   3.7 &   73.4 \\
\colossusfour      &  15.7 &  21.0 &  3.1 &   4.1 &   74.6 \\
\krakenfive        &  15.5 &  21.0 &  2.0 &   2.7 &   73.4 \\
\altschedrolling   &  13.7 &  35.9 &  6.3 &  16.5 &   38.0 \\
\pontusfivezerotwo &  13.5 &  17.7 &  1.0 &   1.4 &   76.3 \\
\nexusseven        &  12.5 &  23.8 &  2.5 &   4.8 &   52.3 \\
\mothranine        &  12.2 &  23.8 &  2.4 &   4.7 &   51.0 \\
\rollingopsim      &  11.9 &  14.9 &  5.4 &   6.8 &   79.8 \\
\krakenthreesix    &  11.4 &  25.2 &  2.1 &   4.7 &   45.2 \\
\mothrafive        &  10.4 &  27.9 &  2.3 &   6.1 &   37.3 \\


\end{tabular}
\caption{This table quantifies 20 different cadence strategies for measuring time delays in LSNe Ia. The second and third column consider LSST as a discovery machine in combination with follow-up observations in 3 filters (g,r,i) every second night. We assume follow-up observation would start 2 days after the third LSST data point in any filter exceeding the 5-$\sigma$ depth. In the fourth and fifth column only LSST data is used to measure time delays. The sixth column shows the total amount of LSNe Ia (69 \% doubles and 31 \% quads) calculated via \eqref{eq: total number of LSNe Ia from modified OM 10}. The columns "total fraction with delays" contain the fraction of the  investigated mock LSNe Ia where the time delay could be measured with accuracy better than 1 percent and precision better than 5 percent. We have investigated 100 mock systems for LSST + follow-up and 202 mock systems for the case where only LSST data is used. The columns "total number with delays" combine the columns "total fraction with delays" with the column "total number of LSNe Ia". Since for LSST only, the total numbers with delays are very low, we advocate using LSST as a discovering machine with observational follow up. Therefore the second column is the relevant one to quantify different cadences. For a more qualitative conclusion see Table \ref{tab: favoured strategies}.}
\label{tab: quantify different observing strategies}
\end{table}
%
\begin{table}
\centering
\begin{tabular}{c|c|c|c}
& good with obs.   &  favored with . \\
& follow-up in addition to LSST  & LSST data only \\
\hline
\krakenfour        &  x &  x \\
\hline
\colossusseven     &  x &  x \\
\hline
\altsched          &  x &  x \\
\hline
\pontusfivezerosix &  x &  x \\
\hline
\krakentwo         &  x &   \\
\hline
\pontusnine        &  x &  x \\
\hline
\rollingmixopsim   &  x &  x \\
\hline
\krakentwosix      &  x &  \\
\hline
\pontuszerozerotwo &  x &  \\
\hline
\colossusfive      &  x &   \\
\hline
\baseline          &  x &  \\
\hline
\colossusfour      &  &  \\
\hline
\krakenfive        &  &   \\
\hline
\altschedrolling   &  &  x \\
\hline
\pontusfivezerotwo &  &   \\
\hline
\nexusseven        &   &   \\
\hline
\mothranine        &  &   \\
\hline
\rollingopsim      & &  x \\
\hline
\krakenthreesix    &  &   \\
\hline
\mothrafive        &  &  \\

\end{tabular}
\caption{This table ranks 20 different cadences for the scenarios combining LSST with follow-up and using LSST data only. Cadences marked with a "x" are good for the given scenario, where unmarked ones are disfavored.}
\label{tab: favoured strategies}
\end{table}
%
\FloatBarrier
\subsection{Supernovae Lensed by Galaxy Clusters}
\textit{Contributor: Tanja Petrushevska}

\

Here, we focus on prospects of observing supernovae which are
lensed by known galaxy clusters. High-z galaxies that appear as
multiple images in the cluster field can host supernova
explosions. Strongly lensed supernovae by galaxy clusters not only
can be used as tools to examine both global cosmology, but also
the local environment of the cluster lenses. Cluster lensing time
scales are typically much longer and the microlensing effects are
almost negligible, which makes their measurement potentially more
feasible, especially if the lens potential is well studied and the
predicted time delays have small uncertainties. We calculate the
expected number of supernovae Ia in the multiply lensed background
galaxies by using the Hubble Frontier Fields cluster and Abell
1689. These clusters have been extensively studied, and given the
good quality data, well constrained magnification maps and time
delays can be obtained from the lensing models. We only considered
those that have a spectroscopic redshift. To obtain better image
depth, we combine the images that are taken closer than 5 days in
time. We note that these are a lower limits, since we have only
considered few clusters and the galaxies with spectroscopic
redshift. For this science case, the most important bands are i, z
and y. Since most of the light of nearby SNe is in the optical
bands, these filters are optimal for finding high-z SNe, as their
light is redshifted to the longer wavelengths. When we consider
the different observing strategies, we find that the rolling
cadences (mothra\_2045 and pontus\_2502) are disfavored given that
they do not return to the clusters as the other observing
strategies. The strategy pontus\_2489 provides slightly better
prospects compared to the others because it provides the most
number of visits to the same cluster fields.

\begin{figure}
\centering
\includegraphics[scale=0.65]{figures/sl_galaxy_lensing.pdf}\caption{The expected total number of strongly lensed SNe Ia arising from the multiply imaged galaxies in the Hubble Frontier Fields and Abell 1689 in function of the observing strategy. \todo{Tanja}{include new cadences and do same estimates for CC SNe*}}
\end{figure}


\subsection{Lensed Quasars\footnote{Summarized and updated version of  \citep[Cosmology chapter of][]{LSSTScienceCollaboration2017}}}
\textit{Contributors: Phil Marshall, Timo Anguita, Lynne Jones}


The goal of this section is to evaluate the precision we can achieve in measuring time delays in strongly lensed AGN, and as such, the precision on the measurement of the Hubble constant from all systems with measured time delays.

Anticipating that the time delay accuracy would depend on night-to-night
cadence, season length, and campaign length, we carried out a large
scale simulation and measurement program that coarsely sampled these
schedule properties. In \cite{Liao2015}, we simulated 5 different
light curve datasets, each containing 1000 lenses, and presented them to
the strong lensing community in a ``Time Delay Challenge.'' These 5
challenge ``rungs'' differed by their schedule properties. Focusing on the best challenge
submissions made by the community, we derived a simple power law model
for the variation of each of the time delay accuracy, time delay
precision, and useable sample fraction, with the schedule properties
cadence, season length and campaign length. They are
given by the following equations:

\begin{align}
|A|_{\rm model} &\approx 0.06\% \left(\frac{\rm cad} {\rm 3 days}  \right)^{0.0}
\left(\frac{\rm sea}  {\rm 4 months}\right)^{-1.0}
\left(\frac{\rm camp}{\rm 5 years} \right)^{-1.1}  \notag \\
P_{\rm model} &\approx 4.0\% \left(\frac{\rm cad} {\rm 3 days}  \right)^{ 0.7}
\left(\frac{\rm sea}  {\rm 4 months}\right)^{-0.3}
\left(\frac{\rm camp}{\rm 5 years} \right)^{-0.6}  \notag \\
f_{\rm model} &\approx 30\% \left(\frac{\rm cad} {\rm 3 days}  \right)^{-0.4}
\left(\frac{\rm sea}  {\rm 4 months}\right)^{ 0.8}
\left(\frac{\rm camp}{\rm 5 years} \right)^{-0.2} \notag 
\end{align}

All three of these diagnostic metrics would, in an ideal world, be
optimized: this could be achieved by decreasing the night-to-night
cadence (to better sample the light curves), extending the observing
season length (to maximize the chances of capturing a strong variation
and its echo), and extending the campaign length (to increase the number
of effective time delay measurements).

The quantity of greatest scientific interest is the {\it accuracy in
	cosmological parameters}: this could be computed as follows. Setting a
required accuracy threshold  defines the available number of lenses,
which in turns gives us the mean precision per lens there. Combining the
whole sample, we would get the error on the weighted mean time delay, as
used by \cite{Coe2009}. This uncertainty, which scales as one
over the square root of the number of available lenses,  can be roughly
equated to the statistical uncertainty on the Hubble constant.

Our Opsim analysis consists in selecting only the sky survey area that allows time delay measurements with accuracies of $<0.2\%$ \citep{Hojjati2014}. This high accuracy area can be used
to define a ``Gold Sample'' of lenses, whose mean precision per lens we
can compute. The TDC2 useable fraction averaged over this area gives us
the approximate size of this sample: we simply re-scale the 400 lenses
predicted by \cite{Liao2015} by this fraction over the 30\% found
in TDC2. While these numbers are approximate, the ratios between
different observing and analysis strategies provide a useful indication of relative merit.

As described above, we follow \cite{Coe2009} and compute a
very simple time delay distance Figure of Merit ``\texttt{DPrecision}''
as follows. We first combine the fractional time delay precision in
quadrature with an assumed 4\% ``modeling uncertainty,'' and then divide
this by the square root of the number of Gold Sample lenses. This
estimated ensemble distance precision can be straightforwardly related
to cosmological parameter precision, as cite{Coe2009} show
(it's very roughly the precision on the Hubble constant).

Our calculations are performed using the full 10 years of LSST operations with both the single i-band observations and all bands merged together. Naturally, significantly better results are obtained making no distinctions between photometric bands, however, it is important to note that there is a rather large caveat: Even when AGN variability can show almost negligible difference between bands close in wavelength, the difference can be important between the bluest and reddest LSST bands. As such, the i-band results can be interpreted as a very conservative upper limit in the precision attainable and the "all-bands" result as a very optimistic lower limit.

\begin{figure}
\centering
\includegraphics[width=\linewidth]{figures/sl_QSO_Nlens.png}    
		\caption{Number of ``golden lenses'' in sky areas with accuracies $<$0.2\%}   
\end{figure}

\begin{figure}
\centering
\includegraphics[width=\linewidth]{sl_QSO_Precision.png}    
		\caption{Mean precision per lens (percent) in  sky areas with accuracies $<$0.2\%}  
\end{figure}
\begin{figure}
\centering
		\includegraphics[width=\linewidth]{sl_QSO_Dprec.png}    
		\caption{Time delay distance precision figure of merit (percent) in  sky areas with accuracies $<$0.2\%} 
\end{figure}



% \pagebreak
% \documentclass [11pt,a4paper]{article}
\usepackage{graphicx,subfigure}
%\documentstyle [epsf,epsfig,11pt,amssymb]{article}
\usepackage{rotating}
\usepackage{lsstdesc_macros}
\usepackage[title]{appendix}
%\usepackage{arydshln}

\RequirePackage[colorlinks,breaklinks]{hyperref}  
\hypersetup{linkcolor=blue,citecolor=blue,filecolor=black,urlcolor=blue} 
\usepackage{longtable}
\usepackage{natbib}
\usepackage{verbatim}
%\usepackage{subcaption}

% Set up generic page
%
%\setlength{\textheight}{23cm}
%\setlength{\textwidth}{17cm}
%\unitlength 1mm
%\font\ninerm=cmr9

\textwidth=16.cm
%\textheight=26cm
\setlength{\hoffset}{-1.cm}
\setlength{\voffset}{-1.cm}
\parskip=0.1cm

%\newcommand{\ttt}{{\mbox{\newfont t}}}

\newcommand{\cosmos}{COSMOS}
\newcommand{\xmmlss}{XMM-LSS}
\newcommand{\cdfs}{CDFS}
\newcommand{\elais}{ELAIS-S1}
\newcommand{\spt}{SPT DEEP}
\newcommand{\ddfa}{DDF\_820}
\newcommand{\ddfb}{DDF\_858}
\newcommand{\ddfc}{DDF\_1200}
\newcommand{\ddfd}{DDF\_2689}
\newcommand{\feature}{feature\_baseline\_10yrs}
\newcommand{\strech}{$X_1$}
\newcommand{\sncolor}{$c$}
\newcommand{\daymax}{$T_0$}
\newcommand{\redshift}{$z$}
\newcommand{\tmin}{$T_{min}$}
\newcommand{\tmax}{$T_{max}$}
\newcommand{\phasemin}{$ph_{min}$}
\newcommand{\phasemax}{$ph_{max}$}
\newcommand{\tgapcosmos}{$T_{gap}^{\cosmos}$}
\newcommand{\tgapothers}{$T_{gap}^{others}$}
\newcommand{\zfaint}{$z_{\mathrm{faint}}\ $}
\newcommand{\nsnfaint}{$N_{z<z_{\mathrm{faint}}}\ $}
\newcommand{\zmed}{$z_{\mathrm{med}}\ $}
\newcommand{\nsnmed}{$N_{z<z_{\mathrm{med}}}\ $}
\newcommand{\FixMe}[1]{{\color{red} \bf \large #1}}
\newcommand{\opsim}{{\tt OpSim\ }}
\newcommand{\slair}{{\tt SLAIR\ }}
\newcommand{\altschedsched}{{\tt AltSched\ }}
\newcommand{\altsched}{{\tt altsched\ }}
\newcommand{\myparagraph}[1]{\paragraph{#1}\mbox{}\\}
\newcommand{\sne}{{SNe Ia }}

\graphicspath{{Figures/}}

\begin{document}

\renewcommand\appendix{\par
  \setcounter{section}{0}
  \setcounter{subsection}{0}
  \setcounter{figure}{0}
  \setcounter{table}{0}
  \renewcommand\thesection{Appendix} %\Alph{section}}
  \renewcommand\thefigure{\Alph{section}\arabic{figure}}
  \renewcommand\thetable{\Alph{section}\arabic{table}} 
}

\begin{titlepage}
   \vspace*{\stretch{1.0}}
   \begin{center}
      \Large\textbf{Final assessment for the white paper call}\\
        \vspace*{0.5cm}
      \Large\textbf{DESC-SN group} \\
      
		  \vspace*{0.5cm}

      \large Contributors: \large\textit{T. Allam Jr.,R.Biswas, J.Carrick, Ph.Gris, R.Hlo\v{z}ek, I.Hook, A.Kim, M.Lochner, J.McEwen, H. Peiris, N.Regnault, R. Schuhmann, D.Scolnic, C.Setzer} \\
	\vspace*{0.25cm}

    \large with fruitful discussions with: \large\textit{O.Boberg, T.Ribeiro, D. Rotchild, P.Yoachim} \\
	\vspace*{0.5cm}

      \large\textit{4 October 2018}
   \end{center}
   \vspace*{\stretch{2.0}}
\end{titlepage}


\tableofcontents


\section{Introduction}



SN cosmology is systematics-limited. Increasing the statistics in the
Hubble diagram must be coupled with (1) advances in the measurement of
the SN distances (i.e.  a control at the per-mil level of the
photometry and survey flux calibration, and possibly a 3-parameter SN
standardization technique) (2) a better control of the SN
astrophysical environment and its potential impacts on the SN
light curves and distances (local host properties, absorption) (3) a
better control of the SN diversity (SN~Ia sub-populations, population
drift with redshift) (4) a precise determination of the survey
selection function (SN identification, residual contamination by
non-SN~Ia's as a function of redshift).

Access to spectroscopy will not scale with the large amount of SNe
LSST will deliver.  About 10\% of LSST SNe will benefit from a live
spectrum.  Securing spectroscopic host redshifts for the full LSST
sample using the fiber spectrographs available in the southern
hemisphere is challenging -- although doable.  As a consequence, all
the studies listed above, in particular SN~Ia identification and the
standardization of SN luminosity distances will rely on the supernova
light curves only.  Obtaining high quality SN light curves is
therefore a key design point of the SN survey.  The average quality of
the SN light curves depends exclusively on the observing strategy.

Furthermore, spectroscopic time being scarse, we cannot afford to
waste it.  This means that all transients identified
spectroscopically, around peak luminosity, as SNe~Ia, {\em must}
eventually have light curves of sufficient quality to end up in the
Hubble diagram.  This puts another requirement on the regularity and
predictability of the observing strategy.

Since light curve quality is at the core of the design of the LSST SN
survey, we propose to adopt as our main metric, the size and redshift
extent of the subset of well sampled SNe~Ia.  We define in the next
section what we mean by ``well-sampled''.


Four key facets of observing strategy that have an impact on the number and on the quality of well-measured supernovae may be identified: {\it a regular cadence} (typical values: three to four days) is important to get well-sampled light curves ; minimal inter-night gaps are mandatory to keep a high detection efficiency of the supernovae; the {\it season length} has an impact on the total number of supernovae that may be collected ; 170 to 180 days are values of interest for supernova science; {\it depth} quantified by m5, the five-sigma depth, which is the magnitude corresponding to a flux with a signal-to-noise ratio (SNR) equal to 5. Since only light curve points of well-measured supernovae with SNR $>$5 are considered, m5 has an impact on the redshift limit of observation (photostatistic limit); {\it spatial coverage and uniformity} which has an impact both on the wide and on the deep surveys; it may be interesting to observe deep fields evenly distributed in Ra (and Galactic/Ecliptic planes avoided) so as to search for anisotropies using individual Hubble diagrams.


\subsection{Requirements on SN sampling}
\label{sec:sn_sampling_requirements}

Light curves are the essential ingredient to (1) measure standardized
luminosity distances and (2) photometrically identify SNe~Ia from
their full light curve. This drives a series of requirements which we
summarize below.

\begin{enumerate}

\item each SN must have good quality measurements in at least three
  bands. We need two bands, covering the restframe $B$ and $V$ region,
  to constrain the restframe color of the SN. We need to provision an
  additional band (redder than restframe $V$), to enable next
  generation standardization techniques, that will likely rely on two
  restframe colors.

\item the follow-up of each supernova must be good enough in the
  observer-frame bands that correspond to the $B$- and $V$-restframe
  spectrum ($3800 \angstrom < \lambda < 7000 \angstrom$).  At
  high-redshift, in particular, one should avoid relying on the $UV$
  restframe region to derive a distance, given the high intrinsic
  dispersion of SN~Ia at those wavelengths.
  
\item we require the light curve shape to be well sampled in the
  (restframe) phase interval $[-10;+30]$ days, with at least five
  visits before peak (each of those visits in any of the eligible
  band), and ten visits after peak.  To obtain this in the lower
  redshift region of the Hubble diagram, one requires an
  observer-frame cadence of 4 days.  At higher redshifts redshifts
  (DDF fields), this requirement may be slightly relaxed. However,
  since we are going to rely almost exclusively on photometric
  identification, it is essential to secure a tight sampling of the SN
  color evolution at all redshift.
  
\item we require that the photon noise contribution to the distance
  measurement is subdominant w.r.t. the intrinsic dispersion of the
  SNe (after standardization).  There are several ways to quantify
  this.  With today's standardization techniques, the SN standardized
  distance modulus is:
  \begin{equation}
    \mu = m^\star_B + \alpha X_1 - \beta C - \cal{M}
  \end{equation}
  where $m^\star_B$ is the peak brightness in restframe $B$, $X_1$
  characterize the lightcurve width, and $C$ is an estimate of the
  restframe color $B-V$. $\alpha$, $\beta$ and $\cal{M}$ are global
  parameters, fit along with the cosmology. If the light curve is
  correctly sampled (see point above), the propagation of the
  measurement uncertainties affecting $m^\star_B$, $X_1$ and $C$ is
  dominated by the contribution of $\sigma_C$. (since $\beta \sim
  3$). In practice, requiring $\sigma C < 0.04$ ensures that $\sigma
  \mu < 0.1$, below the intrinsic dipersion in the Hubble diagram,
  after standardization.
\end{enumerate}



\section{Overview of observing strategies}


\subsection{Available strategies}

Four classes of observing strategies have been studied in detail:
\begin{itemize}

\item the {\tt OpSim}-based cadences released along with the white
  paper call (11 cadences released in June 2018 plus 4 additional
  simulations released in August), 
  
\item the {\tt OpSim}- and {\tt Feature}-based strategies that were
  available before the white paper call,

\item simulations based on the \altschedsched scheduler proposed by
  Stubbs and Rothchild.  This includes one rolling and another
  non-rolling cadence, plus a non-rolling simulation conducted on a
  larger footprint,

\item finally, we have tested a series of experimental observing
  strategies, based on the new feature-based scheduler (a.k.a.  SLAIR,
  Yoachim et al). These variations were produced by P. Yoachim, the
  main author of SLAIR, based on discussions we had at the summer 2018
  DESC week (CMU) and the 2018 LSST community workshop (Tucson).
\end{itemize}

In table \ref{tab:global_cadence_stats} we report, for each cadence
the total number of exposures per filter. This gives a general idea of
the total open shutter time, and the global filter allocation.  In
table \ref{tab:sn_specific_cadence_stats}, we show SN-specific key
statistics, in particular, the median number of visits in each filter,
for a given location of the sky (e.g. a healpixel), the median cadence
delivered by the observing strategy, the median season duration and
the total footprint of the survey.

The DDF  observations involve a  small number of fields  and $O(10^4)$
SNe. The list of simulated DDF is given in table
\ref{tab:ddf_list} and on figure \ref{fig:ddf_map}. The location of four DDF (referred to as reference fields in the following: \cosmos, \xmmlss, \cdfs~and \elais) has already been chosen by the project. The number of considered DDF ranges from 4 to 9.


\begin{table*}[!htbp]
  \caption{List and location of Deep Drilling Fields observed. "All" stands for all simulations but the ones performed with \altschedsched.}\label{tab:ddf_list}
  \begin{center}
  \begin{tabular}{lcccc}
    \hline
    \hline
    Field name & \opsim~ID & Ra (deg) & Dec(deg) & Observing strategy\\
    \hline
    \hline
    \cosmos & 2786 & 150.36 & 2.84 &All \\
    \xmmlss & 2412 & 34.39 & -5.09 & All \\
    \cdfs & 1427 & 53.00 & -27.44 & All \\
    \elais & 744 & 0.  & -45.52 & All \\
    \spt & 290 & 349.39 & -63.32 & All except feature*\\
    \ddfa & 820 & 119.55 & -43.37 & kraken\_2035\\
    \ddfb & 858 & 187.62 & -42.49 & kraken\_2035\\
    \ddfc & 1200 & 176.63 & -33.15 & kraken\_2035\\
    \ddfd & 2689 & 201.85 & 0.93 & kraken\_2035\\
    \hline
  \end{tabular}
 
  \end{center}
\end{table*}


\begin{figure}[htbp]
\begin{center}
\includegraphics[width=14cm,height=10cm]{overview_strategy/All.png}
\caption{Location of the Deep Drilling fields observed (black squares). Deep fields observed by previous surveys (red circles) and potential candidates for spectroscopic follow-up (green stars) are also mentioned. Yellow and magenta lines represent the Galactic and Ecliptic planes, respectively. Blue and red stars indicate potential deep field locations for EUCLID and WFIRST, respecivelly.}\label{fig:ddf_map}
\end{center}
\end{figure}

DDF observations are composed of sequences of 96 visits\footnote{u band visits are also available but were not considered in the following}  (in a row) in r,g,i,z,y bands (namely 20, 10, 20, 26, 20 visits). This corresponds to a total observing time of about one hour and few minutes if filter changes, slew times and telescope overheads are taken into account.

\subsection{Key properties}

\subsubsection {WFD}
\label{sec:wfd_cadence_key_properties}

\paragraph{Effective SN cadence} What ultimately determines the quality of the SN light curves
and distances is the {\em effective} cadence delivered by the survey,
i.e.  the cadence evaluated after having stacked all the same-band
revisit pairs (or sometimes triplets) performed during one single night. The cadence may be estimated, on each
healpixel, by computing the mean time interval between visits -- after
having excluded the long duration gaps when the healpixel is not
observable.

The effective cadence varies considerably from one observing strategy
to another.  As an illustration, we present on figures
\ref{fig:pontus_2502_effective_cadences},
\ref{fig:altsched_effective_cadences} and
\ref{fig:altsched_rolling_effective_cadences}, we present effective
cadence maps for {\tt Pontus\_2502}, \altsched and \altsched
rolling strategies.  On figure \ref{fig:effective_cadence}, we
report the median cadence in $r$ (computed from similar maps).  The
best effective cadences are delivered by the \altsched like
strategies. We note that there is about a factor 7 between the best
and worst effective cadences.

\begin{figure}
  \begin{center}
    \includegraphics[width=0.8\linewidth]{overview_strategy/pontus_2502_cadence.png}
    \caption{Average $\Delta T$ between observations for {\tt
        Pontus\_2502}, a (slightly rolling) strategy that implements
      same-band visit pairs. }
    \label{fig:pontus_2502_effective_cadences}
  \end{center}
\end{figure}


\begin{figure}
  \begin{center}
    \includegraphics[width=0.8\linewidth]{overview_strategy/altsched_cadence.png}
    \caption{Average $\Delta T$ between observations for \altsched, a strategy that avoid same band visit pairs.}
    \label{fig:altsched_effective_cadences}
  \end{center}
\end{figure}


\begin{figure}
  \begin{center}
    \includegraphics[width=0.8\linewidth]{overview_strategy/altsched_rolling_cadence.png}
    \caption{Average $\Delta T$ between observations for \altsched rolling, a rolling strategy that avoids same band
      visit pairs.}
    \label{fig:altsched_rolling_effective_cadences}
  \end{center}
\end{figure}

\begin{figure}
  \begin{center}
    \includegraphics[width=0.8\linewidth]{overview_strategy/cadence.png}
    \caption{Assessment of the effective SN cadence for all
      strategies: median $\Delta T$ between observations (after
      grouping same night observations). \altsched cadences are shown
      in red, \opsim cadences released for the white paper call in
      blue, \slair in orange, and early cadences in deep blue. The
      rolling (resp. non-rolling) cadences are displayed with open
      (resp. solid) markers. }
    \label{fig:effective_cadence}
  \end{center}
\end{figure}

In the remaining of this section, we discuss some of the key design
elements that can explain such large differences.


\paragraph{Global filter allocation} All \opsim and \slair cadences   share the same global filter balance: 
$\sim 7\%, 10\%, 22\%, 22\%, 21\%$ and $18\%$ of the total open shutter time
are allocated to the $u, g, r, i, z$- and $y$ bands respectively.
\altschedsched makes a different choice, based on the fact that the
low-throughput of the LSST camera in $y$, combined with the high sky
brightness in that band limit the impact of $y$-band observations.
About half of the $y$-band observing time is therefore reallocated to
$g$, $r$ and $i$ and $z$ bands, the resulting filter balance is
therefore: 9\%, 11\%, 28\%, 18\% 26\% and 9\% ($ugrizy$ respectively). Of course, having more
observing time in $griz$ can only improve the quality of SN~Ia light curves and distances.


\paragraph{Filter allocation strategy} Another key point, is how the filters are
changed during the night.  In this regard, very different strategies
have been implemented by the various schedulers proposed so far.  This
is illustrated on figure \ref{fig:hourglass_plot_filter_alloc}, which
shows the filter usage for  {\em Pontus 2002} and {\em
  \altsched rolling}.  {\em Pontus 2002} attempts to minimize the
filter changes during a night and in fact keeps observing in the same
filter over several nights.  Conversely, the \altsched family of
cadences attempt to switch for a different filter after each observing
block, so that each revisit of the same field is performed in a
different band.  As we will see below, this has a major impact on the effective cadence
delivered by the survey. As of today, most \opsim cadences released so
far try to minimize the number of filter changes, all \altsched cadences
switch filters after each observing block, and \slair has been
experimenting with both strategies.

\begin{figure}
  \begin{center}
    \subfigure[Pontus 2002]{\includegraphics[width=0.48\linewidth]{overview_strategy/pontus_2002_hourglass.png}}
    \subfigure[\altsched rolling]{\includegraphics[width=0.48\linewidth]{overview_strategy/alt_sched_rolling_hourglass.png}}
    \caption{Hourglass plots showing the filter usage for two
      different observing strategies. Most \opsim and \slair observing
      strategies released so far tend to minimize the number of filter
      changes. The \altschedsched strategy makes sure that each field is
      observed twice a night in different filter. This results in a
      higher number of filter changes, but is highly beneficial for
      the SN follow-up. }
    \label{fig:hourglass_plot_filter_alloc}
  \end{center}
\end{figure}

\paragraph{The combined effet of visit pairs and filter allocation strategy} In all cadences proposed so far
(except a few simulations, notably {\tt colossus\_2667} and {\tt
  kraken\_2044}) all fields are observed twice a night, 1 to 2 hours
apart, in order to improve the detectability of short term transients
and solar system objects.  Depending on the filter allocation
strategy, this has a strong impact on the sampling quality of the
SN~Ia light curves.  Indeed, SN~Ia luminosities vary significantly
over time scales of 1-2 days.  Same band observations taken during a
given night may therefore be considered as one single visit. When
dealing with a fixed total number of visits per sky direction,
re-observing a field in the same band during a night degrades the SN
light curve sampling by a factor $\sim 2$.

\begin{figure}
  \begin{center}
    \includegraphics[width=0.8\linewidth]{overview_strategy/night_visit_ratio.png}
    \caption{Number of nights vs. number of visits for the median
      healpixel. We clearly see the effect of the same-band revisit
      strategy.  \altsched cadences are show in red, \opsim WPC in
      blue, \slair in orange, and early cadences in deep blue. The
      rolling (resp. non-rolling) cadences are displayed with open
      (resp. solid) markers. }
    \label{fig:effective_number_of_visits}
  \end{center}
\end{figure}

To estimate that, we compute, for each cadence, the healpix maps
giving the total number of {\em visits} per pixel and the total number
of {\em nights} (several visits per night).  On figure
\ref{fig:effective_number_of_visits}, we show, for each cadence, the
ratio of the median of these maps.  

The survey strategies are sorted according to the effective $r$-band
cadence delivered by the survey (see figure
\ref{fig:effective_cadence}).  We note that most good-cadence
strategies avoid same band revisit pairs. Some strategies however, are
able to compensate.  These are either rolling strategies ({\tt
  Mothra\_2049}, {\tt Mothra\_2045} or {\tt Kraken\_2036}) or by
increasing the total number of visits ({\tt Pontus\_2489}).

All \opsim cadences, except {\tt colossus\_2667} and {\tt
  kraken\_2044} the number of visits is about twice the number of
distinct nights, which is likely to degrade the effective SN cadence
(deeper visits, less often).  All \altsched simulations attempt to
avoid this, as do the recent experiements conducted with \slair.

\paragraph{Season length} This is another feature that may alter the cadence, and the size of the SN sample.
Nearby supernova light curves are less
affected by redshift time dilation, so season duration is not as
crucial as for DDF observations.  However, it is a quantity worth
estimating, in particular to see whether it correlates with the
effective cadence and/or the total size of the SN sample.
Figure \ref{fig:season_length} shows the average season length for
each cadence studied in this work (sorted by $r$-band effective
cadence).  We see considerable differences in season durations (from 130 to 200 days).
Most \opsim and all \altsched strategies manage to observe each field during about
150 consecutive days. The shorter (130 days) and longer (170 days) seasons come mainly from  \slair 
for reasons that have not been elucidated yet. In any case, we do not
note any significant correlation between season duration and cadence.

\begin{figure}
  \begin{center}
    \includegraphics[width=0.8\linewidth]{overview_strategy/season_length.png}
    \caption{Median season duration for each cadence in this
      study. \altsched cadences are show in red, \opsim WPC in blue,
      \slair in orange, and early cadences in deep blue. The rolling
      (resp. non-rolling) cadences are displayed with open
      (resp. solid) markers. }
    \label{fig:season_length}
  \end{center}
\end{figure}

\paragraph{Inter-night gaps}This is another important metric to estimate the ``quality'' of observing strategies. Large gaps between consecutive observations of the same field degrade in a significant way the quality of the supernovae light curves. For each 'healpixel' the fraction of gaps larger than 15 days has been estimated (excluding the very large gaps when the direction is not observable). The conclusion (Figure \ref{fig:large_gaps}) is that \altsched-like cadences tend to have smaller fractions of 15-days gaps (including ''rolling cadences'' which seems quite surprising).

\begin{figure}
  \begin{center}
    \includegraphics[width=0.8\linewidth]{overview_strategy/cadence_large_gaps.png}
    \caption{Fraction of gaps higher than 15 days (r band).\altsched cadences are show in red, \opsim WPC in blue,
      \slair in orange, and early cadences in deep blue. The rolling
      (resp. non-rolling) cadences are displayed with open
      (resp. solid) markers. }
    \label{fig:large_gaps}
  \end{center}
\end{figure}



\paragraph{} To conclude: we note a very large gap (about a factor 7) between
the best and the worst cadences delivered by the various strategies
released so far.  We will see in the next section, that effective
cadence is the key to maximize the size and depth of the SN sample.
We have discussed a few design elements that help improving the effective
cadence: (1) implementing a rolling strategy (2) change filters often
during the night (3) avoid same band visit pairs.

However, each of these elements taken alone is not
sufficient. Combining all of them has been tried, for example in the
\slair cadences {\tt tms\ roll} and {\tt rolling mix*} and has brought
tremendous improvements with respect to the previous experiments
conducted with this scheduler.  However, the best \slair simulations
are still behind \altschedsched regarding cadence.  On top of the elements
discussed above, it seems that the core of the \altschedsched observing strategy ensures
an extreme regularity of the cadence. This may be explained by the method used by \atlsched~ to observe the sky (see Appendix \ref{sec:opsim_altsched} for more details about differences between \opsim~and \altsched).

\begin{comment}
nrl, 2018-09-27: we still need to estimate (1) the same-filter
 1-day gaps (2) the cadence rms. I think that would help
understanding the difference between the best SLAIR cadences and
Altsched.
\end{comment}



\subsubsection{DDF}


All proposed observing strategies but altsched-like have included DDF.
Plots illustrating some of the key points mentioned above are given on Figures \ref{fig:cosmoscad}-\ref{fig:spt deep_m5} for baseline18a, feature\_baseline\_10yrs, kralen\_2026 and kraken\_2035 observing strategies and for the five to nine above-mentioned DDF. Among these cadences feature\_baseline\_10yrs displays interesting features with respect to supernovae observations for the reference DDF:

\paragraph{Cadence} a median cadence of three days is observed whereas other observing strategies present cadences that may reach up to 14 days. Inter-night gaps are also smaller for \cosmos~and \xmmlss: 10 to 15 days and 5 to 7 days for the first and second maxima respectively whereas  for baseline18a, kralen\_2026 and kraken\_2035 the first (second) maximum is at the level of 18 to 40 (13 to 20) days.  

\paragraph{Season length} feature\_baseline\_10yrs shows the highest season lengths with values around 150 days for \cosmos~and \xmmlss, and 180 days for \cdfs~and \elais~whereas other strategies lead to values of about 130, 140, 120, and 150 days for \cosmos, \xmmlss, \cdfs, and \elais, respectively.

\paragraph{Depth} while median m5-values are compatible among the strategies (the decrease during season 2 for the four fields in  feature\_baseline\_10yrs is due to a known bug in the weather simulations) the coadded m5 depth per season shows clearly that feature\_baseline\_10yrs is a 0.7 (\cosmos), 0.4 (\xmmlss, \cdfs, \elais) magnitude deeper (r-band) survey compared to the others. This results is to be explained by better cadences and longer seasons.
 
Key properties of \spt, \ddfa, \ddfb, \ddfc~and \ddfb~fields are given on Figures \ref{fig:spt deep_cad} to \ref{fig:kraken_m5}.



\section{Metrics}

\subsection{Number of well-sampled type Ia supernovae}

\subsubsection{Requirements on SN sampling}
\label{sec:sn_sampling_requirements}

Light curves are the essential ingredient to (1) measure standardized
luminosity distances and (2) photometrically identify SNe~Ia from
their full light curve. This drives a series of requirements which are
summarized below.

\begin{enumerate}

\item each SN must have good quality measurements in at least three
  bands. We need two bands, covering the restframe $B$ and $V$ region,
  to constrain the restframe color of the SN. We need to provision an
  additional band (redder than restframe $V$), to enable next
  generation standardization techniques, that will likely rely on two
  restframe colors.

\item the follow-up of each supernova must be good enough in the
  observer-frame bands that correspond to the $B$- and $V$-restframe
  spectrum ($3800 \angstrom < \lambda < 7000 \angstrom$).  At
  high-redshift, in particular, one should avoid relying on the $UV$
  restframe region to derive a distance, given the high intrinsic
  dispersion of SN~Ia at those wavelengths.
  
\item we require the light curve shape to be well sampled in the
  (restframe) phase interval $[-10;+30]$ days, with at least five
  visits before peak (each of those visits in any of the eligible
  band), and ten visits after peak.  To obtain this in the lower
  redshift region of the Hubble diagram, one requires an
  observer-frame cadence of 4 days.  At higher redshifts redshifts
  (DDF fields), this requirement may be slightly relaxed. However,
  since we are going to rely almost exclusively on photometric
  identification, it is essential to secure a tight sampling of the SN
  color evolution at all redshift.
  
\item we require that the photon noise contribution to the distance
  measurement is subdominant w.r.t. the intrinsic dispersion of the
  SNe (after standardization).  There are several ways to quantify
  this.  With today's standardization techniques, the SN standardized
  distance modulus is:
  \begin{equation}
    \mu = m^\star_B + \alpha X_1 - \beta C - \cal{M}
  \end{equation}
  where $m^\star_B$ is the peak brightness in restframe $B$, $X_1$
  characterize the lightcurve width, and $C$ is an estimate of the
  restframe color $B-V$. $\alpha$, $\beta$ and $\cal{M}$ are global
  parameters, fit along with the cosmology. If the light curve is
  correctly sampled (see point above), the propagation of the
  measurement uncertainties affecting $m^\star_B$, $X_1$ and $C$ is
  dominated by the contribution of $\sigma_C$. (since $\beta \sim
  3$). In practice, requiring $\sigma C < 0.04$ ensures that $\sigma
  \mu < 0.1$, below the intrinsic dipersion in the Hubble diagram,
  after standardization.
\end{enumerate}


\subsubsection{SN samples}

The SN sample usable for cosmology can be defined from the light curve
requirements listed in the previous section. A key quantity is the
redshift limit, $z_{lim}$, beyond which one starts loosing events
because of poor sampling.  This ``redshift-limit'' is not a
detectability limit.  It is the redshift value beyond which we start
losing a fraction of the events, because their photometric follow-up
does not match the requirements listed in the previous section.

The definition of the redshift limit depends on the intrinsic
luminosity of the supernova considered.  SN~Ia luminosities can be
parametrized using two parameters, e.g. lightcurve width ($X_1$) and
SN restframe B-V color at peak ($C$).  As shown on figure
\ref{fig:jla_X1_C}, the SN distribution in this parameter space is
compact and more than 95\% of the statistics can be enclosed in a
tight ellipse. 

\begin{figure}
  \begin{center}
    \includegraphics[width=0.6\textwidth]{key_design/sn_parameter_space.pdf}
    \caption{JLA supernovae the $(X_1,Color)$ parameter space --
      (blue: nearby, green: SDSS, orange: SNLS).  The large dots
      indicate the position of the faint, median and bright fiducial
      SNe used for the cadence analyses.}
    \label{fig:jla_X1_C}
  \end{center}
\end{figure}

On the same figure, we have represented three fiducial SNe of
particular interest: the red dot represents the faintest SN in this
region of the parameter space $(X_1=-2, C=0.2)$, the green dot and
blue dots show the average $(X_1=0, C=0)$ and brightest SN $(X_1=2,
C=-0.2)$ respectively.

We can define the redshift limit \zfaint as the limit beyond which the
faintest fiducal SN no longer passes the light curve requirements. By
doing this, we ensure that all SNe that live in the fiducial $(X_1,C)$
parameter space and are below \zfaint do pass our light curve
requirements.  \zfaint defines a {\em redshift-limited sample} whose
selection function does not depend on the SN properties.

We can also define a similar limit, using the median supernova,
instead of the faint one.  This defines a SN sample whose upper
redshift bins are affected by a selection bias, which must be
determined using a simulation -- which itself depends on our knowledge
of the SN luminosity distribution at those redshifts.  The uncertainty
affecting the determination of the selection function generally limits
the usefulness of the redshift bins affected by a selection bias.
Since the selection function is generally symmetric around its 50\%
point, the size of the sample limited by \zmed gives a good
approximation of the total number of LSST SNe that will have precise
distances.

For each cadence we may thus estimate the following quantities in addition to the total number of well-sampled supernovae: 

\begin{itemize}
\item the sample redshift limit, \zfaint defined above, and the number
  of well sampled supernovae below the redshift limit \nsnfaint
\item the redshift \zmed at which the median supernova defined above
  no longer passes the signal-to-noise requirements, and the number of
  well-sampled supernovae below this redshift, \nsnmed.  
\end{itemize}
The former give an assessment of the size and depth of the redshift
limited sample, i.e. the sample of supernovae usable for cosmology,
and whose selection function is extremely easy to determine.  The
latter gives an assessment of the size and depth of the sample of SNe
that will have precise distances.

\subsubsection{Deep Drilling fields}

\myparagraph{Method}

The strategy to estimate the number of well-measured type Ia supernovae may be summarized in four steps: (1) light curves are simulated and fitted using observations of a given cadence; (2) selection criteria are applied to get high-quality supernovae; (3) the resulting observing efficiency curves are then convolved with a production rate \cite{perrett} so as to estimate the number of well-measured type Ia supernovae that may be collected by LSST given an observing strategy. 

Table \ref{tab:sim_ddf} summarizes the parameter used in SN simulations. The selection of a sample of well-measured type Ia supernovae is done in two steps. A sample of observable supernovae is selected by requiring light curves to have \phasemin $\leq$ -5 and \phasemax $\geq$ 20 (where \phasemin~ and \phasemax~ are the minimal and maximum phases of the LC points, respectively). For each supernovae in this reference sample, additional selection criteria are applied:
\begin{itemize}
\item $N_{bef} \geq 4$ and $N_{aft} \geq 10$ where $N_{bef}$ and $N_{aft}$ are the number of LC points (with SNR$\geq$ 5) before and after \daymax.
 \item $\sigma_c \leq 0.04$ where $\sigma_c$ is the error on the \sncolor~parameter estimated from the fit of the light curve.
\end{itemize}
The season length which depends on the redshift is estimated using the supernovae of the reference sample.

\begin{table}[!htbp]
  \begin{center}
    \caption{Range of the parameters used to simulate type Ia SN}\label{tab:sim_ddf}
\begin{tabular}{cc}
  \hline
  \hline
Parameter & Range \\
\hline
\hline
                    & (-2.0,0.2),(0.0,0.0),(2.0,-0.2) \\
 (\strech,\sncolor) & (-2.0,0.0),(-2.0,0.2),(0.0,-0.2) \\
                    & (0.0,0.2),(2.0,0.0),(2.0,0.2) \\
\hline
\redshift           & [0.01,1.3] (step: 0.025) \\
\hline
\daymax             & [\tmin,\tmax] (step: 1 day) \\
                    & (\tmin~and \tmax~ are the min and max MJD of a season) \\
                    \hline
\end{tabular}
\end{center}
\end{table}

\myparagraph{Results}

Typical detection efficiencies are given on Fig. \ref{fig:effi} for the \cosmos~field and feature\_baseline\_10 yrs cadence . One may observe that the lowest efficiencies (independently on the (\strech,\sncolor) values) correspond to the first two seasons of observations which are known to be very bad for this observing strategy and for this field (see for instance Figs. \ref{fig:cosmos_cad} and \ref{fig:cosmos_m5}).

\begin{figure}[htbp]
\begin{center}
  
  \includegraphics[width=12cm]{nsn/effi_feature_cosmos.png}
 \caption{Detection efficiency as a function of the redshift for the \cosmos~field and feature\_baseline\_10yrs observing strategy (10 seasons). Blue, red and black lines correspond to faint, medium and bright supernovae, respectively.}\label{fig:effi}
\end{center}
\end{figure}

Efficiency curves are convolved with a production rate \cite{perrett} to estimate the number of well-measured type Ia supernovae that may be collected by LSST. Summary plots are given for the reference fields (Fig \ref{fig:nsn_four}) and for all the DDF (Fig \ref{fig:nsn_all}). One may observe that, despite bad observing years, \feature~ shows the best results in terms of number of well-measured type Ia supernovae. Equivalent results are obtained with Colossus\_2667.

\begin{figure}[htbp]
\begin{center}
  
  \includegraphics[width=15cm,height=10cm]{nsn/NSN_season_4DDF.png}
  \includegraphics[width=15cm,height=10cm]{nsn/NSN_all_4DDF.png}
 \caption{Top: Number of well-measured type Ia supernovae as a function of the season. Bottom: Number of  well-measured type Ia supernovae as a function of observing strategy after ten years of operation. Four DDF (\cosmos,\xmmlss,\cdfs,\elais) have been considered. The red line corresponds to 15k supernovae.}\label{fig:nsn_four}
\end{center}
\end{figure}

\begin{figure}[htbp]
\begin{center}
  
  \includegraphics[width=15cm,height=10cm]{nsn/NSN_season_allDDF.png}
  \includegraphics[width=15cm,height=10cm]{nsn/NSN_all_allDDF.png}
 \caption{Top: Number of well-measured type Ia supernovae as a function of the season. Bottom: Number of  well-measured type Ia supernovae as a function of observing strategy after ten years of operation. All DDF have been considered.The red line corresponds to 27k supernovae.}\label{fig:nsn_all}
\end{center}
\end{figure}

\myparagraph{Conclusion}

When considering all DDF the winner (w.r.t. the total number of oberved and well-sampled supernovae) is of course kraken\_2035 (30K after ten years) since this observing strategy considered 9 DDFs whereas all others observed 4 to 5 fields. One may observe that extrapolating a four fields configuration results (like the ones obtained with \feature) to a 9 DDF observing strategy will probably lead to an overestimation of the resulting number of well-measured type Ia supernovae. It is indeed difficult to maintain the same quality (in terms of cadence, season length and thus depth) when moving from a 4 to a 9 DDF strategy.

\begin{sidewaysfigure}
\begin{center}  
  \includegraphics[width=\linewidth]{nsn/Z95_NSN.png}
 \caption{95\% redshift limit (ie corresponding to the detection of 95\% of the supernovae of the corresponding sample) as a function of the number of faint supernovae. Each point correspond to a field, a season and an observing strategy.}\label{fig:z95}
\end{center}
\end{sidewaysfigure}}

Another way to assess the quality of an observing strategy is to estimate the redshift detection limit for faint supernovae (per season and per field). On Figure \ref{fig:z95} is displayed the 95\% redshift limit (ie corresponding to the detection of 95\% of the supernovae of the corresponding sample) as a function of the number of faint supernovae. Huge variations among and inside strategies are observed. This plot reflects the quality of the proposed cadences. It seems that \redshift~of 0.7-0.75 may be reached with four fields. Once again \feature~tend to give the highest redshifts and the most homogeneous results among the fields and seasons. 


\subsubsection{Wide Fast Deep}
\label{sec:wide_fast_deep_analysis_method}

\myparagraph{Method}

WFD observations involve a large number of fields.  The observations
are dithered, and the size and depth of the final sample depends
heavily on the details of the observing strategy (in particular, the
filter allocation strategy and the field selection function \ldots)
For this reason, we opted for a slightly different approach, which we
describe below.

The celestial sphere is pixellized in Healpix superpixels\footnote{we
  choose nside=64, which corresponds to 0.8 deg$^2$ healpixels.  We
  have verified that (1) larger pixels (nside=32) leads to
  underestimating the number of SNe by \~ 15\% and
  that smaller pixels (nside=128 and above) give exactly the same
  results.}.  The directions / healpixel affected by a Galactic extinction $E(B-V)$ larger than 0.25
are masked and not included in our assessment. We consider only the $griz$ observations which are the ones
that matter to derive SN luminosity distances. 
Using a simple model of the LSST focal plane, we play
the cadence, and estimate the list of superpixels observed for a
given exposure. This allows us to build a log which reports the mjd,
band, and observing conditions of each healpixel observation.

We can then analyze this log, using as a probe, a fiducial SN~Ia,
e.g. the ``faint'' or ``normal'' SNe~Ia defined in the previous
section. For each mjd and each pixel, we determine:
\begin{eqnarray}
  z_{\mathrm{lim}} & = & \mathrm{max}\left(z | \mathrm{LC(z)\ fulfill\ requirements}\right) \\
  N_{z<z_{\mathrm{lim}}} &= & \delta\Omega_{\mathrm{pix}} \int_0^{z_\mathrm{lim}} \frac{\Delta T_{\mathrm{step}}}{1+z}\ {\mathcal{R}}(z)\ dV(z)
\end{eqnarray}
where $\delta\Omega_{\mathrm{pix}}$ is the solid angle subtended by
one pixel, $\Delta T_{\mathrm{step}}$ is the simulation time step (in
observer frame days) and $\mathcal{R}(z)$ is the SN~Ia volumetric rate
(we adopt the rate published in (Perrett et al, 2012)).  We also
compute the average cadence (in day$^{-1}$), i.e. the number of $g, r,
i$ or $z$ visits in a fiducial restframe interval.

The quantities above are determined for each pixel and each night
(identified by its mjd).  We report them in full sky maps, which give
an assessment of how the cadence performs in a $\sim 50$ day time
interval around the current mjd. From these maps, we can build global
maps giving, as a function of the position on the sky (1) the density
of supernovae (2) the median maximum redshift (3) the median cadence.
We can also collate these maps in videos, that are useful to evaluate
the observing strategy as a function of time.

\begin{sidewaysfigure}
    \begin{center}
      \includegraphics[width=\linewidth]{nsn/mothra_2045_02611.png}
      \caption{Example of cadence analysis maps.  Cadence {\em
          Mothra\_2045}, mjd=62464 (2029-11-24) {\em Upper panels:}
        (left) total number of well sampled supernovae per healpixel,
        (middle): median \zmed after 2611 days of survey, (right):
        median cadence, after 2611 days of survey. {\em Lower panels:}
        (left) number of SNe~Ia peaking at mjd=62464 and passing the
        light curve quality cuts (middle) \zmed, i.e.  maximum
        redshift at which a SN peaking at mjd=62464 would pass the
        requirements}
    \end{center}
\end{sidewaysfigure}

\myparagraph{Results}

We have conducted the analysis described above on all the cadences released
so far.  The final maps that report, for each healpixel (1) the final
number of SNe (2) the mean redshift limit
($\left<z_{\mathrm{faint|med}}\right>$) reached (3) the average
(restframe) cadence obtained are included in appendix
\ref{sec:wfd_maps} (figures \ref{fig:altsched_rolling_good_weather} to
\ref{fig:feature_baseline}). Finally, the animations that show the evolution
of the maps as the survey unfolds are available at the following address:
\begin{center}
  \href{http://supernovae.in2p3.fr/~nrl/lsst_sn_cadence}{http://supernovae.in2p3.fr/\~{}nrl/lsst\_sn\_cadence}
\end{center}
Inspecting these maps, we essentially recover the ranking of cadences
presented in the previous section.  We note that on some cadences,
especially the rolling cadences, the average properties (size and
depth) of the SN sample vary significantly as a function of the sky
direction. Some of these effects are related to seasonality (better
seeing allows to go deeper, for example) and depend on how realistic
the \opsim weather and seeing logs are. Others are clearly artefacts
of the observing cadence and can be corrected. For example, the on the
{\tt Mothra} and {\tt SLAIR} rolling cadences, are lines clearly
visible which correspond to depleted regions at the boundary between
the areas observed each year. The same is true on altsched rolling,
which also displays a strongly depleted region, at the region where
the scheduler switches between the upper and lower declination area.
Cures exists for these effects and are being discussed. 

Our primary metrics, ($z_{\mathrm{faint|median}}$, $N_{\mathrm{faint|median}}$) 
presented in section \ref{sec:metrics}, can be derived from these
maps.  Figures \ref{fig:nsn_zmax_med} and \ref{fig:nsn_zmax_faint}
give a synthetic presentation of our cadence evaluation, in the planes
(\zfaint, \nsnfaint) and (\zmed, \nsnmed), respectively.

\begin{sidewaysfigure}
  \begin{center}
    \includegraphics[width=\linewidth]{nsn/summary_plot_wfd_mediansn.pdf}
    \caption{Representation of the cadences analyzed in this study in
      the plane (\zmed, \nsnmed). This gives an assessement, for each
      cadence, of (1) the sample depth, i.e. at which redshift the
      median SN no longer passes the requirements listed in section
      \ref{sec:sn_sampling_requirements} and (2) the size of the
      subset of well-sampled SNe~Ia.}
    \label{fig:nsn_zmax_med}
  \end{center}
\end{sidewaysfigure}

\begin{sidewaysfigure}
  \begin{center}
    \includegraphics[width=\linewidth]{nsn/summary_plot_wfd_faintsn.pdf}
    \caption{Representation of the cadences analyzed in this study in
      the plane (\zfaint, \nsnfaint). This gives an assessement, for
      each cadence, of (1) the redshift limit of the survey, i.e. at
      which redshift the faintest SN no longer passes the requirements
      listed in section \ref{sec:sn_sampling_requirements} and (2) the
      size of the redshift limited SN~Ia sample produced by LSST.}
    \label{fig:nsn_zmax_faint}
  \end{center}
\end{sidewaysfigure}

Again, we recover our previous ranking.  The rolling version of
\altsched is the best performing cadence.  It will allow us to build a
very deep sample \zmed $\sim 0.45$ of more than well sampled 300,000
SNe.  {\tt altsched} covers more area, at the expense of a lower
cadence.  As a consequence, it is not as deep, but allows to obtain a
larger number of well-sampled SNe.

Most of the cadences released for the white paper call do not allow to
go as deep, no to secure as many well sampled SNe. The best \opsim and
\slair rolling cadences allow to reach redshift limits that are
similar to the non-rolling version of \altsched and yield samples that
are about 40\% smaller. All of them, in particular {\tt rolling mix}
and {\tt tms roll} implement the design principles exposed in section
\ref{sec:wfd_cadence_key_properties}. Why they do not reach the
performances of {\tt altsched rolling} is unclear at this point and is
being investigated.


\subsection{Overlap with 4MOST extragalactic surveys}

\myparagraph{Method}

The 4MOST facility will be a highly multiplexed, optical, fibre-fed
spectrograph mounted on the 4-metre VISTA telescope. It is due to
start operations at the end of 2022. Its wide field of view (4.1 sq
deg), high multiplex (1600 fibres feeding low-resolution spectrographs
suitable for extragalactic science plus $\sim 800$ fibres feeding a
high-resolution spectrograph, primarily for galactic science) and
geographical location (near the Paranal site in Chile) makes 4MOST an
ideal spectroscopic follow-up facility for 4MOST. The scientific
synergy potentially covers a very wide variety of science goals. Here
we focus on the synergy for follow-up of transients and varying
sources.

The 4MOST-TiDES survey plans to piggy-back on other 4MOST
extragalactic surveys, and will use a small subset of the fibres in
each pointing to observe live transients and the host galaxies of
previously-discovered transients. The 4MOST extragalactic surveys will
be carried out over a large fraction of the southern sky, shown in
Figure~\ref{4most_sky}.  There is also a component of TiDES that aims
to obtain spectral time-sequences of AGN in 4MOST's deep fields (that
are likely to coincide with at least a subset of LSST's deep fields,
TBD), for reverberation mapping.

\begin{figure}[!htbp]
\begin{center}
  \includegraphics[width=10.0cm]{overlap_4MOST/4most_fndep.png}
\end{center}

\caption{One realisation of the sky coverage of 4MOST extragalactic
  surveys, colour coded by number of visits. This is based on
  4MOST-4FS survey simulation round9/c/run01 dated 2018-5-14. Note
  that the 4MOST survey design is still in progress and the map is
  likely to change.}
\label{4most_sky}
\end{figure}

There is a preference for 4MOST to observe within the approximate
declination range seen in Figure~\ref{4most_sky} for several
reasons. One is not to duplicate other facilities in the North (such
as DESI, Subaru-PFS and WEAVE). Another is that the winds from the
north at Paranal can make it difficult to observe in that
direction. Similarly, 4MOST prefers not to observe far South because
of inefficiency when observing at high airmass. The 4MOST Atmospheric
Dispersion Compensator works optimally upto zenith distances of 55
degrees, equivalent to airmass of $\sim1.75$. At larger airmasses,
light is lost at the ends of the spectrum. Because VISTA is at
latitude $\sim -24.7$ degrees, one can therefore observe theoretically
to declination of $\sim -80$ degrees at the meridian. In practice
4MOST will not go much below declination of $-70$ degrees, except in
the Magellanic Clouds, as observations get much harder to schedule if
one wants to stay beyond airmass $\sim 1.75$ for an hour. However,
there could be exceptions if, for example, there was an interesting
deep field a bit north or south of the current range.

In addition to the extragalactic surveys, 4MOST will survey the Milky
Way disk area with the same 1600 fibres feeding low-resoution + 800
fibres feeding high resolution spectrographs, but this will be mainly
in bright time and geared toward brighter targets and hence less
suitable for extra-galactic follow-up. However, this survey might be
interesting for follow-up of stellar variables.

Finally, we note that a Call for Letters of Intent for 4MOST community
observations is expected to be opened next summer (2019).


\myparagraph{Results}

In order to compare the overall spatial coverage of 4MOST and the
various possible LSST survey strategies, we have divided the sky into
healpixels with NSIDE=256 (approx. 0.052 sq deg resolution). We are
considering only the WFD components of the LSST survey strategies
here.
For the LSST surveys we used each OpSim simulation (with a dither
pattern superimposed that gives a high spatial uniformity of depth) to
make a healpix map, and kept all healpixels with more than 500 visits
over 10 years.

\begin{comment}
\footnote{The code used to
  make these is the script
  {\tt https://github.com/rbiswas4/OpSimSummary/blob/master/scripts/make\_simlibs.py}}
\end{comment}

For 4MOST, we used the most recent 4MOST 4FS simulation,
(round9/c/run01 dated 2018-5-14) and made a Healpix map using the
central coordinates of each 4MOST tile, and assuming a hexagonal field
of view with area 4.1 sq deg. Each Healpixel with one or more visits
was kept.
 
We then calculated number of overlapping healpixels and multiplied by the
spatial area of one healpixel to give the overlapping areas shown in
Table~\ref{4most_overlap_tab}. Maps are shown in Figures~\ref{overlap_maps} - \ref{overlap_maps_c}.

\begin{table}[!htbp]
  \begin{center}
 \caption{Overlapping areas between LSST WFD and 4MOST extragalactic surveys}
\begin{tabular}{cc}\hline \hline
  LSST OpSim run & Overlapping area (sq deg) \cr\hline \hline
  mothra\_2045   &	11702 \cr
  colossus\_2667 &	12644 \cr
kraken\_2026   &	12644 \cr
kraken\_2036   &        12644 \cr
pontus\_2489   &	12644 \cr  
pontus\_2502   &	12644 \cr
colossus\_2664 &        12805 \cr
colossus\_2665 &	13096 \cr
pontus\_2002   &	15000 \cr
  \hline
\end{tabular}
\end{center}
\label{4most_overlap_tab}
\end{table}




\subparagraph{Temporal overlap}

4MOST will typically visit each part of the sky twice, for about one
hour exposure ($3\times$ 20 mins) exposures each time (but we
emphasise again that the details of the 4MOST survey are still under
discussion). Clearly, for follow-up of live transients it is important
that 4MOST observes in the areas of sky the LSST has visited recently
(within the lifetime of the transients concerned). For example if LSST
follows a rolling cadence then to maximise trabnsient science, 4MOST
should observe in the same declination range.  Coordination of LSST
and 4MOST surveys is essential.


\myparagraph{Conclusion}

Of the LSST cadence simulations considered, pontus\_2002 gives the
largest overall spatial overlap, of approx. 15,000 sq deg, compared to
approximately 12,000-13,000 sq deg for the other surveys. The extended
declination range of pontus\_2002 increases the overlap with 4MOST,
but goes a little too far in that some of extended LSST area is then
not covered by 4MOST.

As stressed above, temporal coordination of the 4MOST and LSST surveys is
essential if transient science is to be maximised.


\subsection{Peculiar velocity}

\myparagraph{Method}

Peculiar velocities provide a measure of $f\sigma_8$, which in turn probes gravity.  As precise distance indicators Type~Ia supernovae (SNe~Ia)
can provide precise peculiar velocities of their host galaxies \citep{2006PhRvD..73l3526H,2011ApJ...741...67D}.

\citet{2017ApJ...847..128H} calculated the  expected precision of $f\sigma_8$ using LSST-discovered SNe~Ia.
The Fisher information matrix of a random Gaussian field with mean zero and covariance $C(k)$ parameterized by $\lambda$ is
\begin{equation}
F_{ij} = \frac{V}{2}\int \frac{d^3k}{(2\pi)^3} \text{Tr}\left[ C^{-1} \frac{\partial C}{\partial \lambda_i} C^{-1}
\frac{\partial C}{\partial \lambda_j} \right].
\end{equation}
The covariance
\begin{equation}
C = P_{vv}(k) + \frac{\sigma^2}{n}
\end{equation}
has contributions from the power spectrum, noise in the velocity measurement, and the density of velocity probes.
In the sample variance limit for a sample with fixed depth, the variance in $f\sigma_8$ (and other $\lambda$ parameters)
is thus inversely proportional to the survey solid-angle $\Omega$, whereas
in the shot-noise limit the variance is inversely proportional to $\Omega n^2 \propto N^2/\Omega$, where $n$ is the number density,
and $N$ is the total number of supernovae.  


\myparagraph{Results}
N.~Regnault kindly provided us with his assessment of the number and solid-angle of $z<0.2$ supernova pre-maximum discoveries 
for the survey candidates provided by the Project.  Based on these numbers we calculate the Figures of Merit in the two regimes 
(normalized with respect to the average of all the surveys),
with our results shown in 
Table~\ref{table:ref}.  The results of \citet{2017ApJ...847..128H}  indicate that after 10 years,
LSST supernovae are at neither extreme; we thus adopt the average of the two FoM's as what we advocate for the survey.


\begin{table}
\caption{SN~Ia Figures of Merit of Project survey candidates
in  the shot-noise and the survey-variance limits.  The final column is the average
of the two Figures of the Merit.\label{table:ref}}
\centering
\begin{tabular}{crrr}
  \hline
  \hline
Survey & FoM (shot) & FoM (survey) & FoM (avg)\\
\hline
\hline
feature\_baseline\_10yrs & 0.025 & 0.642 & 0.334  \\
blobs\_same\_zmask10yrs & 0.048 & 0.859 & 0.453  \\
blobs\_same\_10yrs & 0.064 & 0.859 & 0.462  \\
feature\_rolling\_half\_mask\_10yrs & 0.175 & 0.987 & 0.581  \\
minion\_1016 & 0.093 & 1.109 & 0.601  \\
pontus\_2502 & 0.132 & 1.107 & 0.62  \\
feature\_rolling\_twoThird\_10yrs & 0.187 & 1.104 & 0.645  \\
mothra\_2045 & 0.226 & 1.134 & 0.68  \\
pontus\_2002 & 0.224 & 1.157 & 0.691  \\
rolling\_10yrs & 0.581 & 0.838 & 0.709  \\
nexus\_2097 & 0.28 & 1.166 & 0.723  \\
kraken\_2035 & 0.705 & 0.823 & 0.764  \\
tms\_roll\_10yrs & 0.424 & 1.119 & 0.771  \\
kraken\_2036 & 0.437 & 1.141 & 0.789  \\
baseline2018a & 0.779 & 0.834 & 0.807  \\
kraken\_2042 & 0.794 & 0.858 & 0.826  \\
colossus\_2665 & 0.797 & 0.861 & 0.829  \\
colossus\_2664 & 0.814 & 0.884 & 0.849  \\
alt\_sched\_rolling & 0.804 & 0.92 & 0.862  \\
altsched\_rolling\_good\_weather & 0.815 & 0.92 & 0.868  \\
tight\_mask\_simple\_10yrs & 0.6 & 1.141 & 0.871  \\
kraken\_2026 & 0.947 & 0.838 & 0.892  \\
tight\_mask\_10yrs & 0.646 & 1.155 & 0.901  \\
mothra\_2049 & 0.671 & 1.161 & 0.916  \\
rolling\_mix\_75\_10yrs & 0.85 & 1.007 & 0.928  \\
tms\_drive\_10yrs & 0.806 & 1.09 & 0.948  \\
rolling\_mix\_10yrs & 0.958 & 1.011 & 0.984  \\
cadence\_mix\_10yrs & 1.012 & 1.012 & 1.012  \\
blobs\_mix\_zmask10yrs & 1.432 & 1.008 & 1.22  \\
pontus\_2489 & 2.052 & 0.97 & 1.511  \\
colossus\_2667 & 2.365 & 1.004 & 1.684  \\
alt\_sched & 2.788 & 0.91 & 1.849  \\
kraken\_2044 & 2.565 & 1.162 & 1.863  \\
alt\_sched\_twi & 2.816 & 0.927 & 1.871  \\
altsched\_18\_\_90\_30 & 2.581 & 1.187 & 1.884  \\
altsched\_18\_\_90\_40 & 2.594 & 1.187 & 1.89  \\
altsched\_good\_weather & 2.914 & 0.91 & 1.912  \\
\hline
\end{tabular}
\end{table}

\begin{comment}
rolling\_10yrs  & 0.581 & 0.838 & 0.709  \\
mothra\_2045  & 0.226 & 1.134 & 0.680  \\
blobs\_same\_zmask10yrs  & 0.048 & 0.859 & 0.453  \\
rolling\_mix\_75\_10yrs  & 0.850 & 1.007 & 0.928  \\
alt\_sched\_twi  & 2.816 & 0.927 & 1.871  \\
alt\_sched\_rolling  & 0.804 & 0.920 & 0.862  \\
pontus\_2489  & 2.052 & 0.970 & 1.511  \\
kraken\_2044  & 2.565 & 1.162 & 1.863  \\
pontus\_2002  & 0.224 & 1.157 & 0.691  \\
tms\_drive\_10yrs  & 0.806 & 1.090 & 0.948  \\
tight\_mask\_simple\_10yrs  & 0.600 & 1.141 & 0.871  \\
baseline2018a  & 0.779 & 0.834 & 0.807  \\
altsched\_18\_\_90\_40  & 2.594 & 1.187 & 1.890  \\
kraken\_2026  & 0.947 & 0.838 & 0.892  \\
feature\_baseline\_10yrs  & 0.025 & 0.642 & 0.334  \\
blobs\_same\_10yrs  & 0.064 & 0.859 & 0.462  \\
cadence\_mix\_10yrs  & 1.012 & 1.012 & 1.012  \\
feature\_rolling\_half\_mask\_10yrs  & 0.175 & 0.987 & 0.581  \\
pontus\_2502  & 0.132 & 1.107 & 0.620  \\
colossus\_2667  & 2.365 & 1.004 & 1.684  \\
kraken\_2036  & 0.437 & 1.141 & 0.789  \\
tms\_roll\_10yrs  & 0.424 & 1.119 & 0.771  \\
rolling\_mix\_10yrs  & 0.958 & 1.011 & 0.984  \\
altsched\_rolling\_good\_weather  & 0.815 & 0.920 & 0.868  \\
altsched\_18\_\_90\_30  & 2.581 & 1.187 & 1.884  \\
tight\_mask\_10yrs  & 0.646 & 1.155 & 0.901  \\
alt\_sched  & 2.788 & 0.910 & 1.849  \\
colossus\_2665  & 0.797 & 0.861 & 0.829  \\
colossus\_2664  & 0.814 & 0.884 & 0.849  \\
mothra\_2049  & 0.671 & 1.161 & 0.916  \\
kraken\_2042  & 0.794 & 0.858 & 0.826  \\
minion\_1016  & 0.093 & 1.109 & 0.601  \\
kraken\_2035  & 0.705 & 0.823 & 0.764  \\
nexus\_2097  & 0.280 & 1.166 & 0.723  \\
blobs\_mix\_zmask10yrs  & 1.432 & 1.008 & 1.220  \\
feature\_rolling\_twoThird\_10yrs  & 0.187 & 1.104 & 0.645  \\
altsched\_good\_weather  & 2.914 & 0.910 & 1.912  \\
\end{comment}

\myparagraph{Conclusion}

We are in the course of refining the Figure or Merit, taking into account survey geometry, combining both survey variance and shot noise.
Any significant changes in our findings will lead to an update of our relative
ranking of the surveys.








\subsection{Photometric classification}

\myparagraph{Method}

The aim of this investigation is to analyse the affect a particular cadence has
on ones ability to photometrically classify supernova. This investigation has been carried out by the
developers of {\tt snmachine} who work in the Supernova Working Group under the umbrella of the Dark Energy Science
Collaboration (DESC). {\tt snmachine} is a DESC product that is used as a photometric
classification pipeline \cite{lochner2016photometric}.

The motivation for this work comes from the desire to identify as many Supernova
as Type 1a in order to help constrain the nature of Dark Energy.
LSST will observe many more Supernova than ever before, but at a rate that is
not feasible for all of these transients to be spectroscopically followed up and
confirmed as Type 1a or not.
Thus, the ability to classify these objects photometrically will be very
important. If one can obtain a greater set of Type 1a's, an updated Hubble
digram plot can be produced and the fundamental parameters of the cosmological
model can be tested further.

In order to conduct this analysis, a third
party software, {\tt SNANA}, has been used to generate light curves that correspond to
different cadences runs from \opsim outputs. The generated light curves are
then used as inputs in the {\tt snmachine} pipeline, an example of which is
shown below.

An interpolation is done for the between the sampled points to produce a smooth
light curves with one can then apply a wavelet decomposition to. The resulting
wavelet coefficients are then processed further to reduce dimensionality using a
principle component analysis. These principle components are then
provided as features to a random forest classifier. To ensure a controlled test,
for each cadence run a classifier is trained on 2000 light
curves only and then tested on the remaining set of light curves that are in the
corresponding dataset produced from {\tt SNANA} in relation to specific \opsim cadence
simulations. 

\myparagraph{Results}

%The results for which are shown below in Figure\ref{fig:rocs}
\begin{figure}
  \centering
  \subfigure[DDFY1]{\label{fig:ddfy1}\includegraphics[width=0.8\textwidth]{classification/photometric_classification_roc_results_ddfY1.png}}
    \subfigure[DDFY10]{\label{fig:ddfy10}\includegraphics[width=0.8\textwidth]{classification/photometric_classification_roc_results_ddfY10.png}}
   \caption{Comparison for the DDFY1 and DDFY10 ROC curves}\label{fig:rocs}
\end{figure}

Figure\ref{fig:ddfy1}~shows the comparative classification performance between 13 cadences for the Deep
Drilling Fields of Year 1 whereas Figure\ref{fig:ddfy10}~compares the same
cadences for the Deep Drilling Fields over the entire survey.

The area under the Receiver Operating Characteristic (auROC) curves were chosen as the metric
to evaluate the performance. It was felt this single scalar would be most useful
for being able to differentiate the ability of the various cadence strategies
to perform photometric classification.


\myparagraph{Conclusion}

By interpolating the sampled light curved with Gaussian process and then applying a
wavelet decomposition to these interpolated light curves, one obtains features
that can be provided to a classifier, in this case a Random Forest algorithm.
The performance of the interpolation is directly affected by the amount of
samples one has on the light curve. More samples improves the reliability of the
Gaussian processes and thus provided better features through via the wave decomposition.

Therefore it can be understood that in order to classify transients, short sampling of a light
curve is important. This is particularly important for early classification
leading to possible spectroscopic follow up. A well sampled light curve would
indeed boost classification rate. Thus, before any analysis is conducted, one
could assume a rolling cadence with 2 to 3 days would be beneficial to the
cause.
With this in mind, the results of the analysis shown here do not in particular
point to a specific cadence as be favourable, but the performance of some
strategies do not perform as well as others in the Year 1 case and over the
Year 10 full survey.


This research can be taken further in several ways and there are plans to
continue this analysis for Wide-Fast-Deep cadence runs also. Work is currently
taking place to investigate the performance of classifiers trained on DDF runs
and then tested on WFD. Work is also being carried out for a comparative stufy
of WFD cadence runs much like what has been shown above for DDFY1 and DDFY10.
Further to the two DDF runs presented, further analysis will be carried out for
more years to see the performances for years in between and to see if there
are classification trends that change through the lifetime of the survey. It is
also hoped that number of Supernova used for training affects classification
performances, this work is done but alternations are required to the code for a complete
analysis. Finally it would be beneficial to understand the elements of each cadence
strategy to give further insight as to why certain cadences perform the way they do.


\subsection{$w_0~w_a$ constraints}

\myparagraph{Method}

In order to assess the impact of the cadence on the cosmological analysis possible with supernovae, we compute the Dark Energy Task force figure of merit \cite{2006astro.ph..9591A} in the Dark Energy parameters $w_0-w_a,$ for a few scenarios. The supernova sample is simulated using (\strech, \sncolor) distributions estimated from \cite{2016ApJ...822L..35S} (High-z G10 parameters) and production rates from  \cite{2008ApJ...682..262D}.

In the first instance, the supernova sample is simulated following roughly the same prescription as that outlined in the recent Science Requirements Document \cite{2018arXiv180901669T}, with small changes to the host redshift selection. For the SRD we adjusted the survey size simulated to ensure roughly 112 000 SNe after host selection cuts from a 4MOST-like ground based telescope. This is the largest determinant of the final size, and so in order to test for differences in the survey strategy we initially doubled the survey size simulated, and then tested a few cadences without requiring any spectrosopic host redshift. This was to ensure that the host-$z$ follow up was not the most important characteristic.
In addition, we imposed a more restrictive cut on the fitted colour in the SAL2T fit from $\sigma(c) < 0.08$ to $\sigma(c) < 0.05.$ Insodoing we are forcing that the supernovae that survive are of higher quality than in the DESC SRD.

In order to test more strongly the impact of the wide field cadence on the overall cosmology, we ran simulations including \textit{only the WFD survey} in addition to a low-$z$ data set. (e.g.  Foundation-like: 2400 \sne~with $z~\less$ 0.1). We compared the cosmology results from the WFD+Foundation surveys for the {\tt kraken\_2026, kraken\_2035} and {\tt altsched, altsched\_rolling} cadences. Finally we show the results with and without spectroscopic host redshift selection for the baseline {\tt kraken\_2026} survey.

We have included only statistical errors: for speed of computation we have neglected the astrophysical systematics. This will be relaxed in future versions.


\begin{figure}[!htbp]
  \begin{center}
  \subfigure[w0-wa ellipses]{\label{fig:w0wa}\includegraphics[width=0.7\textwidth]{w0wa/FM_plot_cadence_updated.png}}
    \subfigure[FoM vs observing strategy]{\label{fig:snfom}\includegraphics[width=0.7\textwidth]{w0wa/FoM_cadence_updated.png}}
    \caption{Cosmology constraints across cadence types: the ellipses in the $w_0-w_a$ plane (a) and the Dark Energy figure of merit is shown (b) are shown. }
    \end{center}
\end{figure}
\myparagraph{Results}
The largest difference in the Figure of Merit (Fig. \ref{fig:w0wa}) we find is a factor of two between the highest FoM (kraken\_2035) and the lowest FoM (pontus\_2502) if both WDF and DDF surveys are taken into account. The lowest-performing strategy has two alternating bands in declination, switching in alternate years, which is expect to perform poorly. The fact kraken\_2035 is performing better compared to other observing strategies may be explained by the higher number of DD fields observed (9 whereas all other strategies considered 5 fields).

\altsched~and kraken\_2026 show highest FoMs in WFD-only scenarios (Fig. \ref{fig:snfom}) followed by rolling\_mix, kraken\_2035 , \altsched\_rolling and rolling observing strategies. One may observe that these cadences exhibit the highest FoMs as the following classification shows:
\begin{itemize}
\item{500 $\leq$ FoM}: kraken\_2035, \altsched, kraken\_226, rolling\_mix, kraken\_2035 , \altsched\_rolling and rolling ;
\item{400 $\leq$ FoM $\leq$ 500}: pontus\_2489, kraken\_2042, colossus\_2665, colossus\_2664, kraken\_2044, pontus\_2002 ;
\item{FoM $\leq$ 400}: nexus\_2049, nexus\_2097, kraken\_2026.
\end{itemize}

In the same way as for the number of well-sampled supernovae metric, it is possible to estimate the number of supernovae as a function of the  redshift (observation) limit for \sne~selected to estimate the FoM. The mean redshift is in the range 0.4-0.5 (Fig. \ref{fig:w0wazlim}) and up to 225k supernovae may be collected after ten years of operation.

\begin{figure}[!htbp]
  \begin{center}
    \includegraphics[width=0.8\textwidth]{w0wa/test_numz.pdf}
    \caption{Number of \sne~as a function of the mean redshift. Parameters of the supernovae collected were used as inputs to estimated the FoMs.}
    \label{fig:w0wazlim}
  \end{center}
\end{figure}

A comparison of the deep fields to the WFD for the nominal cadence (kraken\_2026) has shown that the DDF-only survey has about a quarter the number as WFD-only, with mean redshifts of $z~=$ 0.73 and  $z~=$ 0.41 respectively. The DDF survey could enable a competitive dark energy constraint by itself, but two-third of the of the constrainig power will come from WFD: the FoM is equal to 460, 262, 337 for DDF+WFD, DDF, WFD (all +low-$z$) respectively.

\myparagraph{Conclusion}

Even if we may keep in mind that the results presented do not included calibrated (astrophysical) systematics and hence may be optimistic, quite a strong dependence of the FoM values on observing strategies may be noticed: FoM are in the range [340-560] that is a factor 1.6 between the highest and the smallest values. Some of the proposed observing strategies may not be so far to reach the FoM value of 500 (target of the Stage-IV surveys) using supernovae alone.




\section{Metrics and proposed observing strategies: summary}
 


\begin{table}[!htbp]
  \begin{center}
    \caption{Top 5 of the best observing strategies.}\label{tab:summary_top}
\begin{tabular}{clcc}
  \hline
  \hline
Metric  & Top 5 (ranked) & Remark \\
\hline
\hline
{\bf $N_{SN}$ (well-sampled)}  &  & \\
   & 1. altsched/altsched wide & \\
WFD             & 3. Rolling mix & \\
             & 4. colossus\_2667 &  \\
& 5. feature\_rolling\_2/3 & \\
   & 1. kraken\_2025/feature\_baseline\_10yrs &  \\
DDF & 3. colossus\_2665 &  \\
& 4. colossus\_2667 & \\
& 5. pontus\_2002 &  \\
\hline
{\bf Peculiar velocity} & 1. altsched/altsched wide &  \\
                & 3. kraken\_2044 &  \\
      WFD                          & 4. colossus\_2667 & \\
                               & 5. pontus\_2489 & \\
      \hline
{\bf Photometric}   & 1. pontus\_2489/kraken\_2042 &  \\
{\bf Classification}  & 3. colossus\_2665 &  \\
      DDF             & 4. mothra\_2045 &  \\
                               & 5. colossus\_2667 &  \\
      \hline
          {\bf w0wa FoM}  & 1. kraken\_2035 &  \\
& 2. colossus\_2667 & \\
      WFD+DDF             & 3. pontus\_2489 &  \\
      & 4. colossus\_2026 &  \\
      & 5. kraken\_2042 &  \\
      \hline
 {\bf Overlap with 4MOST}  & 1. pontus\_2002 &  \\
     & 2. colossus\_2667 & \\
   WFD                   & 3. colossus\_2664 &  \\
      & 4. pontus\_2502 &  \\
      & 5. pontus\_2489 &  \\
     \hline
\end{tabular}
\end{center}
\end{table}


\begin{table}[!htbp]
  \begin{center}
    \caption{Worst 5 observing strategies.}\label{tab:summary_worst}
\begin{tabular}{clcc}
  \hline
  \hline
Metric  & Worst 5 (ranked) & Remark \\
\hline
\hline
{\bf $N_{SN}$ (well-sampled)}  &  & \\
WFD   & 1. feature\_baseline\_10yrs & \\
             & 2. blobs same zmask & \\
             & 3. blob same &  \\
& 4. pontus\_2002 & \\
& 5. pontus\_2502 & \\
%\hdashline
DDF    & 1. kraken\_2036 &  \\
& 2. mothra\_2045 &  \\
& 3. pontus\_2489 & \\
& 4. colossus\_2664 &  \\
& 5. baseline\_2018a &  \\
\hline
{\bf Peculiar velocity} & 1. feature\_baseline\_10yrs &\\
      WFD                & 2. blobs same zmask &  \\
                               & 3. blob same &  \\
      & 4. feature\_rolling\_1/2 & \\
      & 5. minion\_1016 & \\
      \hline
{\bf Photometric}   & 1. kraken\_2044 &  \\
{\bf Classification}  & 2. kraken\_2036 &  \\
      DDF             & 3. kraken\_2035 &  \\
                               & 4. kraken\_2026 &  \\
      \hline
          {\bf w0wa FoM}  & 1. pontus\_2502 &  \\
& 2. nexus\_2097 & \\
      WFD+DDF             & 3. mothra\_2049 &  \\
      & 4. mothra\_2045 &  \\
      & 5. kraken\_2036 &  \\
      \hline
 {\bf Overlap with 4MOST}  & 1. mothra\_2045 &  \\
     WFD       & 2. colossus\_2667 & \\
                   & 3. kraken\_2026 &  \\
      & 4. kraken\_2036 &  \\
      & 5. pontus\_2489 &  \\
     \hline
\end{tabular}
\end{center}
\end{table}


\section{New proposals}

\subsection{DDF observing strategy}

In OpSim the choice to observe a given field is done on the fly according to an optimized selection function depending on criteria including (among others) observing conditions, slew time minimization, last time of visit, and total observing time. While the use of a simple metric (see below) would help in choosing another approach could be to use a pre-defined table that would specify which DDF are to be observed on a given night.

Let us consider the four reference fields \cosmos, \xmmlss, \cdfs~and \elais. Since the location are well known, it is possible to estimate when these fields are visible (that is with an altitude between 20 and 86.5 degrees) for a large enough period (typically 20 to 40 minutes) of good observing conditions (ie with airmass lower than a reference value - typically 1.5). We may define a (boolean) parameter dubbed observability equal to one when the above-mentioned conditions are filled, and 0 otherwise. This parameter is displayed on Figure \ref{fig:observability} (top). \xmmlss,\cdfs, and \elais~ are located in the same (Ra) region and may thus be observed during the same night whereas \cosmos~may be observed when all others are not visible. Season lengths of observability (Figure \ref{fig:observability} - bottom) range from 200 to 290 days.


\begin{figure}[!htbp]
\begin{center}
  
  \includegraphics[width=18cm,height=12cm]{new_proposals/observability.png}
  \includegraphics[width=18cm,height=8cm]{new_proposals/Season_length.png}
 \caption{Top: observability (see definition in the text) and airmass as a function of the night number (first year of LSST operation). Bottom: season length as a function of the season for an airmass limit of 1.5 (full lines) and 1.9 (dashed lines).}\label{fig:observability}
\end{center}
\end{figure}

The table of observations may be defined using median gap values \tgapcosmos~and \tgapothers~when \cosmos~and \xmmlss,\cdfs,\elais~are not observable, respectively. Observing time windows are then defined by:
\begin{itemize}
\item{[\tgapothers-w/2,\tgapothers+w/2]: \cosmos~ is observed}
\item{[\tgapcosmos-w,\tgapcosmos+w]: \xmmlss, \cdfs, \elais~are observed.}
\end{itemize}
with a width w equal to 2/3*(\tgapcosmos-\tgapothers).

\subsection{altsched rolling 80/20}

\subsection{altsched rolling 75/25}

\subsection{A simple metric for SN that may be implemented in OpSim/SLAIR}
If we fit a light curve model $L(t) = A \times \ell(t)$ on a
lightcurve $(t_i, y_i, \sigma_i)$, the least square estimate of $A$ is
given by:
$$
\hat{A} = \frac{\sum w_i \ell_i y_i}{\sum w_i \ell_i^2}
$$
and the signal-to-noise ratio on $\hat{A}$ is:
$$
SNR = \sum_i (w_i L_i^2)^{1/2}
$$ since we are in the background dominated regime, the weights may be
expressed as a function of the $5-\sigma$ limiting flux of each visit
$f_{i|5}$, and we have:
$$
SNR_{\mathrm{band}} = \sum_{i} 5 \times (f^{-2}_{i|5} L_i^2)^{1/2}
$$
where $5-\sigma$ limiting flux of each visit $f_{i|5}$.

This metrics is simple in the sense that it does not require to use a SN
light curve fitter. One just need lightcurve templates and the
limiting magnitudes of each visit -- given in the cadence
databases. In practice, using $SNR_g > 30, SNR_r > 40, SNR_i > 30$
(for $z<0.3$), and $SNR_r > 40$, $SNR_i > 30$ and $SNR_z > 20$ for
$z>0.3$ allows to fulfill the requirement on color resolution above.


\section{Conclusion}

\addcontentsline{toc}{section}{Bibliography}
\bibliographystyle{plain}
\bibliography{biblio}

\begin{comment}
\begin{thebibliography}{9}
\expandafter\ifx\csname natexlab\endcsname\relax\def\natexlab#1{#1}\fi
\providecommand{\url}[1]{\href{#1}{#1}}
\providecommand{\dodoi}[1]{doi:~\href{http://doi.org/#1}{\nolinkurl{#1}}}
\providecommand{\doeprint}[1]{\href{http://ascl.net/#1}{\nolinkurl{http://ascl.net/#1}}}
\providecommand{\doarXiv}[1]{\href{https://arxiv.org/abs/#1}{\nolinkurl{https://arxiv.org/abs/#1}}}


\bibitem{perrett} Evolution in the Volumetric Type Ia Supernova Rate from the Supernova Legacy Survey, K.Perrett {\it et al}, The Astronomical Journal, Volume 144, Issue 2 (2012).

\bibitem[Davis et al.(2011)]{2011ApJ...741...67D} Davis, T.~M., Hui, L., Frieman, J.~A., et al.\ 2011, \apj, 741, 67.  
\bibitem[Hui, \& Greene(2006)]{2006PhRvD..73l3526H} Hui, L., \& Greene, P.~B.\ 2006, \prd, 73, 123526.
  
\bibitem[Howlett et al.(2017)]{2017ApJ...847..128H} Howlett, C., Robotham, A.~S.~G., Lagos, C.~D.~P., et al.\ 2017, \apj, 847, 128.

%\bibitem{2011ApJ...741...67D} Davis, T.~M., Hui, L., Frieman, J.~A., et al.\ 2011, \apj, 741, 67.  
%\bibitem{2006PhRvD..73l3526H} Hui, L., \& Greene, P.~B.\ 2006, \prd, 73, 123526.
  
%\bibitem{2017ApJ...847..128H} Howlett, C., Robotham, A.~S.~G., Lagos, C.~D.~P., et al.\ 2017, \apj, 847, 128.
  
 \end{thebibliography}
\end{comment}

\clearpage


\appendix
  

\input appendix.tex


\end{document}


\pagebreak
\section{Weak Lensing}\label{sec:wl}

\pagebreak
\section{Static probes}

\newcommand{\todorm}[1]{\textbf{#1}}

Here we present results for joint cosmological forecasts of the ``static'' probes: WL, LSS, and CL.
Unlike Section~\ref{sec:wl}, which focuses specifically on systematics mitigation for weak lensing,
this section focuses on dark energy constraining power.  The forecasting methodology follows
that in the DESC Science Requirements Document v1 \citep{DESCSRD2018}, though with modified
assumptions about the baseline LSST analysis for Y1 and Y10 based on the new strategies, and filling
in Y3 and Y6 forecasts as well.

\todorm{Say why Y3 and Y6.  Tie results to LSS, WL section results.}


\pagebreak

% \input{joint}
% \pagebreak

\bibliographystyle{unsrt}
\bibliography{refs}


\end{document}
