\section{Photometric Redshifts}\label{sec:pz}

A one page summary of our analysis of the {\tt OpSim} and {\tt ALTSched} results is provided in Section \ref{mlg_sec:pz_summary}. A more in-depth evaluation of how the photometric redshifts derived from LSST photometry will evolve over the 10 year survey, for a variety of observing strategies that includes the formal {\tt OpSim} runs along with more general scenarios in which e.g., 5\% of the total observing time is redistributed to or from each filter in turn, is currently underway and being documented in this Overleaf draft: \url{https://www.overleaf.com/read/fgnvddbnrmgk}.


% % % % % % % % % % % % % % % % % % % % % % % % % % % % %
\subsection{Evaluating the Photometric Evolution of the Observing Strategies}\label{ssec:pz_opsim_phot}

To evaluate the photometric quality produced by each of the proposed observing strategies, we use the Metric Analysis Framework (MAF) as follows.

{\bf Slicer:} We use a HEALpix slicer with {\tt nside} $=16$ and a random dither to the field RA and Dec ({\tt latCol = randomDitherFieldPerNightRa} and {\tt lonCol = randomDitherFieldPerNightDec}). This slicer returns metrics in fields with an approximate resolution of $220$ arcminutes ($3.67$ degrees), similar to a single LSST pointing, and the dither helps to smooth out field-to-field variations in a realistic way. The dither option is not needed or applied with simulations from the feature-based scheduler ({\tt rolling\_10yrs}) or {\tt ALTSched}. 

{\bf Metric:} We use {\tt ExgalM5}, which returns the $5{\sigma}$ limiting magnitude for a given filter, corrected for Galactic dust extinction. We require that a given field have observations in all $6$ filters to be considered as conceivably be appropriate for photo-$z$ estimation (and recall that we impose the constraint that a galaxy must be {\it detected} in $griz$ to get a photo-$z$ estimate). We furthermore impose that the Galactic extinction $E(B-V) \leq 0.2$ mag and that the $5{\sigma}$ $i$-band detection limit be at least $24.5/25.0/25.5/26.0$ magnitudes at $1/3/6/10$ years.

{\bf Bundle:} We run the metric and slicer for years $1$, $3$, $6$, and $10$. We only consider images that are obtained as part of the WFD survey component, to avoid depth pockets from mini-surveys or deep drilling fields. We furthermore only consider fields with a Galactic longitude of $60 < l < 300$ degrees {\em or} Galactic latitude of $b>30$ degrees, as would a cosmological science analysis relying on the LSST photo-$z$, but the $E(B-V) \leq 0.2$ mag is actually more constraining than this location constraint.

Using the above slicer, metric, and bundle, we generate a single file per year per {\tt OpSim} or {\tt ALTSched} run, in which each row represents a ``field" that was observed in all $6$ filters as part of the WFD survey, and the columns are right ascension, declination, and the $5{\sigma}$ limiting magnitude in filters $ugrizy$. It turns out that many simulations result in a similar evolution for the median limiting magnitude for each filter as a function of survey year, and thus would not produce different photo-$z$ results, except for 3: {\tt pontus\_2002}, which has a larger WFD survey area of $24700$ square degrees ($-78 < \delta < +18$); {\tt pontus\_2489}, which does $20$-second visits in $grizy$ filters and $40$-second visits in $u$-band, increasing the $u$-band depth and leading to a larger total number of visits in the others filters; and {\tt rolling_10yrs}, a rolling cadence that alternates between two bands of declination. 


% % % % % % % % % % % % % % % % % % % % % % % % % % % % %
\subsection{The CMNN Photo-$z$ Estimator}\label{ssec:pz_exp_cmnn}

For this work we have used the color-matched nearest-neighbors (CMNN) photometric redshift estimator from \cite[][herafter G18]{2018AJ....155....1G}. The CMNN should not be taken as representing the ``best" photo-$z$ estimator or the ``official" LSST algorithm -- it is neither of these things. It is a simple algorithm that produces photo-$z$ results of a statistical quality that directly correlates with the photometric quality of the input, and thus is very useful for evaluating the impact on photo-$z$ of any changes to the LSST photometric quality -- such as the {\tt OpSim} runs that we consider in this work.

As described in G18, this estimator uses a training set of galaxies with known redshifts and a test set for which photo-$z$ are to be estimated. For each galaxy in the test set, the estimator first identifies a color-matched subset of training galaxies by calculating the Mahalanobis distance in color-space between the test galaxy and all training-set galaxies. The Mahalanobis distance in this case is the difference between the test- and training-set galaxy color, divided by the photometric error of the test-set galaxy color, summed over all available colors (Equation 1 in G18). Then, a threshold value is applied that defines a ``good" color match. This threshold is set by the percent point function (PPF): for example, for $N_{\rm dof}=5$, PPF$=95$ per cent of all training galaxies consistent with the test galaxy will have $D_M < 11.07$ (where $N_{\rm dof}$, the number of degrees of freedom, is the number of colors). The estimator then chooses one of the color-matched training-set galaxies (e.g. the nearest-neighbor or a random selection), and uses that galaxy's known redshift as the test-set galaxy's photo-$z$ (a "redshift donor"). The uncertainty in the photo-$z$ estimate is taken to be the standard deviation of the true redshifts of training-set galaxies in the color-matched subset. 

Compared to G18, there are some minor differences in how this photo-$z$ estimator was applied in this work. Here, we require that a test-set galaxy be detected in the LSST filters $griz$ and thus have colors $g-r$, $r-i$, and $i-z$ or else a photo-$z$ estimate is not attempted, and we've used a threshold of PPF=$0.68$ to define the color-matched subset of training galaxies. Unlike G18, we choose randomly from the color-matched subset of training galaxies instead of choosing the nearest neighbor. This difference leads to less accurate and less precise photometric redshifts, but by using a random selection in the photo-$z$ estimator, the {\tt OpSims} runs that degrade/enhance the LSST photometry end up having a larger relative impact on the resulting photometric redshifts. This helps with the experiment at hand, which is to assess the relative impacts of different {\tt OpSim} runs, and {\em not} to generate the best possible photo-$z$ results for each {\tt OpSim} run.

As a final note, to accelerate processing time, we have applied both the color and magnitude pre-cuts to the training set, as described in G18. The color cut is fairly benign, but the magnitude pre-cut effectively works as a ``pseudo-prior" by cutting down the training set to the 10\% of training galaxies with an $i$-band magnitude nearest to the test galaxy's $i$-band magnitude. The ``pseudo-prior" may improve accuracy of the photo-$z$ estimate for some, but can also introduce a bias in the results. Since all of the experiments in this work will be looking at {\it relative} changes to the photo-$z$ quality as various inputs are changed, this kind of degradation to the {\it absolute} photo-$z$ quality is acceptable in this case.


% % % % % % % % % % % % % % % % % % % % % % % % % % % % %
\subsection{Simulating LSST Photometry for the Training and Test Catalogs} \label{ssec:pz_exp_cats}

For this work we use the same simulated mock galaxy catalog as used in G18. This catalog contains the ``true" redshift and the ``true" apparent magnitudes in $ugrizy$ for all galaxies. The training and test set galaxies are drawn randomly from this catalog. We use $1\times10^6$ galaxies for the training set, and $5\times10^4$ galaxies for the test set. Justification for the sizes of the test and training sets is provided in G18, but here we note that the size of the test set is adequate to achieve statistical measures in the high-$z$ bins.

For the training set, we simulate observed apparent magnitudes for all galaxies assuming that the $5{\sigma}$ photometric depths in each filter are equivalent to the LSST baseline 10-year survey: $u=26.09,g=27.38,r=27.53,i=26.83,z=26.06,y=24.86$. This is the same as what was used in G18. For the test set, we also simulate observed apparent magnitudes for all galaxies using the $5{\sigma}$ photometric depths in each filter, but these depths change for the different LSST observing strategies that we consider in this work. We randomly assign each test galaxy to one of the simulated ``fields" from our MAF, as described above in Section \ref{ssec:pz_opsim_phot}. For all training- and test-set galaxies, once we know the $5{\sigma}$ photometric depth to apply, we derive the expected photometric error based on each galaxy's ``true" catalog magnitude (using Equation 5 from \citealt{2008arXiv0805.2366I}), and then add a random scatter proportional to this uncertainty to simulate observational uncertainties. This method is described in more depth, with plots of error {\it vs.} apparent magnitude, in G18.

In all of our experiments, both the test and training sets are limiting to $i<25$ mag, so that we are simulating photo-$z$ for samples of ``good" galaxies. Furthermore, in our implementation of the CMNN algorithm, galaxies are required to have at least 3 colors or else a photo-$z$ estimate is not attempted. Thus, our results represent the photo-$z$ quality of a ``gold'' sample of galaxies for cosmological analyses, and do not reflect any impact on the survey area or number of galaxies available for a cosmological analyses -- only the photo-$z$ quality of the ``best'' observed galaxies.


% % % % % % % % % % % % % % % % % % % % % % % % % % % % %
\subsection{Analysis Methodology}\label{ssec:pz_exp_meth}

We take these blurbs from G18 to describe our statistical measures of photo-$z$ quality and the common plot styles that we will use to represent our results.

{\bf Statistical Measures --} Where $z_{\rm true}$ is the ``true" catalog redshift and $z_{\rm phot}$ is the photo-$z$, the error is $\Delta z_{(1+z)} = (z_{\rm true} - z_{\rm phot})/(1+z_{\rm phot})$. Including a factor of $(1+z)$ in the denominator acts to compensate for larger uncertainties at high-$z$. For all of our results we calculate the robust standard deviation in $\Delta z_{(1+z)}$ as the FWHM of the interquartile range (IQR) divided by $1.349$ ($\sigma_{\rm IQR}$) and the robust bias as the mean value of $\Delta z_{(1+z)}$ in the IQR ($\overline{\Delta z_{\rm(1+z), IQR}}$). We reject catastrophic outliers ($|z_{\rm spec}-z_{\rm phot}| > 1.5$) from the IQR before calculating the standard deviation and bias, {\em which is different from our analyses in past work} such as \cite{2018AJ....155....1G}. We bootstrap our uncertainties on these statistical measures by randomly drawing a subsets and recalculating the statistics 1000 times. Outlier galaxies are identified as those with $\Delta z_{(1+z)} > 3\sigma_{\rm IQR}$ or $\Delta z_{(1+z)} > 0.06$, whichever is {\it larger}, where $\sigma_{\rm IQR}$ is calculated from all galaxies in $0.3 \leq z_{\rm phot} \leq 3.0$ (i.e., outliers are defined globally). The fraction of outlier galaxies that we calculate {\it includes} catastrophic outliers (as defined above).

{\bf Plot Styles --} To visualize our photo-$z$ results we create plots that compare the true $vs.$ photometric redshifts, in which outlier galaxies are typically colored red and a solid line of $z_{\rm true} = z_{\rm phot}$ is drawn to guide the eye. These plots are useful to obtain a global sense of the photo-$z$ quality and the structure in the outliers positions, especially the features that are perpendicular to the $z_{\rm true} = z_{\rm phot}$, which represent photo-$z$ degeneracies caused by the Balmer break passing between filters. We also plot the robust standard deviation, robust bias, and fraction of outliers as a function of $z_{\rm phot}$, typically as a way to directly compare the bulk photo-$z$ results of experiments in which the simulated galaxy photometry has been altered in some way. 

